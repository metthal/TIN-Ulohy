\documentclass[12pt]{article}

\usepackage[margin=1in]{geometry} 
\usepackage{amsmath,amsthm,amssymb,amsfonts}
\usepackage[slovak]{babel}
\usepackage[utf8]{inputenc}
\usepackage{enumitem}
\usepackage{ifthen}
\usepackage{tikz}

\usetikzlibrary{arrows,automata,calc,positioning}

\newcommand{\task}[2]{\par \noindent \textbf{{#1}.} \hspace{3pt} #2 \vspace{10pt}}
\newcommand{\solution}{\vspace{10pt}}
\newcommand{\pipesep}{\hspace{3pt} \vert \hspace{3pt}}

\newenvironment{subtasklist}[0]{\begin{enumerate}[label=(\alph*)]}{\end{enumerate}}
\newenvironment{mysolution}[1]{
    \par \textbf{Riešenie} \newline
    \ifthenelse{\equal{#1}{subtasks}}{\begin{enumerate}[label=(\alph*)]}
            {\begin{enumerate}[label={}] \item}
}{\end{enumerate} \newpage}
\newcommand{\subtask}{\item}
 
\begin{document}
 
\title{TIN 2015/2016: Úloha 1}
\author{Marek Milkovič}
\maketitle

\task{1}{Uvažujte jazyk $L_{1} = \{a^{i}b^{j}c^{i}d^{k} \pipesep i,j,k \ge 0\}$.
    \begin{subtasklist}
            \subtask Zostavte gramatiku $G_{1}$ takú, že $L(G_{1}) = L_{1}$.
            \subtask Akého typu (podľa Chomského hierachie jazykov) je $G_{1}$ a akého typu
            je $L_{1}$? Môžu sa tieto typy všeobecne líšiť? Svoje tvrdenie zdôvodnite
            (formálny dôkaz nie je požadovaný).
    \end{subtasklist}}

\begin{mysolution}{subtasks}
\subtask % (a)
\begin{flalign*}
    G_{1} = (&N, \Sigma, P, S) &\\[5pt]
    N &= \{S, A, B, D\} &\\
    \Sigma &= \{\texttt{a}, \texttt{b}, \texttt{c}\} &\\
    P&: \\
    &S \to A D &\\
    &A \to \texttt{a} A \texttt{c} \pipesep B \pipesep \varepsilon &\\
    &B \to \texttt{b} B \pipesep \varepsilon &\\
    &D \to \texttt{d} D \pipesep \varepsilon &
\end{flalign*}

\subtask % (b)
Gramatika $G_{1}$ je typu 2 z dôvodu tvaru prepisovacích pravidiel.
Tie odpovedajú tvaru $A \to \alpha$ kde $A \in N, \alpha \in (N \cup \Sigma)^{*}$.

Jazyk $L_{1}$ môže byť typu 2 či 3. To vyplýva z toho, že gramatika typu $n$ môže generovať jazyk
typu $n$ a vyšší. Je teda potrebné dokázať, o aký jazyk sa jedná pomocou pumping lemma pre
regulárne jazyky. Predpokladajme, že $L_{1}$ je nekonečný regulárny jazyk (typu~3).
Potom $\exists p > 0$ také, že platí
\begin{equation*}
    w \in L_{1} \land \vert w \vert \ge p \Rightarrow
    w = xyz \land y \ne \varepsilon
        \land \vert xy \vert \le p
        \land xy^{i}z \in L_{1}\ \text{pre}\ i \ge 0
\end{equation*}
Nech reťazec $w$ je generovaný ako $a^{p}b^{j}c^{p}d^{k}$. Pokiaľ platí, že $\vert xy \vert \le p$,
potom môžeme povedať, že reťazec $xy$ pozostáva len zo znakov $a$. Môžeme teda zapísať, že
$xy^{i}z = a^{p + (i - 1)\vert y \vert}b^{j}c^{p}d^{k}$. Pre $i = 0$ teda platí, že
$xy^{0}z = a^{p - \vert y \vert}b^{j}c^{p}d^{k}$. Avšak $xy^{0}z \not\in L_{1}$ pretože
$\#_{a} \neq \#_{c}$. Dospeli sme ku sporu. Pôvodný predpoklad, že jazyk $L_{1}$ je regulárny
(typu~3) je teda taktiež neplatný. Jazyk $L_{1}$ teda musí byť typu 2.

Typ gramatiky a typ jazyka sa môže všeobecne líšiť. Z definície tried Chomského hierarchie jazykov plynie

\begin{equation*}
    \mathcal{L}_{3} \subseteq \mathcal{L}_{2} \subseteq \mathcal{L}_{1} \subseteq \mathcal{L}_{0}
\end{equation*}

kde $\mathcal{L}_{i}$ je trieda všetkých jazykov typu $i$. Gramatiky typu $i$ sú tým pádom
schopné generovať jazyky typu $i$ a vyššie, pretože sa jedná len o podmnožiny z triedy
jazykov typu $i$.
\end{mysolution}

\task{2}{Uvažujte regulárny výraz $r_{2} = (abc)^{*}(a + \varepsilon)(abc)^{*}$.
    \begin{subtasklist}
        \subtask Preveďte $r_{2}$ algoritmicky na redukovaný deterministický konečný automat
        $M_{2}$ (tj. RV~$\to$ RKA $\to$ DKA $\to$ redukovaný DKA), prijímajúci jazyk
        popísaný výrazom $r_{2}$.
        \subtask Pre jazyk $L(M_{2})$ určte počet tried ekvivalencie relácie $\sim_{L}$
        (viz. Myhill-Nerodova veta) a vypíšte tieto triedy. Jednotlivé triedy môžete popísať
        napríklad konečným automatom, ktorý prijíma všetky slová patriace do danej triedy.
    \end{subtasklist}
}
\begin{mysolution}{subtasks}
    \subtask Budeme postupovať podľa Algoritmu 3.7 zo študíjnej opory. Najskôr rozložíme regulárny výraz
    $r_{2}$ na primitívne zložky. Tie predstavujú regulárne výrazy $a, b, c, \varepsilon$.

    Podľa Algoritmu 3.7 2ab zostrojíme konečné automaty pre tieto primitívne regulárne výrazy.
    \begin{figure}[!h]
        \centering
        \begin{minipage}[b]{0.4\textwidth}
            \centering
            \begin{tikzpicture}[node distance = 3cm, ->, >=stealth, line width=1pt]
                \node[state, initial, initial text=] (s) {$s_{\varepsilon}$};
                \node[state, accepting, right of = s] (f) {$f_{\varepsilon}$};

                \path (s) edge node[above]{$\varepsilon$} (f);
            \end{tikzpicture}
            \caption{KA ekvivalentný RV $\varepsilon$.}
        \end{minipage}
        \begin{minipage}[b]{0.4\textwidth}
            \centering
            \begin{tikzpicture}[node distance = 3cm, ->, >=stealth, line width=1pt]
                \node[state, initial, initial text=] (s) {$s_{a}$};
                \node[state, accepting, right of = s] (f) {$f_{a}$};

                \path (s) edge node[above]{$a$} (f);
            \end{tikzpicture}
            \caption{KA ekvivalentný RV $a$.}
        \end{minipage}
    \end{figure}
    \begin{figure}[!h]
        \centering
        \begin{minipage}[b]{0.4\textwidth}
            \centering
            \begin{tikzpicture}[node distance = 3cm, ->, >=stealth, line width=1pt]
                \node[state, initial, initial text=] (s) {$s_{b}$};
                \node[state, accepting, right of = s] (f) {$f_{b}$};

                \path (s) edge node[above]{$b$} (f);
            \end{tikzpicture}
            \caption{KA ekvivalentný RV $b$.}
        \end{minipage}
        \begin{minipage}[b]{0.4\textwidth}
            \centering
            \begin{tikzpicture}[node distance = 3cm, ->, >=stealth, line width=1pt]
                \node[state, initial, initial text=] (s) {$s_{c}$};
                \node[state, accepting, right of = s] (f) {$f_{c}$};

                \path (s) edge node[above]{$c$} (f);
            \end{tikzpicture}
            \caption{KA ekvivalentný RV $c$.}
        \end{minipage}
    \end{figure}

    Z konečných automatov pre primitívne regulárne výrazy môžeme skladať konečné automaty pre zložitejšie
    regulárne výrazy. Budeme postupovať podľa zátvoriek v regulárnom výraze $r_{2}$. Najskôr vyriešime regulárny výraz
    v prvej zátvorke $(abc)^{*}$. Ten sa skladá z konkatenácie, ktorá je následne iterovaná. Konkatenáciu $abc$, alebo
    aj $a.b.c$ zostrojíme pomocou Algoritmu 3.7 2d, ii.
    \begin{figure}[!h]
        \centering
        \begin{tikzpicture}[node distance = 2cm, ->, >=stealth, line width=1pt]
            \node[state, initial, initial text=] (sa) {$s_{a}$};
            \node[state, right of = sa] (fa) {$f_{a}$};
            \node[state, right of = fa] (sb) {$s_{b}$};
            \node[state, right of = sb] (fb) {$f_{b}$};
            \node[state, right of = fb] (sc) {$s_{c}$};
            \node[state, accepting, right of = sc] (fc) {$f_{c}$};

            \path (sa) edge node[above]{$a$} (fa);
            \path (fa) edge node[above]{$\varepsilon$} (sb);
            \path (sb) edge node[above]{$b$} (fb);
            \path (fb) edge node[above]{$\varepsilon$} (sc);
            \path (sc) edge node[above]{$c$} (fc);
        \end{tikzpicture}
        \caption{KA ekvivalentný RV $abc$.}
    \end{figure}

    Regulárny výraz $(abc)^{*}$ zostrojíme pomocou Algortimu 3.7 2d, iii.
    \begin{figure}[!h]
        \centering
        \begin{tikzpicture}[node distance = 1.7cm, ->, >=stealth, line width=1pt]
            \node[state, initial, initial text=] (sx) {$s_{x}$};
            \node[state, right of = sx] (sa) {$s_{a}$};
            \node[state, right of = sa] (fa) {$f_{a}$};
            \node[state, right of = fa] (sb) {$s_{b}$};
            \node[state, right of = sb] (fb) {$f_{b}$};
            \node[state, right of = fb] (sc) {$s_{c}$};
            \node[state, right of = sc] (fc) {$f_{c}$};
            \node[state, accepting, right of = fc] (fx) {$f_{x}$};

            \path (sx) edge node[above]{$\varepsilon$} (sa);
            \path (sx) edge[bend right=40] node[above]{$\varepsilon$} (fx);
            \path (sa) edge node[above]{$a$} (fa);
            \path (fa) edge node[above]{$\varepsilon$} (sb);
            \path (sb) edge node[above]{$b$} (fb);
            \path (fb) edge node[above]{$\varepsilon$} (sc);
            \path (sc) edge node[above]{$c$} (fc);
            \path (fc) edge node[above]{$\varepsilon$} (fx);
            \path (fc) edge[bend right=40] node[below]{$\varepsilon$} (sa);
        \end{tikzpicture}
        \caption{KA ekvivalentný RV $(abc)^{*}$.}
    \end{figure}

    \newpage
    Ďalšia zátvorka odpovedá regulárnemu výrazu $(a + \varepsilon)$. Pre tento regulárny výraz zostrojíme
    konečný automat pomocou Algoritmu 3.7 2d, i.

    \begin{figure}[!h]
        \centering
        \begin{tikzpicture}[node distance = 2.5cm, ->, >=stealth, line width=1pt]
            \node[state, initial, initial text=] (sy) {$s_{y}$};
            \node[state, above right = 0.5cm and 1.7cm of sy] (sa) {$s_{a}$};
            \node[state, right of = sa] (fa) {$f_{a}$};
            \node[state, below right = 0.5cm and 1.7cm of sy] (se) {$s_{\varepsilon}$};
            \node[state, right of = se] (fe) {$f_{\varepsilon}$};
            \node[state, accepting, right = 6.5cm of sy] (fy) {$f_{y}$};

            \path (sy) edge node[above]{$\varepsilon$} (sa);
            \path (sy) edge node[below]{$\varepsilon$} (se);
            \path (sa) edge node[above]{$a$} (fa);
            \path (se) edge node[above]{$\varepsilon$} (fe);
            \path (fa) edge node[above]{$\varepsilon$} (fy);
            \path (fe) edge node[below]{$\varepsilon$} (fy);
        \end{tikzpicture}
        \caption{KA ekvivalentný RV $(a + \varepsilon)$.}
    \end{figure}

    Tretia zátvorka odpovedá presne prvej zátvorke. Konečný automat pre ňu teda už máme zkonštruovaný.
    Ostáva vytvoriť konečný automat predstavujúci konkatenáciu vzniknutých konečných automatov.

    Z dôvodu toho, aby boli množiny stavov jednotlivých konečných automatov disjunktné, tak sa zaviedlo
    premenovanie stavov, čo nijak nezmenilo prijímaný jazyk. Výsledný KA je rozšírený KA (RKA) nakoľko
    obsahuje $\varepsilon$ prechody.

    \newpage
    \begin{figure}[!h]
        \centering
        \begin{tikzpicture}[node distance = 1.8cm, ->, >=stealth, line width=1pt, scale=0.72, every node/.style={transform shape}]
            % (abc)*
            \node[state, initial, initial text=] (a) {$A$};
            \node[state, right of = a] (b) {$B$};
            \node[state, right of = b] (c) {$C$};
            \node[state, right of = c] (d) {$D$};
            \node[state, right of = d] (e) {$E$};
            \node[state, right of = e] (f) {$F$};
            \node[state, right of = f] (g) {$G$};
            \node[state, right of = g] (h) {$H$};
            % (a + eps)
            \node[state, below right = 3cm and 0.2cm of f] (i) {$I$};
            \node[state, above left = 0.5cm and 1.7cm of i] (j) {$J$};
            \node[state, left = 1.7cm of j] (k) {$K$};
            \node[state, below left = 0.5cm and 1.7cm of i] (l) {$L$};
            \node[state, left = 1.7cm of l] (m) {$M$};
            \node[state, left = 6.5cm of i] (n) {$N$};
            % (abc)*
            \node[state, below = 6.2cm of a] (o) {$O$};
            \node[state, right of = o] (p) {$P$};
            \node[state, right of = p] (q) {$Q$};
            \node[state, right of = q] (r) {$R$};
            \node[state, right of = r] (s) {$S$};
            \node[state, right of = s] (t) {$T$};
            \node[state, right of = t] (u) {$U$};
            \node[state, accepting, right of = u] (v) {$V$};

            \path (a) edge node[above]{$\varepsilon$} (b);
            \path (a) edge[bend left=40] node[above]{$\varepsilon$} (h);
            \path (b) edge node[above]{$a$} (c);
            \path (c) edge node[above]{$\varepsilon$} (d);
            \path (d) edge node[above]{$b$} (e);
            \path (e) edge node[above]{$\varepsilon$} (f);
            \path (f) edge node[above]{$c$} (g);
            \path (g) edge node[above]{$\varepsilon$} (h);
            \path (g) edge[bend right=40] node[below]{$\varepsilon$} (b);

            \path (i) edge node[above]{$\varepsilon$} (j);
            \path (i) edge node[below]{$\varepsilon$} (l);
            \path (j) edge node[above]{$a$} (k);
            \path (l) edge node[above]{$\varepsilon$} (m);
            \path (k) edge node[above]{$\varepsilon$} (n);
            \path (m) edge node[below]{$\varepsilon$} (n);

            \path (o) edge node[above]{$\varepsilon$} (p);
            \path (o) edge[bend right=40] node[below]{$\varepsilon$} (v);
            \path (p) edge node[above]{$a$} (q);
            \path (q) edge node[above]{$\varepsilon$} (r);
            \path (r) edge node[above]{$b$} (s);
            \path (s) edge node[above]{$\varepsilon$} (t);
            \path (t) edge node[above]{$c$} (u);
            \path (u) edge node[above]{$\varepsilon$} (v);
            \path (u) edge[bend left=40] node[above]{$\varepsilon$} (p);

            \path (h) edge[bend left=30] node[right]{$\varepsilon$} (i);
            \path (n) edge[bend right=30] node[right]{$\varepsilon$} (o);
        \end{tikzpicture}
        \caption{KA ekvivalentný RV $(abc)^{*}(a + \varepsilon)(abc)^{*}$.}
    \end{figure}

    \subtask TODO
\end{mysolution}

\task{3}{Uvažujte NKA $M_{3}$ nad abecedou $\Sigma = \{a, b, c\}$ z Obrázku 1: \\
    \begin{figure}[!h]
        \centering
        \begin{tikzpicture}[node distance = 3cm, ->, >=stealth, line width=1pt]
            \node[state, initial, initial text=] (X) {};
            \node[state, right of = X] (Y) {};
            \node[state, accepting, right of = Y] (Z) {};

            \path (X) edge node[above]{a} (Y);
            \path (Y) edge node[above]{b} (Z);
            \path (Y) edge[loop above] node[above]{c} (Y);
            \path (Y) edge[bend left] node[below]{c} (X);
            \path (Z) edge[bend left=60] node[above]{b} (X);
            \path (Z) edge[bend right=60] node[above]{a} (Y);
        \end{tikzpicture}
    \caption{NKA $M_{3}$}
    \end{figure}

    K tomuto automatu zostrojte sústavu rovníc nad regulárnymi výrazmi a jej riešením zostavte
ekvivalentný regulárny výraz.}
\begin{mysolution}

\vspace{-10pt}
Jednotlivé regulárne výrazy každého stavu si označíme $X, Y\text{a}\ Z$ podľa toho ako sú
stavy znázornené v diagrame na Obrázku 1 z ľava do prava. Pre nájdenie riešenia, teda
ekvivalntného regulárneho výrazu značiaceho rovnaký jazyk, ako je príjmaný jazyk $L(M_{3})$,
je potrebné určiť regulárny výraz $X$ predstavujúci regulárny výraz pre počiatočný stav.
\begin{flalign*}
    X =&\ aY & \\
    Y =&\ cX + cY + bZ & \\
    Z =&\ bX + aY + \varepsilon &
\end{flalign*}
Dosadíme za $Z$ do rovnice číslo 2. Dostávame sústavu
\begin{flalign*}
    X =&\ aY & \\
    Y =&\ cX + cY + bZ & \\
      =&\ cX + cY + b(bX + aY + \varepsilon) & \\
      =&\ cX + cY + bbX + baY + b & \\
      =&\ (c + ba)Y + (c + bb)X + b & \\
      =&\ (c + ba)^{*}((c + bb)X + b) &
\end{flalign*}
Rovnicu číslo 2 sme upravili do tvaru najmenšieho pevného bodu podľa Vety 2.3, ktorá hovorí, že
najmenším pevným bodom rovnice v tvare $X = pX + q$ je $X = p^{*}q$.

Za $Y$ následne dosadíme do rovnice číslo 1 a získame tak 1 rovnicu, ktorú znova upravíme do
tvaru najmenšieho pevného bodu.
\begin{flalign*}
    X =&\ aY & \\
      =&\ a(c + ba)^{*}((c + bb)X + b) & \\
      =&\ a(c + ba)^{*}(c + bb)X + a(c + ba)^{*}b & \\
      =&\ (a(c + ba)^{*}(c + bb))^{*}a(c + ba)^{*}b &
\end{flalign*}
Regulárny výraz $X$ značí rovnaký jazyk, ako je príjmaný jazyk $L(M_{3})$.
\end{mysolution}

\task{4}{Pre dva formálne jazyky $L_{1}$ a $L_{2}$ definujeme operáciu $restrict$ nasledovne:
    \begin{equation*}
        restrict(L_{1},L_{2}) = \{ w | w \in L_{1} \land \exists w' \in L_{2} : 
            \vert w \vert = \vert w' \vert \}
    \end{equation*}

    Navrhnite a \textit{formálne popíšte} algoritmus, ktorý má na vstupe dva konečné automaty
$M_{1} = (Q_{1}, \Sigma_{1}, \delta_{1}, q^{0}_{1}, F_{1})$ a
$M_{2} = (Q_{2}, \Sigma_{2}, \delta_{2}, q^{0}_{2}, F_{2})$ (môžu byť aj nedeterministické)
a ktorého výstupom bude konečný automat $M_{restrict}$ taký, že
$L(M_{restrict}) = restrict(L(M_{1}), L(M_{2}))$.}
\begin{mysolution}

TODO
\end{mysolution}

\task{5}{Majme jazyk $L = \{ a^{i}b^{2i}c^{j} \pipesep i > 0 \land i < j < 2i \}$.
    Je jazyk $L$ regulárny? Dokážte alebo vyvráťte.}
\begin{mysolution}

TODO pumping lemma
\end{mysolution}

\task{6}{Uvažujme algebru regulárnych množín ($A_{RM}$) nad abecedou $\Sigma$.
    \begin{subtasklist}
        \subtask Ukážte, že pre $A_{RM}$ platí nasledujúci vzťah
            $\{\varepsilon\} \cup A.A^{*} = A^{*}$,
        kde $A$ je ľubovolná regulárna množina. Nepoužívajte fakt, že $A_{RM}$ je
        Kleenova algebra.
        \subtask Určite, či je $\subseteq$ \textit{totálne usporiadanie} v $A_{RM}$. Teda, že pre
        ľubovolné dve regulárne množiny A a B platí vždy aspoň jedna z nerovností
        $A \subseteq B$ alebo $B \subseteq A$. Svoje tvrdenie dokážte.
    \end{subtasklist}
}
\begin{mysolution}{subtasks}

    \subtask TODO
    \subtask TODO
\end{mysolution}

\end{document}
