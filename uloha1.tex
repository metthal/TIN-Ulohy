\documentclass[12pt]{article}

\usepackage[margin=1in]{geometry} 
\usepackage{amsmath,amsthm,amssymb,amsfonts}
\usepackage[slovak]{babel}
\usepackage[utf8]{inputenc}
\usepackage{enumitem}
\usepackage{ifthen}
\usepackage{tikz}
\usepackage{tikz-qtree}
\usepackage{caption}
\usepackage{background}

\usetikzlibrary{arrows,automata,calc,positioning}

\newcommand{\task}[2]{\par \noindent \textbf{{#1}.} \hspace{3pt} #2 \vspace{10pt}}
\newcommand{\solution}{\vspace{10pt}}
\newcommand{\pipesep}{\hspace{3pt} \vert \hspace{3pt}}

\newenvironment{subtasklist}[0]{\begin{enumerate}[label=(\alph*)]}{\end{enumerate}}
\newenvironment{mysolution}[1]{
    \par \textbf{Riešenie} \newline
    \ifthenelse{\equal{#1}{subtasks}}{\begin{enumerate}[label=(\alph*)]}
            {\begin{enumerate}[label={}] \item}
}{\end{enumerate} \newpage}
\newcommand{\subtask}{\item}

\SetBgContents{
        \parbox{0.4\textwidth}{
            \raggedleft
                Marek Milkovič \\
                \texttt{xmilko01@stud.fit.vutbr.cz}
        }
}
\SetBgScale{0.75}
\SetBgOpacity{1}
\SetBgAngle{0}
\SetBgColor{black}
\SetBgPosition{current page.north east}
\SetBgVshift{-0.8cm}
\SetBgHshift{-3.5cm}
 
\begin{document}
 
\title{TIN 2015/2016: Úloha 1}
\author{Marek Milkovič \\ \small\texttt{xmilko01@stud.fit.vutbr.cz}}
\maketitle

\task{1}{Uvažujte jazyk $L_{1} = \{a^{i}b^{j}c^{i}d^{k} \pipesep i,j,k \ge 0\}$.
    \begin{subtasklist}
            \subtask Zostavte gramatiku $G_{1}$ takú, že $L(G_{1}) = L_{1}$.
            \subtask Akého typu (podľa Chomského hierachie jazykov) je $G_{1}$ a akého typu
            je $L_{1}$? Môžu sa tieto typy všeobecne líšiť? Svoje tvrdenie zdôvodnite
            (formálny dôkaz nie je požadovaný).
    \end{subtasklist}}

\begin{mysolution}{subtasks}
\subtask % (a)
\begin{flalign*}
    G_{1} = (&N, \Sigma, P, S) &\\[5pt]
    N &= \{S, A, B, D\} &\\
    \Sigma &= \{a, b, c\} &\\
    P&: \\
    &S \to A D &\\
    &A \to a A c \pipesep B \pipesep \varepsilon &\\
    &B \to b B \pipesep \varepsilon &\\
    &D \to d D \pipesep \varepsilon &
\end{flalign*}

\subtask % (b)
Gramatika $G_{1}$ je typu 2 z dôvodu tvaru prepisovacích pravidiel.
Tie odpovedajú tvaru $A \to \alpha$ kde $A \in N, \alpha \in (N \cup \Sigma)^{*}$.

Jazyk $L_{1}$ môže byť typu 2 či 3. To vyplýva z toho, že gramatika typu $n$ môže generovať jazyk
typu $n$ a vyšší. Je teda potrebné dokázať, o aký jazyk sa jedná pomocou pumping lemma pre
regulárne jazyky. Predpokladajme, že $L_{1}$ je nekonečný regulárny jazyk (typu~3).
Potom $\exists p > 0$ také, že platí
\begin{equation*}
    w \in L_{1} \land \vert w \vert \ge p \Rightarrow
    w = xyz \land y \ne \varepsilon
        \land \vert xy \vert \le p
        \land xy^{i}z \in L_{1}\ \text{pre}\ i \ge 0
\end{equation*}
Nech reťazec $w$ je generovaný ako $a^{p}b^{j}c^{p}d^{k}$. Pokiaľ platí, že $\vert xy \vert \le p$,
potom môžeme povedať, že reťazec $xy$ pozostáva len zo znakov $a$. Môžeme teda zapísať, že
$xy^{i}z = a^{p + (i - 1)\vert y \vert}b^{j}c^{p}d^{k}$. Pre $i = 0$ teda platí, že
$xy^{0}z = a^{p - \vert y \vert}b^{j}c^{p}d^{k}$. Avšak $xy^{0}z \not\in L_{1}$ pretože
$\#_{a} \neq \#_{c}$. Dospeli sme ku sporu. Pôvodný predpoklad, že jazyk $L_{1}$ je regulárny
(typu~3) je teda taktiež neplatný. Jazyk $L_{1}$ teda musí byť typu 2.

Typ gramatiky a typ jazyka sa môže všeobecne líšiť. Z definície tried Chomského hierarchie jazykov plynie

\begin{equation*}
    \mathcal{L}_{3} \subseteq \mathcal{L}_{2} \subseteq \mathcal{L}_{1} \subseteq \mathcal{L}_{0}
\end{equation*}

kde $\mathcal{L}_{i}$ je trieda všetkých jazykov typu $i$. Gramatiky typu $i$ sú tým pádom
schopné generovať jazyky typu $i$ a vyššie, pretože sa jedná len o podmnožiny z triedy
jazykov typu $i$.
\end{mysolution}

\task{2}{Uvažujte regulárny výraz $r_{2} = (abc)^{*}(a + \varepsilon)(abc)^{*}$.
    \begin{subtasklist}
        \subtask Preveďte $r_{2}$ algoritmicky na redukovaný deterministický konečný automat
        $M_{2}$ (tj. RV~$\to$ RKA $\to$ DKA $\to$ redukovaný DKA), prijímajúci jazyk
        popísaný výrazom $r_{2}$.
        \subtask Pre jazyk $L(M_{2})$ určte počet tried ekvivalencie relácie $\sim_{L}$
        (viz. Myhill-Nerodova veta) a vypíšte tieto triedy. Jednotlivé triedy môžete popísať
        napríklad konečným automatom, ktorý prijíma všetky slová patriace do danej triedy.
    \end{subtasklist}
}
\begin{mysolution}{subtasks}
    \subtask Budeme postupovať podľa Algoritmu 3.7 zo študíjnej opory. Najskôr rozložíme regulárny výraz
    $r_{2}$ na primitívne zložky. Rozklad znázorníme stromom.

    \begin{figure}[!h]
        \centering
        \begin{tikzpicture}[level 1/.style={sibling distance = 3.75cm}, level 2/.style={sibling distance = 1.5cm}]
            \node {$r_{2}$}
                child
                {
                    node {$r_{M}$} edge from parent
                        child
                        {
                            node {$r_{G}$} edge from parent
                                child
                                {
                                    node {$r_{F}$} edge from parent
                                        child { node {$($} edge from parent }
                                        child
                                        {
                                            node {$r_{E}$} edge from parent
                                                child
                                                {
                                                    node {$r_{C}$} edge from parent
                                                        child
                                                        {
                                                            node {$r_{A}$} edge from parent
                                                                child { node {$a$} edge from parent };
                                                        }
                                                        child { node {\LARGE$\cdot$} edge from parent }
                                                        child
                                                        {
                                                            node {$r_{B}$} edge from parent
                                                                child { node {$b$} edge from parent };
                                                        };
                                                }
                                                child { node {\LARGE$\cdot$} edge from parent }
                                                child
                                                {
                                                    node {$r_{D}$} edge from parent
                                                        child { node {$c$} edge from parent };
                                                };
                                        }
                                        child { node {$)$} edge from parent };
                                }
                                child { node {$\ast$} edge from parent };
                        }
                        child { node {\LARGE$\cdot$} edge from parent }
                        child
                        {
                            node {$r_{L}$} edge from parent
                                child { node {$($} edge from parent }
                                child
                                {
                                    node {$r_{K}$} edge from parent
                                        child
                                        {
                                            node {$r_{I}$} edge from parent
                                                child { node {$a$} edge from parent };
                                        }
                                        child { node {$+$} edge from parent }
                                        child
                                        {
                                            node {$r_{J}$} edge from parent
                                                child {node {$\varepsilon$} edge from parent };
                                        };
                                }
                                child { node {$)$} edge from parent };
                        };
                }
                child { node {\LARGE$\cdot$} edge from parent }
                child
                {
                    node {$r_{T}$} edge from parent
                        child
                        {
                            node {$r_{S}$} edge from parent
                                child { node {$($} edge from parent }
                                child
                                {
                                    node {$r_{R}$} edge from parent
                                        child
                                        {
                                            node {$r_{P}$} edge from parent
                                                child
                                                {
                                                    node {$r_{N}$} edge from parent
                                                        child { node {$a$} edge from parent };
                                                }
                                                child { node {\LARGE$\cdot$} edge from parent }
                                                child
                                                {
                                                    node {$r_{O}$} edge from parent
                                                        child { node {$b$} edge from parent };
                                                };
                                        }
                                        child { node {\LARGE$\cdot$} edge from parent }
                                        child
                                        {
                                            node {$r_{Q}$} edge from parent
                                                child { node {$c$} edge from parent };
                                        };
                                }
                                child { node {$)$} edge from parent };
                        }
                        child { node {$\ast$} edge from parent };
                };
        \end{tikzpicture}
        \caption{Rozklad regulárneho výrazu $r_{2}$.}
    \end{figure}

    Ďalej zostrojíme konečný automat $M_{x}$, kde $x$ je index odpovedajúci indexu regulárneho výrazu $r_{x}$,
    pre každý regulárny výraz rozkladu postupujúc od listovej úrovne až ku koreňu stromu,
    teda samotnému regulárnemu výrazu $r_{2}$.

    \newpage
    \begin{figure}[!h]
        \centering
        \begin{minipage}[b]{0.45\textwidth}
            \centering
            \captionsetup{justification=centering}
            \begin{tikzpicture}[node distance = 3cm, ->, >=stealth, line width=1pt]
                \node[state, initial, initial text=] (s) {};
                \node[state, accepting, right of = s] (f) {};

                \path (s) edge node[above]{$\varepsilon$} (f);
            \end{tikzpicture}
            \caption{RKA $M_{J}$ ekvivalentný\\RV $r_{J}$.}
        \end{minipage}
        \begin{minipage}[b]{0.45\textwidth}
            \centering
            \captionsetup{justification=centering}
            \begin{tikzpicture}[node distance = 3cm, ->, >=stealth, line width=1pt]
                \node[state, initial, initial text=] (s) {};
                \node[state, accepting, right of = s] (f) {};

                \path (s) edge node[above]{$a$} (f);
            \end{tikzpicture}
            \caption{RKA $M_{A}, M_{I}, M_{N}$ ekvivalentný\\RV $r_{A}, r_{I}, r_{N}$.}
        \end{minipage}
    \end{figure}
    \begin{figure}[!h]
        \centering
        \begin{minipage}[b]{0.45\textwidth}
            \centering
            \captionsetup{justification=centering}
            \begin{tikzpicture}[node distance = 3cm, ->, >=stealth, line width=1pt]
                \node[state, initial, initial text=] (s) {};
                \node[state, accepting, right of = s] (f) {};

                \path (s) edge node[above]{$b$} (f);
            \end{tikzpicture}
            \caption{RKA $M_{B}, M_{O}$ ekvivalentný\\RV $r_{B}, r_{O}$.}
        \end{minipage}
        \begin{minipage}[b]{0.45\textwidth}
            \centering
            \captionsetup{justification=centering}
            \begin{tikzpicture}[node distance = 3cm, ->, >=stealth, line width=1pt]
                \node[state, initial, initial text=] (s) {};
                \node[state, accepting, right of = s] (f) {};

                \path (s) edge node[above]{$c$} (f);
            \end{tikzpicture}
            \caption{RKA $M_{D}, M_{Q}$ ekvivalentný\\RV $r_{D}, r_{Q}$.}
        \end{minipage}
    \end{figure}

    \begin{figure}[!h]
        \centering
        \begin{tikzpicture}[node distance = 2cm, ->, >=stealth, line width=1pt]
            \node[state, initial, initial text=] (sa) {};
            \node[state, right of = sa] (fa) {};
            \node[state, right of = fa] (sb) {};
            \node[state, accepting, right of = sb] (fb) {};

            \path (sa) edge node[above]{$a$} (fa);
            \path (fa) edge node[above]{$\varepsilon$} (sb);
            \path (sb) edge node[above]{$b$} (fb);
        \end{tikzpicture}
        \caption{RKA $M_{C}, M_{P}$ ekvivalentný RV $r_{C}, r_{P}$.}
    \end{figure}

    \begin{figure}[!h]
        \centering
        \begin{tikzpicture}[node distance = 2cm, ->, >=stealth, line width=1pt, scale=0.8, every node/.style={transform shape}]
            \node[state, initial, initial text=] (sa) {};
            \node[state, right of = sa] (fa) {};
            \node[state, right of = fa] (sb) {};
            \node[state, right of = sb] (fb) {};
            \node[state, right of = fb] (sc) {};
            \node[state, accepting, right of = sc] (fc) {};

            \path (sa) edge node[above]{$a$} (fa);
            \path (fa) edge node[above]{$\varepsilon$} (sb);
            \path (sb) edge node[above]{$b$} (fb);
            \path (fb) edge node[above]{$\varepsilon$} (sc);
            \path (sc) edge node[above]{$c$} (fc);
        \end{tikzpicture}
        \caption{RKA $M_{E}, M_{F}, M_{R}, M_{S}$ ekvivalentný RV $r_{E}, r_{F}, r_{R}, r_{S}$.}
    \end{figure}

    \begin{figure}[!h]
        \centering
        \begin{tikzpicture}[node distance = 1.7cm, ->, >=stealth, line width=1pt]
            \node[state, initial, initial text=] (sx) {};
            \node[state, right of = sx] (sa) {};
            \node[state, right of = sa] (fa) {};
            \node[state, right of = fa] (sb) {};
            \node[state, right of = sb] (fb) {};
            \node[state, right of = fb] (sc) {};
            \node[state, right of = sc] (fc) {};
            \node[state, accepting, right of = fc] (fx) {};

            \path (sx) edge node[above]{$\varepsilon$} (sa);
            \path (sx) edge[bend right=40] node[above]{$\varepsilon$} (fx);
            \path (sa) edge node[above]{$a$} (fa);
            \path (fa) edge node[above]{$\varepsilon$} (sb);
            \path (sb) edge node[above]{$b$} (fb);
            \path (fb) edge node[above]{$\varepsilon$} (sc);
            \path (sc) edge node[above]{$c$} (fc);
            \path (fc) edge node[above]{$\varepsilon$} (fx);
            \path (fc) edge[bend right=40] node[below]{$\varepsilon$} (sa);
        \end{tikzpicture}
        \caption{RKA $M_{G}, M_{T}$ ekvivalentný RV $r_{G}, r_{T}$.}
    \end{figure}

    \newpage
    \begin{figure}[!h]
        \centering
        \begin{tikzpicture}[node distance = 2.5cm, ->, >=stealth, line width=1pt, scale=0.8, every node/.style={transform shape}]
            \node[state, initial, initial text=] (sy) {};
            \node[state, above right = 0.5cm and 1.7cm of sy] (sa) {};
            \node[state, right of = sa] (fa) {};
            \node[state, below right = 0.5cm and 1.7cm of sy] (se) {};
            \node[state, right of = se] (fe) {};
            \node[state, accepting, right = 6.5cm of sy] (fy) {};

            \path (sy) edge node[above]{$\varepsilon$} (sa);
            \path (sy) edge node[below]{$\varepsilon$} (se);
            \path (sa) edge node[above]{$a$} (fa);
            \path (se) edge node[above]{$\varepsilon$} (fe);
            \path (fa) edge node[above]{$\varepsilon$} (fy);
            \path (fe) edge node[below]{$\varepsilon$} (fy);
        \end{tikzpicture}
        \caption{RKA $M_{K}, M_{L}$ ekvivalentný RV $r_{K}, r_{L}$.}
    \end{figure}

    \begin{figure}[!h]
        \centering
        \begin{tikzpicture}[node distance = 1.8cm, ->, >=stealth, line width=1pt, scale=0.8, every node/.style={transform shape}]
            % (abc)*
            \node[state, initial, initial text=] (a) {};
            \node[state, right of = a] (b) {};
            \node[state, right of = b] (c) {};
            \node[state, right of = c] (d) {};
            \node[state, right of = d] (e) {};
            \node[state, right of = e] (f) {};
            \node[state, right of = f] (g) {};
            \node[state, right of = g] (h) {};
            % (a + eps)
            \node[state, below right = 2cm and 0.2cm of f] (i) {};
            \node[state, above left = 0.5cm and 1.7cm of i] (j) {};
            \node[state, left = 1.7cm of j] (k) {};
            \node[state, below left = 0.5cm and 1.7cm of i] (l) {};
            \node[state, left = 1.7cm of l] (m) {};
            \node[state, accepting, left = 6.5cm of i] (n) {};

            \path (a) edge node[above]{$\varepsilon$} (b);
            \path (a) edge[bend left=40] node[above]{$\varepsilon$} (h);
            \path (b) edge node[above]{$a$} (c);
            \path (c) edge node[above]{$\varepsilon$} (d);
            \path (d) edge node[above]{$b$} (e);
            \path (e) edge node[above]{$\varepsilon$} (f);
            \path (f) edge node[above]{$c$} (g);
            \path (g) edge node[above]{$\varepsilon$} (h);
            \path (g) edge[bend right=40] node[below]{$\varepsilon$} (b);

            \path (i) edge node[above]{$\varepsilon$} (j);
            \path (i) edge node[below]{$\varepsilon$} (l);
            \path (j) edge node[above]{$a$} (k);
            \path (l) edge node[above]{$\varepsilon$} (m);
            \path (k) edge node[above]{$\varepsilon$} (n);
            \path (m) edge node[below]{$\varepsilon$} (n);

            \path (h) edge[bend left=30] node[right]{$\varepsilon$} (i);
        \end{tikzpicture}
        \caption{RKA $M_{M}$ ekvivalentný RV $r_{M}$.}
    \end{figure}

    \begin{figure}[!h]
        \centering
        \begin{tikzpicture}[node distance = 1.8cm, ->, >=stealth, line width=1pt, scale=0.8, every node/.style={transform shape}]
            % (abc)*
            \node[state, initial, initial text=] (a) {$1$};
            \node[state, right of = a] (b) {$2$};
            \node[state, right of = b] (c) {$3$};
            \node[state, right of = c] (d) {$4$};
            \node[state, right of = d] (e) {$5$};
            \node[state, right of = e] (f) {$6$};
            \node[state, right of = f] (g) {$7$};
            \node[state, right of = g] (h) {$8$};
            % (a + eps)
            \node[state, below right = 2cm and 0.2cm of f] (i) {$9$};
            \node[state, above left = 0.5cm and 1.7cm of i] (j) {$10$};
            \node[state, left = 1.7cm of j] (k) {$11$};
            \node[state, below left = 0.5cm and 1.7cm of i] (l) {$12$};
            \node[state, left = 1.7cm of l] (m) {$13$};
            \node[state, left = 6.5cm of i] (n) {$14$};
            % (abc)*
            \node[state, below = 4.5cm of a] (o) {$15$};
            \node[state, right of = o] (p) {$16$};
            \node[state, right of = p] (q) {$17$};
            \node[state, right of = q] (r) {$18$};
            \node[state, right of = r] (s) {$19$};
            \node[state, right of = s] (t) {$20$};
            \node[state, right of = t] (u) {$21$};
            \node[state, accepting, right of = u] (v) {$22$};

            \path (a) edge node[above]{$\varepsilon$} (b);
            \path (a) edge[bend left=30] node[above]{$\varepsilon$} (h);
            \path (b) edge node[above]{$a$} (c);
            \path (c) edge node[above]{$\varepsilon$} (d);
            \path (d) edge node[above]{$b$} (e);
            \path (e) edge node[above]{$\varepsilon$} (f);
            \path (f) edge node[above]{$c$} (g);
            \path (g) edge node[above]{$\varepsilon$} (h);
            \path (g) edge[bend right=30] node[below]{$\varepsilon$} (b);

            \path (i) edge node[above]{$\varepsilon$} (j);
            \path (i) edge node[below]{$\varepsilon$} (l);
            \path (j) edge node[above]{$a$} (k);
            \path (l) edge node[above]{$\varepsilon$} (m);
            \path (k) edge node[above]{$\varepsilon$} (n);
            \path (m) edge node[below]{$\varepsilon$} (n);

            \path (o) edge node[above]{$\varepsilon$} (p);
            \path (o) edge[bend right=30] node[below]{$\varepsilon$} (v);
            \path (p) edge node[above]{$a$} (q);
            \path (q) edge node[above]{$\varepsilon$} (r);
            \path (r) edge node[above]{$b$} (s);
            \path (s) edge node[above]{$\varepsilon$} (t);
            \path (t) edge node[above]{$c$} (u);
            \path (u) edge node[above]{$\varepsilon$} (v);
            \path (u) edge[bend left=30] node[above]{$\varepsilon$} (p);

            \path (h) edge[bend left=30] node[right]{$\varepsilon$} (i);
            \path (n) edge[bend right=30] node[right]{$\varepsilon$} (o);
        \end{tikzpicture}
        \caption{RKA $M_{2}$ ekvivalentný RV $r_{2}$.}
    \end{figure}

    \newpage
    Jednotlivé stavy RKA $M_{2}$ boli pomenované prirodzenými číslami. Ďalej budeme postupovať
    podla Algoritmu 3.6 zo študíjnej opory na prevod RKA $M_{2}$ na DKA $M_{D}$, ktorý
    zadefinujeme ako $(Q_{D}, \Sigma, \delta_{D}, q^{0}_{D}, F_{D})$. Z dôvodu prehladnosti
    nie sú uvedené tie prechody, u ktorých je výsledkom $\varnothing$.
    \begin{flalign*}
        &q^{0}_{D} = \varepsilon\text{-uzáver}(1)
            = \{1, 2, 8, 9, 10, 12, 13, 14, 15, 16, 22\} = A & \\
                                                             & \\
        &\delta_{D}(A, a) = \varepsilon\text{-uzáver}(\{3, 11, 17\})
            = \{3, 4, 11, 14, 15, 16, 17, 18, 22\} = B & \\
        &\delta_{D}(B, a) = \varepsilon\text{-uzáver}(\{17\}) = \{17, 18\} = C & \\
        &\delta_{D}(B, b) = \varepsilon\text{-uzáver}(\{5, 19\})
            = \{5, 6, 19, 20\} = D & \\
       &\delta_{D}(C, b) = \varepsilon\text{-uzáver}(\{19\}) = \{19, 20\} = E & \\
       &\delta_{D}(D, c) = \varepsilon\text{-uzáver}(\{7, 21\})
            = \{2, 7, 8, 9, 10, 12, 13, 14, 15, 16, 21, 22\} = F & \\
       &\delta_{D}(E, c) = \varepsilon\text{-uzáver}(\{21\})
            = \{16 ,21, 22\} = G & \\
       &\delta_{D}(F, a) = \varepsilon\text{-uzáver}(\{3, 11, 17\})
            = \{3, 4, 11, 14, 15, 16, 17, 18, 22\} = B & \\
        &\delta_{D}(G, a) = \varepsilon\text{-uzáver}(\{17\}) = \{17, 18\} = C & \\
        & \\
        &Q_{D} = \{A, B, C, D, E, F, G\} & \\
        &F_{D} = \{A, B, F, G\}
    \end{flalign*}

    \begin{figure}[!h]
        \centering
        \begin{tikzpicture}[node distance = 2.2cm, ->, >=stealth, line width=1pt]
            \node[state, accepting, initial, initial text=] (a) {$A$};
            \node[state, accepting, right of = a] (b) {$B$};
            \node[state, below of = b] (c) {$C$};
            \node[state, right of = b] (d) {$D$};
            \node[state, right of = c] (e) {$E$};
            \node[state, accepting, right of = d] (f) {$F$};
            \node[state, accepting, right of = e] (g) {$G$};

            \path (a) edge node[above]{$a$} (b);
            \path (b) edge node[left]{$a$} (c);
            \path (b) edge node[above]{$b$} (d);
            \path (c) edge node[above]{$b$} (e);
            \path (d) edge node[above]{$c$} (f);
            \path (e) edge node[above]{$c$} (g);
            \path (f) edge[bend right=40] node[above]{$a$} (b);
            \path (g) edge[bend left=40] node[below]{$a$} (c);
        \end{tikzpicture}
        \caption{DKA $M_{D}$.}
    \end{figure}

    Aby sme mohli DKA minimalizovať, je potrebné najskôr vykonať niekoľko krokov. Najprv musíme
    DKA $M_{D}$ transformovať na úplný DKA $M_{U}$ doplnením tzv \emph{sink} stavu, tak aby platilo, že
    $\delta_{D}$ je totálnou funkciou na $Q \times \Sigma$.

    \newpage
    \begin{figure}[!h]
        \centering
        \begin{tikzpicture}[node distance = 2.5cm, ->, >=stealth, line width=1pt]
            \node[state, accepting, initial, initial text=] (a) {$A$};
            \node[state, accepting, right of = a] (b) {$B$};
            \node[state, below = 3cm of b] (c) {$C$};
            \node[state, right of = b] (d) {$D$};
            \node[state, right of = c] (e) {$E$};
            \node[state, accepting, right of = d] (f) {$F$};
            \node[state, accepting, right of = e] (g) {$G$};
            \node[state, below = 1cm of d] (z) {$Z$};

            \path (a) edge node[above]{$a$} (b);
            \path (b) edge node[left]{$a$} (c);
            \path (b) edge node[above]{$b$} (d);
            \path (c) edge node[above]{$b$} (e);
            \path (d) edge node[above]{$c$} (f);
            \path (e) edge node[above]{$c$} (g);
            \path (f) edge[bend right=40] node[above]{$a$} (b);
            \path (g) edge[bend left=40] node[below]{$a$} (c);

            \draw (a) .. controls ($(b.north) + (0,2cm)$) and ($(f.north) + (1cm,2cm)$)
                      .. ($(f.east) + (1cm,0)$)
                      .. controls ($(z.east) + (3.5cm,2cm)$) and ($(z.east) + (3.5cm,0)$)
                      .. (z.east)
                node[near start, right] {$b,c$};
            \path (b) edge node[right]{$c$} (z);
            \path (d) edge node[right]{$a,b$} (z);
            \path (f) edge node[right]{$b,c$} (z);
            \path (c) edge node[right]{$a,c$} (z);
            \path (e) edge node[right]{$a,b$} (z);
            \path (g) edge node[right]{$b,c$} (z);
            \path (z) edge[loop left] node[left]{$a,b,c$} (z);
        \end{tikzpicture}
        \caption{Úplný DKA $M_{U}$ so \emph{sink} stavom $Z$.}
    \end{figure}

    Nakoniec spravíme prevod na redukovaný DKA $M_{Z}$ pomocou Algoritmu 3.5 zo študíjnej
    opory.
    \begin{figure}[!h]
        \begin{minipage}[b]{0.45\textwidth}
            \begin{center}
                \begin{tabular}{c  c | c  c  c}
                    $\overset{0}{\equiv}$ & $\delta_{U}$ & $a$ & $b$ & $c$ \\
                    \hline
                    $I:$   & $A$ & $B_{I}$  & $Z_{II}$ & $Z_{II}$ \\
                           & $B$ & $C_{II}$ & $D_{II}$ & $Z_{II}$ \\
                           & $F$ & $B_{I}$  & $Z_{II}$ & $Z_{II}$ \\
                           & $G$ & $C_{II}$ & $Z_{II}$ & $Z_{II}$ \\
                    \hline
                    $II:$  & $C$ & $Z_{II}$ & $E_{II}$ & $Z_{II}$ \\
                           & $D$ & $Z_{II}$ & $Z_{II}$ & $F_{I}$  \\
                           & $E$ & $Z_{II}$ & $Z_{II}$ & $G_{I}$  \\
                           & $Z$ & $Z_{II}$ & $Z_{II}$ & $Z_{II}$ \\
                \end{tabular}
            \end{center}
        \end{minipage}
        \begin{minipage}[b]{0.45\textwidth}
            \begin{flalign*}
                &\overset{0}{\equiv}\ = \{ \{A, B, F, G\}, \{C, D, E, Z\} \}& \\
            \end{flalign*}
        \end{minipage}
    \end{figure}

    \begin{figure}[!h]
        \begin{minipage}[b]{0.45\textwidth}
            \begin{center}
                \begin{tabular}{c  c | c  c  c}
                    $\overset{1}{\equiv}$ & $\delta_{U}$ & $a$ & $b$ & $c$ \\
                    \hline
                    $I:$   & $A$ & $B_{II}$  & $Z_{III}$ & $Z_{III}$ \\
                           & $F$ & $B_{II}$  & $Z_{III}$ & $Z_{III}$ \\
                    \hline
                    $II:$  & $B$ & $C_{III}$ & $D_{IV}$  & $Z_{III}$ \\
                           & $G$ & $C_{III}$ & $Z_{III}$ & $Z_{III}$ \\
                    \hline
                    $III:$ & $C$ & $Z_{III}$ & $E_{IV}$  & $Z_{III}$ \\
                           & $Z$ & $Z_{III}$ & $Z_{III}$ & $Z_{III}$ \\
                    \hline
                    $IV:$  & $D$ & $Z_{III}$ & $Z_{III}$ & $F_{I}$  \\
                           & $E$ & $Z_{III}$ & $Z_{III}$ & $G_{II}$  \\
                \end{tabular}
            \end{center}
        \end{minipage}
        \begin{minipage}[b]{0.45\textwidth}
            \begin{flalign*}
                &\overset{1}{\equiv}\ = \{ \{A, F\}, \{B, G\}, \{C, Z\}, \{D, E\} \}& \\
            \end{flalign*}
        \end{minipage}
    \end{figure}

    \newpage
    \begin{figure}[!h]
        \begin{minipage}[b]{0.45\textwidth}
            \begin{center}
                \begin{tabular}{c  c | c  c  c}
                    $\overset{2}{\equiv}$ & $\delta_{U}$ & $a$ & $b$ & $c$ \\
                    \hline
                    $I:$   & $A$ & $B_{II}$  & $Z_{V}$ & $Z_{V}$ \\
                           & $F$ & $B_{II}$  & $Z_{V}$ & $Z_{V}$ \\
                    \hline
                    $II:$  & $B$ & $C_{IV}$ & $D_{VI}$  & $Z_{V}$ \\
                    \hline
                    $III:$ & $G$ & $C_{IV}$ & $Z_{V}$ & $Z_{V}$ \\
                    \hline
                    $IV:$  & $C$ & $Z_{V}$ & $E_{VII}$  & $Z_{V}$ \\
                    \hline
                    $V:$   & $Z$ & $Z_{V}$ & $Z_{V}$ & $Z_{V}$ \\
                    \hline
                    $VI:$  & $D$ & $Z_{V}$ & $Z_{V}$ & $F_{I}$  \\
                    \hline
                    $VII:$ & $E$ & $Z_{V}$ & $Z_{V}$ & $G_{III}$  \\
                \end{tabular}
            \end{center}
        \end{minipage}
        \begin{minipage}[b]{0.45\textwidth}
            \begin{flalign*}
                &\overset{2}{\equiv}\ = \{ \{A, F\}, \{B\}, \{G\}, \{C\}, \{Z\}, \{D\}, \{E\} \}& \\
            \end{flalign*}
        \end{minipage}
    \end{figure}

    U $\overset{2}{\equiv}$ končíme, nakoľko nie je možné vykonávať ďalší rozklad, teda
    $\overset{2}{\equiv}\ =\ \equiv$. Redukovaný DKA $M_{Z} = (Q_{Z}, \Sigma, \delta_{Z},
    q^{0}_{Z}, F_{Z})$ bude vyzerať nasledovne.

    \begin{flalign*}
        &Q_{D} = \{ \{A, F\}, \{B\}, \{C\}, \{D\}, \{E\}, \{G\}, \{Z\} \}
            = \{ [A], [B], [C], [D], [E], [G], [Z] \} & \\
        &q^{0}_{D} = [A] & \\
        &F_{D} = \{ [A], [B], [G] \} & \\
    \end{flalign*}
    \begin{figure}[!h]
        \centering
        \begin{tikzpicture}[node distance = 2.5cm, ->, >=stealth, line width=1pt]
            \node[state, accepting, right of = a] (b) {$[B]$};
            \node[state, below = 3cm of b] (c) {$[C]$};
            \node[state, right of = b] (d) {$[D]$};
            \node[state, right of = c] (e) {$[E]$};
            \node[state, initial, initial text=, initial where=right, accepting, right of = d] (f) {$[A]$};
            \node[state, accepting, right of = e] (g) {$[G]$};
            \node[state, below = 1cm of d] (z) {$[Z]$};

            \path (b) edge node[left]{$a$} (c);
            \path (b) edge node[above]{$b$} (d);
            \path (c) edge node[above]{$b$} (e);
            \path (d) edge node[above]{$c$} (f);
            \path (e) edge node[above]{$c$} (g);
            \path (f) edge[bend right=40] node[above]{$a$} (b);
            \path (g) edge[bend left=40] node[below]{$a$} (c);

            \path (b) edge node[right]{$c$} (z);
            \path (d) edge node[right]{$a,b$} (z);
            \path (f) edge node[right]{$b,c$} (z);
            \path (c) edge node[right]{$a,c$} (z);
            \path (e) edge node[right]{$a,b$} (z);
            \path (g) edge node[right]{$b,c$} (z);
            \path (z) edge[loop right] node[right]{$a,b,c$} (z);
        \end{tikzpicture}
        \caption{Redukovaný DKA $M_{Z}$.}
    \end{figure}

    \subtask Relácia $\sim_{L}$ je podľa Myhill-Nerodovej vety v definovaná definíciou 3.17 v študíjnej opore nasledovne.

    Nech $L$ je ľubovolný jazyk nad abecedou $\Sigma$. Na množine $\Sigma^{*}$ definujeme reláciu $\sim_{L}$ zvanú
    \emph{prefixová ekvivalencia} pre $L$ takto:
    \begin{equation*}
        u \sim_{L} v \overset{def}{\Longleftrightarrow} \forall w \in \Sigma^{*} : uw \in L \Longleftrightarrow vw \in L
    \end{equation*}

    Podľa vety 3.21 zo študíjnej opory taktiež platí, že počet stavov ľubovolného minimálneho DKA príjmajúceho $L$
    je rovný indexu $\sim_{L}$. Tento minimálny automat sme si zostrojili v úlohe (a). Zostáva nám určiť triedy ekvivalencie,
    ktoré odpovedajú triedam ekvivalencie minimálneho DKA $M_{Z}$ úlohy (a). Pre každú triedu resp. stav zostrojíme jazyk $L^{-1}$
    a k nemu odpovedajúci konečný automat.

    \begin{figure}[!h]
        \centering
        \begin{tikzpicture}[node distance = 2.5cm, ->, >=stealth, line width=1pt]
            \node[state, initial, initial text=, initial where=right, accepting, right of = d] (f) {$[A]$};
            \node[state, right of = a] (b) {$[B]$};
            \node[state, right of = b] (d) {$[D]$};

            \path (b) edge node[above]{$b$} (d);
            \path (f) edge[bend right=40] node[above]{$a$} (b);
            \path (d) edge node[above]{$c$} (f);
        \end{tikzpicture}
        \caption{KA príjmajúci jazyk $L^{-1}([A]) = (abc)^{*}$.}
    \end{figure}

    \begin{figure}[!h]
        \centering
        \begin{tikzpicture}[node distance = 2.5cm, ->, >=stealth, line width=1pt]
            \node[state, initial, initial text=, initial where=right, right of = d] (f) {$[A]$};
            \node[state, accepting, right of = a] (b) {$[B]$};
            \node[state, right of = b] (d) {$[D]$};

            \path (b) edge node[above]{$b$} (d);
            \path (f) edge[bend right=40] node[above]{$a$} (b);
            \path (d) edge node[above]{$c$} (f);
        \end{tikzpicture}
        \caption{KA príjmajúci jazyk $L^{-1}([B]) = (abc)^{*}a$.}
    \end{figure}

    \begin{figure}[!h]
        \centering
        \begin{tikzpicture}[node distance = 2.5cm, ->, >=stealth, line width=1pt]
            \node[state, initial, initial text=, initial where=right, right of = d] (f) {$[A]$};
            \node[state, right of = a] (b) {$[B]$};
            \node[state, accepting, right of = b] (d) {$[D]$};

            \path (b) edge node[above]{$b$} (d);
            \path (f) edge[bend right=40] node[above]{$a$} (b);
            \path (d) edge node[above]{$c$} (f);
        \end{tikzpicture}
        \caption{KA príjmajúci jazyk $L^{-1}([D]) = (abc)^{*}ab$.}
    \end{figure}

    \begin{figure}[!h]
        \centering
        \begin{tikzpicture}[node distance = 2.5cm, ->, >=stealth, line width=1pt]
            \node[state, right of = a] (b) {$[B]$};
            \node[state, accepting, below = 1.1cm of b] (c) {$[C]$};
            \node[state, right of = b] (d) {$[D]$};
            \node[state, right of = c] (e) {$[E]$};
            \node[state, initial, initial text=, initial where=right, right of = d] (f) {$[A]$};
            \node[state, right of = e] (g) {$[G]$};

            \path (b) edge node[left]{$a$} (c);
            \path (b) edge node[above]{$b$} (d);
            \path (c) edge node[above]{$b$} (e);
            \path (d) edge node[above]{$c$} (f);
            \path (e) edge node[above]{$c$} (g);
            \path (f) edge[bend right=40] node[above]{$a$} (b);
            \path (g) edge[bend left=40] node[below]{$a$} (c);
        \end{tikzpicture}
        \caption{KA príjmajúci jazyk $L^{-1}([C]) = (abc)^{*}a(abc)^{*}a$.}
    \end{figure}

    \newpage
    \begin{figure}[!h]
        \centering
        \begin{tikzpicture}[node distance = 2.5cm, ->, >=stealth, line width=1pt]
            \node[state, right of = a] (b) {$[B]$};
            \node[state, below = 1.1cm of b] (c) {$[C]$};
            \node[state, right of = b] (d) {$[D]$};
            \node[state, accepting, right of = c] (e) {$[E]$};
            \node[state, initial, initial text=, initial where=right, right of = d] (f) {$[A]$};
            \node[state, right of = e] (g) {$[G]$};

            \path (b) edge node[left]{$a$} (c);
            \path (b) edge node[above]{$b$} (d);
            \path (c) edge node[above]{$b$} (e);
            \path (d) edge node[above]{$c$} (f);
            \path (e) edge node[above]{$c$} (g);
            \path (f) edge[bend right=40] node[above]{$a$} (b);
            \path (g) edge[bend left=40] node[below]{$a$} (c);
        \end{tikzpicture}
        \caption{KA príjmajúci jazyk $L^{-1}([E]) = (abc)^{*}a(abc)^{*}ab$.}
    \end{figure}

    \vspace{-15pt}
    \begin{figure}[!h]
        \centering
        \begin{tikzpicture}[node distance = 2.5cm, ->, >=stealth, line width=1pt]
            \node[state, right of = a] (b) {$[B]$};
            \node[state, below = 1.1cm of b] (c) {$[C]$};
            \node[state, right of = b] (d) {$[D]$};
            \node[state, right of = c] (e) {$[E]$};
            \node[state, initial, initial text=, initial where=right, right of = d] (f) {$[A]$};
            \node[state, accepting, right of = e] (g) {$[G]$};

            \path (b) edge node[left]{$a$} (c);
            \path (b) edge node[above]{$b$} (d);
            \path (c) edge node[above]{$b$} (e);
            \path (d) edge node[above]{$c$} (f);
            \path (e) edge node[above]{$c$} (g);
            \path (f) edge[bend right=40] node[above]{$a$} (b);
            \path (g) edge[bend left=40] node[below]{$a$} (c);
        \end{tikzpicture}
        \caption{KA príjmajúci jazyk $L^{-1}([G]) = (abc)^{*}a(abc)^{*}abc$.}
    \end{figure}

    \vspace{-15pt}
    \begin{figure}[!h]
        \centering
        \begin{tikzpicture}[node distance = 2.5cm, ->, >=stealth, line width=1pt]
            \node[state, right of = a] (b) {$[B]$};
            \node[state, below = 2.4cm of b] (c) {$[C]$};
            \node[state, right of = b] (d) {$[D]$};
            \node[state, right of = c] (e) {$[E]$};
            \node[state, initial, initial text=, initial where=right, right of = d] (f) {$[A]$};
            \node[state, right of = e] (g) {$[G]$};
            \node[state, accepting, below = 0.7cm of d] (z) {$[Z]$};

            \path (b) edge node[left]{$a$} (c);
            \path (b) edge node[above]{$b$} (d);
            \path (c) edge node[above]{$b$} (e);
            \path (d) edge node[above]{$c$} (f);
            \path (e) edge node[above]{$c$} (g);
            \path (f) edge[bend right=40] node[above]{$a$} (b);
            \path (g) edge[bend left=40] node[below]{$a$} (c);

            \path (b) edge node[right]{$c$} (z);
            \path (d) edge node[right]{$a,b$} (z);
            \path (f) edge node[right]{$b,c$} (z);
            \path (c) edge node[right]{$a,c$} (z);
            \path (e) edge node[right]{$a,b$} (z);
            \path (g) edge node[right]{$b,c$} (z);
            \path (z) edge[loop right] node[right]{$a,b,c$} (z);
        \end{tikzpicture}
        \caption{KA príjmajúci jazyk $L^{-1}([Z])$.}
    \end{figure}

    Pričom platí, že jazyk $L(M_{Z}) = L^{-1}([A]) \cup L^{-1}([B]) \cup L^{-1}([G])$.

\end{mysolution}

\task{3}{Uvažujte NKA $M_{3}$ nad abecedou $\Sigma = \{a, b, c\}$ z Obrázku 1: \\
    \begin{figure}[!h]
        \centering
        \begin{tikzpicture}[node distance = 3cm, ->, >=stealth, line width=1pt]
            \node[state, initial, initial text=] (X) {};
            \node[state, right of = X] (Y) {};
            \node[state, accepting, right of = Y] (Z) {};

            \path (X) edge node[above]{a} (Y);
            \path (Y) edge node[above]{b} (Z);
            \path (Y) edge[loop above] node[above]{c} (Y);
            \path (Y) edge[bend left] node[below]{c} (X);
            \path (Z) edge[bend left=60] node[above]{b} (X);
            \path (Z) edge[bend right=60] node[above]{a} (Y);
        \end{tikzpicture}
    \caption{NKA $M_{3}$}
    \end{figure}

    K tomuto automatu zostrojte sústavu rovníc nad regulárnymi výrazmi a jej riešením zostavte
ekvivalentný regulárny výraz.}
\begin{mysolution}

\vspace{-10pt}
Jednotlivé regulárne výrazy každého stavu si označíme $X, Y\text{a}\ Z$ podľa toho ako sú
stavy znázornené v diagrame na Obrázku 1 z ľava do prava. Pre nájdenie riešenia, teda
ekvivalntného regulárneho výrazu značiaceho rovnaký jazyk, ako je príjmaný jazyk $L(M_{3})$,
je potrebné určiť regulárny výraz $X$ predstavujúci regulárny výraz pre počiatočný stav.
\begin{flalign}
    X =&\ aY & \\
    Y =&\ cX + cY + bZ & \\
    Z =&\ bX + aY + \varepsilon &
\end{flalign}
Dosadíme za $X$ a $Z$ do rovnice číslo 2. Dostávame nasledovnú sústavu.
\begin{flalign}
    X =&\ aY & \\
    Y =&\ caY + cY + b(bX + aY + \varepsilon) & \\
      =&\ caY + cY + bbX + baY + b & \\
    Z =&\ bX + aY + \varepsilon &
\end{flalign}
Znovu dosadíme za $X$, tentoraz do rovnice číslo 6.
\begin{flalign}
    X =&\ aY & \\
    Y =&\ caY + cY + bbaY + baY + b & \\
    Z =&\ bX + aY + \varepsilon &
\end{flalign}
Rovnicu číslo 9 môžeme riešiť samostatne, nakoľko sa z nej stala rovnica s jednou neznámou.
Budeme ju riešiť pomocou vety 3.14 zo študíjnej opory, ktorá vraví, že najmenším pevným bodom (najmenším riešením)
rovnice nad regulárnymi výrazmi $X = pX + q$ je $X = p^{*}q$.
\begin{flalign}
    Y =&\ caY + cY + bbaY + baY + b & \\
      =&\ (ca + c + bba + ba)Y + b & \\
      =&\ (ca + c + bba + ba)^{*}b &
\end{flalign}
Rovnicu číslo 13 následne môžeme dosadiť do rovnice číslo 1.
\begin{flalign}
    X =&\ aY & \\
      =&\ a(ca + c + bba + ba)^{*}b &
\end{flalign}
Nakoľko je $X$ počiatočným stavom automatu $M_{3}$,
tak sa jedná o hľadaný regulárny výraz značiaci regulárny jazyk $L(M_{3})$.
\end{mysolution}

\task{4}{Pre dva formálne jazyky $L_{1}$ a $L_{2}$ definujeme operáciu $restrict$ nasledovne:
    \begin{equation*}
        restrict(L_{1},L_{2}) = \{ w \pipesep w \in L_{1} \land \exists w' \in L_{2} : 
            \vert w \vert = \vert w' \vert \}
    \end{equation*}

    Navrhnite a \textit{formálne popíšte} algoritmus, ktorý má na vstupe dva konečné automaty
$M_{1} = (Q_{1}, \Sigma_{1}, \delta_{1}, q^{0}_{1}, F_{1})$ a
$M_{2} = (Q_{2}, \Sigma_{2}, \delta_{2}, q^{0}_{2}, F_{2})$ (môžu byť aj nedeterministické)
a ktorého výstupom bude konečný automat $M_{restrict}$ taký, že
$L(M_{restrict}) = restrict(L(M_{1}), L(M_{2}))$.}
\begin{mysolution}

    \vspace{-30pt}
    \begin{flalign*}
        &M_{restrict} = (Q_{restrict}, \Sigma_{restrict}, \delta_{restrict}, q^{0}_{restrict}, F_{restrict}) & \\
        & \\
        &Q_{restrict} = Q_{1} \times Q_{2} & \\
        &\Sigma_{restrict} = \Sigma_{1} & \\
        &\delta_{restrict} :
                \forall q_{1}, q_{2} \in Q_{1} \forall q_{3}, q_{4} \in Q_{2} \forall a \in \Sigma_{1}\forall b \in \Sigma_{2} : & \\
                    &\hspace{140pt} (q_{2}, q_{4}) \in \delta_{restrict}((q_{1}, q_{3}), a) & \\
                    &\hspace{200pt} \Leftrightarrow & \\
                    &\hspace{140pt} q_{2} \in \delta_{1}(q_{1}, a) \land q_{4} \in \delta_{2}(q_{3}, b)
        & \\
        &q^{0}_{restrict} = (q^{0}_{1}, q^{0}_{2}) & \\
        &F_{restrict} = F_{1} \times F_{2}
    \end{flalign*}
\end{mysolution}

\task{5}{Majme jazyk $L = \{ a^{i}b^{2i}c^{j} \pipesep i > 0 \land i < j < 2i \}$.
    Je jazyk $L$ regulárny? Dokážte alebo vyvráťte.}
\begin{mysolution}

    Predpokladajme, že jazyk $L$ je nekonečný regulárny jazyk.
    Potom $\exists p > 0$ také, že platí
    \begin{equation*}
        w \in L \land \vert w \vert \ge p \Rightarrow
        w = xyz \land y \ne \varepsilon
            \land \vert xy \vert \le p
            \land xy^{i}z \in L\ \text{pre}\ i \ge 0
    \end{equation*}
    Nech reťazec $w$ je generovaný ako $a^{p}b^{2p}c^{j}$.
    Pokiaľ platí, že $\vert xy \vert \le p$, tak môžeme povedať že reťazec $xy$ sa skladá
    výhradne zo symbolov $a$. Môžeme teda zapísať, že
    $xy^{i}z = a^{p + (i-1)\vert y \vert}b^{2p}c^{j}$.
    Pre $i = 0$ teda platí, že $xy^{0}z = a^{p - \vert y \vert}b^{2p}c^{j}$. Z toho vyplýva, že
    $\#_{b} \neq 2\#_{a}$, pretože $\vert y \vert \ge 1$ a teda ani $xy^{0}z \not\in L$.
    Dospeli sme ku sporu a teda ani pôvodný predpoklad nemôže byť pravdivý. Jazyk $L$ nie je
    regulárny.
\end{mysolution}

\task{6}{Uvažujme algebru regulárnych množín ($A_{RM}$) nad abecedou $\Sigma$.
    \begin{subtasklist}
        \subtask Ukážte, že pre $A_{RM}$ platí nasledujúci vzťah
            $\{\varepsilon\} \cup A.A^{*} = A^{*}$,
        kde $A$ je ľubovolná regulárna množina. Nepoužívajte fakt, že $A_{RM}$ je
        Kleenova algebra.
        \subtask Určite, či je $\subseteq$ \textit{totálne usporiadanie} v $A_{RM}$. Teda, že pre
        ľubovolné dve regulárne množiny A a B platí vždy aspoň jedna z nerovností
        $A \subseteq B$ alebo $B \subseteq A$. Svoje tvrdenie dokážte.
    \end{subtasklist}
}
\begin{mysolution}{subtasks}

    \subtask Veta 3.12 a 3.13 v študíjnej opore hovorí, že jazyk $L$ je regulárnou množinou práve vtedy, ak je $L$ jazykom typu 3.
    Budeme sa tým pádom opierať o fakt, že regulárna množina $A$ je regulárny jazyk a dôkaz založíme na definícií 2.9 zo študíjnej opory,
    ktorá definuje iteráciu jazyka nasledovne.

    \setcounter{equation}{0}
    \begin{align}
        L^{0} &= \{\varepsilon\} \\
        L^{n} &= L . L^{n - 1}\ \text{pre}\ n \ge 1 \\
        L^{*} &= \underset{n \ge 0}{\bigcup} L^{n} \\
        L^{+} &= \underset{n \ge 1}{\bigcup} L^{n}
    \end{align}

    Podľa vety 2.1 zo študíjnej opory taktiež platí nasledovné.
    \begin{align}
        L^{*} &= L^{+} \cup \{\varepsilon\} \\
        L^{+} &= L.L^{*} = L^{*}.L
    \end{align}

    U dókazu bude použitá logická spojka pre ekvivalenciu $\Longleftrightarrow$, nad ktorou je vždy uvedené číslo
    predpokladu z rovníc spomenutých vyššie, alebo slovne popísaný krok.

    \begin{align*}
        \{\varepsilon\}\ \cup&\ A.A^{*} = A^{*} \\
                               &\overset{6}{\Longleftrightarrow} \\
        \{\varepsilon\}\ \cup&\ A^{+} = A^{*} \\
            &\hspace{-20pt}\overset{komutativita\ \cup}{\Longleftrightarrow} \\
                    A^{+}\ \cup\ &\{\varepsilon\} = A^{*} \\
                               &\overset{5}{\Longleftrightarrow} \\
        A^{*}& = A^{*}
    \end{align*}

    \subtask TODO
\end{mysolution}

\end{document}
