\documentclass[12pt]{article}

\usepackage[margin=1in]{geometry} 
\usepackage{amsmath,amsthm,amssymb,amsfonts}
\usepackage[slovak]{babel}
\usepackage[utf8]{inputenc}
\usepackage{enumitem}
\usepackage{ifthen}
\usepackage{tikz}

\usetikzlibrary{arrows,automata,calc}

\newcommand{\task}[2]{\par \noindent \textbf{{#1}.} \hspace{3pt} #2 \vspace{10pt}}
\newcommand{\solution}{\vspace{10pt}}
\newcommand{\pipesep}{\hspace{3pt} \vert \hspace{3pt}}

\newenvironment{subtasklist}[0]{\begin{enumerate}[label=(\alph*)]}{\end{enumerate}}
\newenvironment{mysolution}[1]{
    \par \textbf{Riešenie} \newline
    \ifthenelse{\equal{#1}{subtasks}}{\begin{enumerate}[label=(\alph*)]}
            {\begin{enumerate}[label={}] \item}
}{\end{enumerate} \newpage}
\newcommand{\subtask}{\item}
 
\begin{document}
 
\title{TIN 2015/2016: Úloha 1}
\author{Marek Milkovič}
\maketitle

\task{1}{Uvažujte jazyk $L_{1} = \{a^{i}b^{j}c^{i}d^{k} \pipesep i,j,k \ge 0\}$.
    \begin{subtasklist}
            \subtask Zostavte gramatiku $G_{1}$ takú, že $L(G_{1}) = L_{1}$.
            \subtask Akého typu (podľa Chomského hierachie jazykov) je $G_{1}$ a akého typu
            je $L_{1}$? Môžu sa tieto typy všeobecne líšiť? Svoje tvrdenie zdôvodnite
            (formálny dôkaz nie je požadovaný).
    \end{subtasklist}}

\begin{mysolution}{subtasks}
\subtask % (a)
\begin{flalign*}
    G_{1} = (&N, \Sigma, P, S) &\\[5pt]
    N &= \{S, A, B, D\} &\\
    \Sigma &= \{\texttt{a}, \texttt{b}, \texttt{c}\} &\\
    P&: \\
    &S \to A D &\\
    &A \to \texttt{a} A \texttt{c} \pipesep B \pipesep \varepsilon &\\
    &B \to \texttt{b} B \pipesep \varepsilon &\\
    &D \to \texttt{d} D \pipesep \varepsilon &
\end{flalign*}

\subtask % (b)
Gramatika $G_{1}$ je typu 2 z dôvodu tvaru prepisovacích pravidiel.
Tie odpovedajú tvaru $A \to \alpha$ kde $A \in N, \alpha \in (N \cup \Sigma)^{*}$.

Jazyk $L_{1}$ môže byť typu 2 či 3. To vyplýva z toho, že gramatika typu $n$ môže generovať jazyk
typu $n$ a vyšší. Je teda potrebné dokázať, o aký jazyk sa jedná pomocou pumping lemma pre
regulárne jazyky. Predpokladajme, že $L_{1}$ je nekonečný regulárny jazyk (typu~3).
Potom $\exists p > 0$ také, že platí
\begin{equation*}
    w \in L_{1} \land \vert w \vert \ge p \Rightarrow
    w = xyz \land y \ne \varepsilon
        \land \vert xy \vert \le p
        \land xy^{i}z \in L_{1}\ \text{pre}\ i \ge 0
\end{equation*}
Nech reťazec $w$ je generovaný ako $a^{p}b^{j}c^{p}d^{k}$. Pokiaľ platí, že $\vert xy \vert \le p$,
potom môžeme povedať, že reťazec $xy$ pozostáva len zo znakov $a$. Môžeme teda zapísať, že
$xy^{i}z = a^{p + (i - 1)\vert y \vert}b^{j}c^{p}d^{k}$. Pre $i = 0$ teda platí, že
$xy^{0}z = a^{p - \vert y \vert}b^{j}c^{p}d^{k}$. Avšak $xy^{0}z \not\in L_{1}$ pretože
$\#_{a} \neq \#_{c}$. Dospeli sme ku sporu. Pôvodný predpoklad, že jazyk $L_{1}$ je regulárny
(typu~3) je teda taktiež neplatný. Jazyk $L_{1}$ teda musí byť typu 2.

Typ gramatiky a typ jazyka sa môže všeobecne líšiť. Z definície tried Chomského hierarchie jazykov plynie

\begin{equation*}
    \mathcal{L}_{3} \subseteq \mathcal{L}_{2} \subseteq \mathcal{L}_{1} \subseteq \mathcal{L}_{0}
\end{equation*}

kde $\mathcal{L}_{i}$ je trieda všetkých jazykov typu $i$. Gramatiky typu $i$ sú tým pádom
schopné generovať jazyky typu $i$ a vyššie, pretože sa jedná len o podmnožiny z triedy
jazykov typu $i$.
\end{mysolution}

\task{2}{Uvažujte regulárny výraz $r_{2} = (abc)^{*}(a + \varepsilon)(abc)^{*}$.
    \begin{subtasklist}
        \subtask Preveďte $r_{2}$ algoritmicky na redukovaný deterministický konečný automat
        $M_{2}$ (tj. RV~$\to$ RKA $\to$ DKA $\to$ redukovaný DKA), prijímajúci jazyk
        popísaný výrazom $r_{2}$.
        \subtask Pre jazyk $L(M_{2})$ určte počet tried ekvivalencie relácie $\sim_{L}$
        (viz. Myhill-Nerodova veta) a vypíšte tieto triedy. Jednotlivé triedy môžete popísať
        napríklad konečným automatom, ktorý prijíma všetky slová patriace do danej triedy.
    \end{subtasklist}
}
\begin{mysolution}{subtasks}
    \subtask TODO
    \subtask TODO
\end{mysolution}

\task{3}{Uvažujte NKA $M_{3}$ nad abecedou $\Sigma = \{a, b, c\}$ z Obrázku 1: \\
    \begin{figure}[!h]
    \begin{center}
    \begin{tikzpicture}[node distance = 3cm, ->, >=stealth, line width=1pt]
        \node[state, initial, initial text=] (X) {};
        \node[state, right of = X] (Y) {};
        \node[state, accepting, right of = Y] (Z) {};

        \path (X) edge node[above]{a} (Y);
        \path (Y) edge node[above]{b} (Z);
        \path (Y) edge[loop above] node[above]{c} (Y);
        \path (Y) edge[bend left] node[below]{c} (X);
        \path (Z) edge[bend left=60] node[above]{b} (X);
        \path (Z) edge[bend right=60] node[above]{a} (Y);
    \end{tikzpicture}
    \end{center}
    \caption{NKA $M_{3}$}
    \end{figure}

    K tomuto automatu zostrojte sústavu rovníc nad regulárnymi výrazmi a jej riešením zostavte
ekvivalentný regulárny výraz.}
\begin{mysolution}

\vspace{-10pt}
Jednotlivé regulárne výrazy každého stavu si označíme $X, Y\text{a}\ Z$ podľa toho ako sú
stavy znázornené v diagrame na Obrázku 1 z ľava do prava. Pre nájdenie riešenia, teda
ekvivalntného regulárneho výrazu značiaceho rovnaký jazyk, ako je príjmaný jazyk $L(M_{3})$,
je potrebné určiť regulárny výraz $X$ predstavujúci regulárny výraz pre počiatočný stav.
\begin{flalign*}
    X =&\ aY & \\
    Y =&\ cX + cY + bZ & \\
    Z =&\ bX + aY + \varepsilon &
\end{flalign*}
Dosadíme za $Z$ do rovnice číslo 2. Dostávame sústavu
\begin{flalign*}
    X =&\ aY & \\
    Y =&\ cX + cY + bZ & \\
      =&\ cX + cY + b(bX + aY + \varepsilon) & \\
      =&\ cX + cY + bbX + baY + b & \\
      =&\ (c + ba)Y + (c + bb)X + b & \\
      =&\ (c + ba)^{*}((c + bb)X + b) &
\end{flalign*}
Rovnicu číslo 2 sme upravili do tvaru najmenšieho pevného bodu podľa Vety 2.3, ktorá hovorí, že
najmenším pevným bodom rovnice v tvare $X = pX + q$ je $X = p^{*}q$.

Za $Y$ následne dosadíme do rovnice číslo 1 a získame tak 1 rovnicu, ktorú znova upravíme do
tvaru najmenšieho pevného bodu.
\begin{flalign*}
    X =&\ aY & \\
      =&\ a(c + ba)^{*}((c + bb)X + b) & \\
      =&\ a(c + ba)^{*}(c + bb)X + a(c + ba)^{*}b & \\
      =&\ (a(c + ba)^{*}(c + bb))^{*}a(c + ba)^{*}b &
\end{flalign*}
Regulárny výraz $X$ značí rovnaký jazyk, ako je príjmaný jazyk $L(M_{3})$.
\end{mysolution}

\task{4}{Pre dva formálne jazyky $L_{1}$ a $L_{2}$ definujeme operáciu $restrict$ nasledovne:
    \begin{equation*}
        restrict(L_{1},L_{2}) = \{ w | w \in L_{1} \land \exists w' \in L_{2} : 
            \vert w \vert = \vert w' \vert \}
    \end{equation*}

    Navrhnite a \textit{formálne popíšte} algoritmus, ktorý má na vstupe dva konečné automaty
$M_{1} = (Q_{1}, \Sigma_{1}, \delta_{1}, q^{0}_{1}, F_{1})$ a
$M_{2} = (Q_{2}, \Sigma_{2}, \delta_{2}, q^{0}_{2}, F_{2})$ (môžu byť aj nedeterministické)
a ktorého výstupom bude konečný automat $M_{restrict}$ taký, že
$L(M_{restrict}) = restrict(L(M_{1}), L(M_{2}))$.}
\begin{mysolution}

TODO
\end{mysolution}

\task{5}{Majme jazyk $L = \{ a^{i}b^{2i}c^{j} \pipesep i > 0 \land i < j < 2i \}$.
    Je jazyk $L$ regulárny? Dokážte alebo vyvráťte.}
\begin{mysolution}

TODO pumping lemma
\end{mysolution}

\task{6}{Uvažujme algebru regulárnych množín ($A_{RM}$) nad abecedou $\Sigma$.
    \begin{subtasklist}
        \subtask Ukážte, že pre $A_{RM}$ platí nasledujúci vzťah
            $\{\varepsilon\} \cup A.A^{*} = A^{*}$,
        kde $A$ je ľubovolná regulárna množina. Nepoužívajte fakt, že $A_{RM}$ je
        Kleenova algebra.
        \subtask Určite, či je $\subseteq$ \textit{totálne usporiadanie} v $A_{RM}$. Teda, že pre
        ľubovolné dve regulárne množiny A a B platí vždy aspoň jedna z nerovností
        $A \subseteq B$ alebo $B \subseteq A$. Svoje tvrdenie dokážte.
    \end{subtasklist}
}
\begin{mysolution}{subtasks}

    \subtask TODO
    \subtask TODO
\end{mysolution}

\end{document}
