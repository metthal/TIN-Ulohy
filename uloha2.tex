\documentclass[12pt]{article}

\usepackage[margin=1in]{geometry} 
\usepackage{amsmath,amsthm,amssymb,amsfonts}
\usepackage[slovak]{babel}
\usepackage[utf8]{inputenc}
\usepackage{enumitem}
\usepackage{ifthen}
\usepackage{tikz}
\usepackage{tikz-qtree}
\usepackage{caption}
\usepackage{background}

\usetikzlibrary{arrows,automata,calc,positioning}

\newcommand{\task}[2]{\par \noindent \textbf{{#1}.} \hspace{3pt} #2 \vspace{10pt}}
\newcommand{\solution}{\vspace{10pt}}
\newcommand{\pipesep}{\hspace{3pt} \vert \hspace{3pt}}

\newenvironment{subtasklist}[0]{\begin{enumerate}[label=(\alph*)]}{\end{enumerate}}
\newenvironment{mysolution}[1]{
	\par \textbf{Riešenie} \newline
	\ifthenelse{\equal{#1}{subtasks}}{\begin{enumerate}[label=(\alph*)]}
			{\begin{enumerate}[label={}] \item}
}{\end{enumerate} \newpage}
\newcommand{\subtask}{\item}

\SetBgContents{
		\parbox{0.4\textwidth}{
			\raggedleft
				Marek Milkovič \\
				\texttt{xmilko01}
		}
}
\SetBgScale{0.75}
\SetBgOpacity{1}
\SetBgAngle{0}
\SetBgColor{black}
\SetBgPosition{current page.north east}
\SetBgVshift{-0.8cm}
\SetBgHshift{-3.5cm}
 
\begin{document}

\title{TIN 2015/2016: Úloha 2}
\author{Marek Milkovič \\ \small\texttt{xmilko01@stud.fit.vutbr.cz}}
\maketitle

\task{1}{Dokážte alebo vyvráťte, že je jazyk $L = \{ a^{i}b^{j}c^{i}d^{j} \pipesep i,j \in
	\mathbb{N} \}$ bezkontextový.}

\begin{mysolution}

	Použijeme pumping lemmu pre bezkontexové jazyky podľa vety 4.19 v študíjnej opore. Nech je teda
	$L$ bezkontextový jazyk. Potom platí
	\begin{align*}
		\exists p > 0 : \forall z \in L :
		\vert z \vert \ge p \Rightarrow
		\exists u,v,w,x,y \in \Sigma^{*} :&\ z = uvwxy\ \land \\
									  &\ vx \ne \varepsilon\ \land \\
									  &\ \vert vwx \vert \le p\ \land \\
									  &\ \forall i \ge 0 : uv^{i}wx^{i}y \in L\
	\end{align*}
	Nech reťazec $z = a^{p}b^{p}c^{p}d^{p}$. Určite platí $|z| \ge p$, lebo vieme, že
	$|z| = 4p$. Nakoľko $|vwx| \le p$, tak môže nastať len istý počet možností toho, v akom
	tvare je $vwx$. Tie sú
	\begin{enumerate}[label=\arabic*.]
		\item $vwx$ sa skladá výhradne zo symbolov $a$
		\item $vwx$ sa skladá výhradne zo symbolov $b$
		\item $vwx$ sa skladá výhradne zo symbolov $c$
		\item $vwx$ sa skladá výhradne zo symbolov $d$
		\item $vwx$ sa nachádza na rozhraní symbolov $a,b$
		\item $vwx$ sa nachádza na rozhraní symbolov $b,c$
		\item $vwx$ sa nachádza na rozhraní symbolov $c,d$
	\end{enumerate}

	Najskôr vyriešime prípad číslo 1. Musíme ukázať, že $\forall i \ge 0: uv^{i}wx^{i}y \in L$.
	To je to ekvivalentné tomu, že $\forall i \ge 0: a^{p + (i-1)|vx|}b^{p}c^{p}d^{p} \in L$.
	Pokiaľ položíme $i = 0$, tak dostávame $uv^{0}wx^{0}y = a^{p - |vx|}b^{p}c^{p}d^{p}$.
	Vieme, že musí platiť $\#_{a}(z) = \#_{c}(z)$. Potom hľadáme
	také $|vx|$, aby platilo $p - |vx| = p$. To však platí výhradne pre $|vx| = 0$. Z~podmienok
	pumping lemmy však vieme, že $vx \not= \varepsilon$, a teda $|vx| \not= 0$. 
	Z toho vyplýva, že $uv^{0}wx^{0}y \not\in L$. Tým pádom sa dostávame do sporu.

	Prípady 2,3 a 4 sú analogické ku 1. V prípade číslo 3, by sme len určili reťazec
	$uv^{i}wx^{i}y$ ako $a^{p}b^{p}c^{p + (i-1)|vx|}d^{p}$ a znova by sme dospeli k rovnakému
	sporu. V prípadoch číslo 2~a~4 by sme analogicky určili reťazce rovnako ako pre dvojicu
	$a,c$ a vychádzali z toho, že $\#_{b}(z) = \#_{d}(z)$. Dospejeme k rovnakému sporu.

	V prípadoch čislo 5,6 a 7 zanedbáme také tvary reťazca $uvwxy$, kedy podreťazec $v$ alebo $x$
	sa nebude skladať výhradne z jediného symbolu. Takéto prípady môžeme zanedbať kvôli tomu,
	že pri ich mocnení v reťazci $uv^{i}wx^{i}y$ by došlo k porušeniu poradia symbolov, ktoré
	musí byť v každom prípade zachované. Budeme preto uvažovať, že rozhranie symbolov na ktorých
	sa $vwx$ nachádza, je súčasťou podreťazca $w$.

	V prípade číslo 5 môžeme zapísať, že $uv^{i}wx^{i}y$ odpovedá $a^{p + (i-1)|v|}b^{p + (i-1)|x|}c^{p}d^{p}$.
	Položme $i = 0$. Potom možeme zapísať, že $uv^{0}wx^{0}y = a^{p - |v|}b^{p - |x|}c^{p}d^{p}$.
	Hľadáme teda také $|v|$ a $|x|$, aby platilo $p - |v| = p$ a zároveň $p - |x| = p$, pretože vieme, že
	musí platiť $\#_{a}(z) = \#_{c}(z)$ a $\#_{b}(z) = \#_{d}(z)$. To platí len pre $|v| = 0$ a $|x| = 0$.
	Z toho ale vyplýva, že aj $|vx| = 0$. Z podmienok pumping lemmy ale musí platiť $vx \not= \varepsilon$
	a teda $|vx| \not= 0$. Dostávame sa do sporu.

	Prípady čislo 6 a 7 môžu byť vyriešené analogicky, ako prípad číslo 5. Nakoľko sme sa dostali v každom
	prípade do sporu, tak aj pôvodné tvrdenie, že $L$ je bezkontextový jazyk nemôže platiť. Jazyk $L$ tým
	pádom nie je bezkontextový.
\end{mysolution}

\task{2}{S využitím uzáverových vlastností bezkontextových a regulárnych jazykov dokážte
	alebo vyvráťte, že je jazyk $L \setminus L'$ nutne bezkontextový, a to pre každý zo štyroch
	nasledujúcich prípadov.
	\begin{subtasklist}
	\item $L \in \mathcal{L}_{3}, L' \in \mathcal{L}_{3}$
	\item $L \in \mathcal{L}_{3}, L' \in \mathcal{L}_{2}$
	\item $L \in \mathcal{L}_{2}, L' \in \mathcal{L}_{3}$
	\item $L \in \mathcal{L}_{2}, L' \in \mathcal{L}_{2}$
	\end{subtasklist}
}

\begin{mysolution}{subtasks}
	\subtask Podľa vety 3.23 vieme, že trieda regulárnych jazykov $\mathcal{L}_{3}$
	tvorí množinovú Booleovu algebru. Rozdiel
	$L \setminus L'$ pritom môžeme vyjadriť nasledovne.
	\begin{equation*}
		L \setminus L' = L \cap \overline{L'}
	\end{equation*}
	Z Booleovej algebry plynie uzavretosť voči komplementu (doplnku) a prieniku. Tým pádom
	$L \setminus L'$ je jazyk regulárny. $L \setminus L'$ je teda nutne bezkotextový.

	\subtask Použitím DeMorganových zákonov vyjadríme $L \setminus L'$ ako
	\begin{equation*}
		L \setminus L' = L \cap \overline{L'} = \overline{\overline{L \cap \overline{L'}}}
		= \overline{\overline{L} \cup \overline{\overline{L'}}} = \overline{\overline{L} \cup L'}
	\end{equation*}
	Podľa vety 3.23, ktorá hovorí, že trieda regulárnych jazykov $\mathcal{L}_{3}$ tvorí množinovú
	Booleovu algebru vieme, že $\mathcal{L}_{3}$ je uzavretá voči komplementu. 
	$\overline{L}$ teda zostane regulárny jazyk. Pokiaľ teda jazyk
	$\overline{L} \in \mathcal{L}_{3}$, tak existuje gramatika typu 3
	$G_{1} = (N_{1}, \Sigma_{1}, P_{1}, S_{1})$, ktorá
	tento jazyk generuje. Analogicky, ak $L' \in \mathcal{L}_{2}$, tak existuje gramatika
	typu 2 $G_{2} = (N_{2}, \Sigma_{2}, P_{2}, S_{2})$ generujúca $L'$.
	Zjednotenie $\overline{L} \cup L'$ teda odpovedá jazyku, ktorý je generovaný
	gramatikou $G = (N_{1} \cup N_{2} \cup \{S\}, \Sigma_{1} \cup \Sigma_{2},
	P_{1} \cup P_{2} \cup \{S \to S_{1}|S_{2}\}, S)$. Musí ale platiť podmienka, že
	$N_{1} \cap N_{2} = \varnothing$. Nakoľko gramatika $G$ obsahuje prepisovacie pravidlá
	v tvare $A \to \alpha$, kde $A \in N, \alpha \in (N \cup \Sigma)^{*}$, ktoré definujú
	gramatiku typu 2 podľa Chomského klasifikácie gramatík, tak $G$ je typu 2.
	Jazyk $\overline{L} \cup L'$ je teda nutne bezkontextový.
	Pokiaľ je jazyk $\overline{L} \cup L'$ nutne bezkontextový, tak 
	podľa vety 4.24 zo študíjnej opory, bezkontextové
	jazyky nie sú uzavrené voči doplnku. Takže $L \setminus L'$ nie je nutne bezkontextový.
	\subtask Opäť upravíme $L \setminus L'$ na ekvivalentný tvar.
	\begin{equation*}
		L \setminus L' = L \cap \overline{L'}
	\end{equation*}
	Vzhľadom na to, že trieda regulárnych jazykov $\mathcal{L}_{3}$ tvorí množinovú
	Booleovu algebru, tak jazyk $\overline{L'}$ je opäť regulárny, kvôli uzavretosti
	komplementu. Podľa vety 4.22 zo študíjnej opory, bezkontextové jazyky sú uzavreté
	voči prieniku s regulárnymi jazykmi. $L \setminus L'$ je teda v tomto prípade nutne
	bezkontextový jazyk.
	\subtask Pokiaľ má platiť, že bezkontextové jazyky majú byť uzavreté voči rozdielu,
	tak musí platiť, že $L \setminus (L \setminus L')$ je tiež bezkontextový jazyk.
	Tento výraz, však môžeme napísať v ekvivalentom tvare ako
	\begin{equation*}
		L \setminus (L \setminus L') = L \cap L'
	\end{equation*}
	Podľa vety 4.24 zo študíjnej opory, bezkontextové jazyky nie sú uzavreté voči prieniku.
	Nie je preto možné, aby boli teda uzavreté voči rozdielu. $L \setminus L'$ tým pádom
	nie je nutne bezkontextový.
\end{mysolution}

\task{3}{Uvažujme bezkontextové gramatiky nad abecedou $\Sigma \subseteq \mathbb{Z}$.
	Definujme váhu slova $w = k_{1}...k_{n} \in \Sigma^{*}$ ako
	$\vert\vert w \vert\vert = \sum\limits_{i=1}^{n} k_{i}$. Váha bezkontextovej gramatiky $G$
	nad abecedou $\Sigma$ je potom definovaná ako minimálna váha slova jejho jazyka,
	$\vert\vert G \vert\vert = \text{min}\{\vert\vert w \vert\vert \pipesep w \in L(G)\}$.
	Minimum množiny čísel je definované štandardným spôsobom s tým, že
	$\text{min}(\varnothing) = \infty$, a pokiaľ je $S$ neprázdna množina, ktorá neobsahuje
	minimálny prvok, je $\text{min}(S) = -\infty$. Pre $-\infty$ a $\infty$ počítame s pravidlami
	$\text{min}(\{-\infty\} \cup S) = -\infty$ a $\text{min}(\{\infty\} \cup S) = \text{min}(S)$.
	Navrhnite algoritmus, ktorý pre danú $G$ vráti $\vert\vert G \vert\vert$, a to
	\begin{subtasklist}
	\item najskôr pre prípad, kedy $\Sigma \subset \mathbb{N}$, teda medzi symbolmi nie sú
		záporné čísla,
	\item a potom vo všeobecnom prípade, kedy $\Sigma \subset \mathbb{Z}$.
	\end{subtasklist}}

\begin{mysolution}{subtasks}
	\subtask Pred samotným popisom algoritmu zavedieme pomocné pojmy pre algoritmus.
	\begin{itemize}
		\item \textbf{Ohodnotenie neterminálneho symbolu $A$}, kde $A \in N$ je definované
			ako množina $V_{A} \subseteq (\mathbb{Z} \cup \{-\infty, \infty\})$.
		\item \textbf{Váha neterminálneho symbolu $A$}, kde $A \in N$ je definovaná ako
			$\text{min}(V_{A})$ a značí sa ako $||A||$.
		\item \textbf{Váha vetnej formy $\alpha$}, kde
			$\alpha = X_{1}X_{2}...X_{n}$ a $X_{i} \in (N \cup \Sigma), i \in \{1,...,n\}$,
			je definovaná ako $||\alpha|| = \sum\limits_{i=1}^{n} X_{i}$.
	\end{itemize}
	Algoritmus funguje na princípe postupné ohodnocovanie neterminálnych symbolov na základe
	pravidiel v gramatike, pričom sa začína od pravidiel generujúcich len reťazec skladajúce
	sa z terminálnych symbolov a vykonáva sa pokým sa ohodnotenie neterminálov mení.
	Pokiaľ sa nejakému neterminálu zmení ohodnotenie, tak v ďaľšej iterácií budú spracované
	všetky pravidlá obsahujúce tento neterminál na pravej strane svojich pravidiel.
	Množina pravidiel ktoré sa aktuálne spracúvajú je označená $R_{i}$, kde $i$ odpovedá
	indexu iterácie. Predpokladá sa, že každá množina $R_{i}$ je zpočiatku prázdna.
	Zápis algoritmu v pseudokóde s použitím formálnych matematických zápisov,
	by vyzeral nasledovne. (Na ďalšej strane)

	\begin{flalign*}
		&\texttt{for all}\ A \in N & \\
		&\hspace{1cm} V_{A} := \{\infty\} & \\
		&\texttt{endfor} & \\
		&R_{0} := \{A \to w \in P \pipesep w \in \Sigma^{*}\} & \\
		&i := 0 & \\
		&\texttt{while}\ R_{i} \not= \varnothing & \\
		&\hspace{1cm} \texttt{for all}\ (A \to \alpha) \in R_{i} & \\
		&\hspace{2cm} \texttt{if}\ ||\alpha|| < ||A||\ \texttt{then} & \\
		&\hspace{3cm} V_{A} := V_{A} \cup \{||\alpha||\} & \\
		&\hspace{3cm} R_{i+1} := R_{i+1} \cup
			\{B \to \beta A \gamma \in P \pipesep B \in N \land \beta,\gamma
			\in (N \cup \Sigma)^{*}\} & \\
		&\hspace{2cm} \texttt{endif} & \\
		&\hspace{1cm} \texttt{endfor} & \\
		&\hspace{1cm} i := i + 1 & \\
		&\texttt{endwhile} & \\
		&||G|| := ||S|| &
	\end{flalign*}
	\newpage
	\subtask V prípade celých čísel je nutné zmodifikovať algoritmus.
	Algoritmus je rozšírený o~ukladanie už spracovaných množín pravidiel $R_{i}$ do množiny
	$H \subseteq 2^{P}$. Pokiaľ
	je spracovávaná nejaká množina $R_{i}$, ktorá je totožná už s inou spracovanou množinou,
	tak je táto skutočnosť považovaná za prítomnosť cyklu v gramatike a heuristicky
	sú označené neterminály na ľavej strane pravidiel v $R_{i}$ ohodnotením $-\infty$.
	Zápis algoritmu v pseudokóde s použitím formálnych matematických zápisov,
	by vyzeral nasledovne.
	\begin{flalign*}
		&\texttt{for all}\ A \in N & \\
		&\hspace{1cm} V_{A} := \{\infty\} & \\
		&\texttt{endfor} & \\
		&R_{0} := \{A \to w \in P \pipesep w \in \Sigma^{*}\} & \\
		&i := 0 & \\
		&H := \{\varnothing\} & \\
		&\texttt{while}\ R_{i} \not= \varnothing & \\
		&\hspace{1cm} \texttt{if}\ R_{i} \in H\ \texttt{then} & \\
		&\hspace{2cm} \texttt{for all}\ (A \to \alpha) \in R_{i} & \\
		&\hspace{3cm} \texttt{if}\ ||A|| \not= -\infty\ \texttt{then} & \\
		&\hspace{4cm} V_{A} := V_{A} \cup \{-\infty\} & \\
		&\hspace{4cm} R_{i+1} := R_{i+1} \cup
			\{B \to \beta A \gamma \in P \pipesep B \in N \land \beta,\gamma
			\in (N \cup \Sigma)^{*}\} & \\
		&\hspace{3cm} \texttt{endif} & \\
		&\hspace{2cm} \texttt{endfor} & \\
		&\hspace{1cm} \texttt{endif} & \\
		&\hspace{1cm} \texttt{for all}\ (A \to \alpha) \in R_{i} & \\
		&\hspace{2cm} \texttt{if}\ ||\alpha|| < ||A||\ \texttt{then} & \\
		&\hspace{3cm} V_{A} := V_{A} \cup \{||\alpha||\} & \\
		&\hspace{3cm} R_{i+1} := R_{i+1} \cup
			\{B \to \beta A \gamma \in P \pipesep B \in N \land \beta,\gamma
			\in (N \cup \Sigma)^{*}\} & \\
		&\hspace{2cm} \texttt{endif} & \\
		&\hspace{1cm} \texttt{endfor} & \\
		&\hspace{1cm} H := H \cup \{R_{i}\} & \\
		&\hspace{1cm} i := i + 1 & \\
		&\texttt{endwhile} & \\
		&||G|| := ||S|| &
	\end{flalign*}
\end{mysolution}

\task{4}{Dokážte formálne, že $L \subseteq L(G)$, kde
	$L = \{0^{i}1^{j} \pipesep 0 \le 2i \le j \le 3i\}$ a $G = (\{S\},\{0,1\},P,S)$ je
	bezkontextová gramatika s pravidlami
	\begin{equation*}
		S \to 0S11 \pipesep 0S111 \pipesep \varepsilon .
	\end{equation*}
	Dôkaz veďte indukciou k počtu symbolov $0$ v slove $w \in L$.}

\begin{mysolution}

	Určime $n = \#_{0}(w)$ pre $w \in L$.
	\begin{itemize}
		\item $n = 0$ \\
			V tomto prípade sa $w = \varepsilon$. V gramatike $G$ existuje derivácia
			$S \Rightarrow \varepsilon$, ktorá generuje túto vetu.
		\item $n = i$ \\
			Predpokladajme, že gramatika $G$ dokáže generovať ľubovolnú vetu $0^{i}0^{j}$,
			pre ktorú platí $2i \le j \le 3i$,
			použitím derivácií $S \Rightarrow 0S11$ a $S \Rightarrow 0S111$. Táto veta bude
			vygenerovaná po $i + 1$ derivačných krokoch. Môžeme teda zapísať
			\begin{equation*}
				S \Rightarrow^{i+1} 0^{i}1^{j}
			\end{equation*}
		\item $n = i + 1$ \\
			Dokážeme, že gramatika $G$ generuje ľubovolnú vetu $w = 0^{i+1}0^{j+k}$, kde
			$k \in \{2,3\}$ z dôvodu dodržania podmienky $2(i+1) \le j+k \le 3(i+1)$.
			Túto podmienku môžeme upraviť na $2i+2 \le j+k \le 3i+3$. Pokiaľ platí
			$2i \le j \le 3i$, tak $2 \le k \le 3$.
			Potom by malo platiť, že
			\begin{equation*}
				S \Rightarrow^{i+2} 0^{i+1}1^{j+k}
			\end{equation*}
			Vetu $0^{i+1}1^{j+k}$ môžeme zapísať aj ako $00^{i}1^{j}1^{k}$.
			Z indukčného predpokladu vieme, že platí
			\begin{equation*}
				S \Rightarrow^{i+1} 0^{i}1^{j}
			\end{equation*}
			Pre dve možné hodnoty $k$ môžeme zapísať, že
			\begin{itemize}
				\item $k = 2$
					\begin{equation*}
						S \Rightarrow 0S11 \Rightarrow^{i+1} 00^{i}1^{j}11
					\end{equation*}
				\item $k = 3$
					\begin{equation*}
						S \Rightarrow 0S111 \Rightarrow^{i+1} 00^{i}1^{j}111
					\end{equation*}
			\end{itemize}
			Všeobecne teda platí, že
			\begin{equation*}
				S \Rightarrow 0S1^{k} \Rightarrow^{i+1} 00^{i}1^{j}1^{k}
			\end{equation*}
			$00^{i}1^{j}1^{k}$ sa ale rovná $0^{i+1}1^{j+k}$, a teda platí aj indukčný
			krok. 
			\begin{equation*}
				S \Rightarrow^{i+2} 0^{i+1}1^{j+k}
			\end{equation*}

	\end{itemize}
\end{mysolution}

\task{5}{Majme gramatiku $G = (\{S\},\{\textbf{if},\textbf{else},cond,com\},P,S)$ s pravidlami
\begin{equation*}
	S \to \textbf{if}\ cond\ S \pipesep \textbf{if}\ cond\ S\ \textbf{else}\ S \pipesep com .
\end{equation*}
\begin{subtasklist}
	\item Je gramatika jednoznačná? Dokážte.
	\item Je $L(G)$ jazyk s inherentnou viacznačnosťou? Zdôvodnite.
	\item Navrhnite deterministický zásobníkový automat akceptujúci $L(G)$. Prezentujte ho
		prechodovým diagramom a demonštrujte jeho beh na slove
	\begin{equation*}
	\textbf{if}\ cond\ \textbf{if}\ cond\ com\ \textbf{else}\ com\ \textbf{else}\ 
	\textbf{if}\ cond\ com.
	\end{equation*}
\end{subtasklist}}

\begin{mysolution}{subtasks}
	\subtask Podľa definície 4.5 zo študíjnej opory hovoríme, že veta $w$ generovaná gramatikou $G$
	je viacznačná, pokiaľ existujú aspoň 2 rôzne derivačné stromy s koncovými uzlami tvoriacmi vetu $w$.
	Gramatika $G$ sa potom nazýva viacznačná, pokiaľ generuje aspoň jednu viacznačnú vetu. Stačí nám teda
	vygenerovať gramatikou $G$ nejakú vetu, pre ktorú vieme zostrojiť aspoň 2 derivačné stromy a tým
	dokázať jej viacznačnosť. Takou vetou je napríklad veta
	\begin{equation*}
	\textbf{if}\ cond\ \textbf{if}\ cond\ com\ \textbf{else}\ com
	\end{equation*}
	Táto veta mohla vzniknúť dvoma možnými deriváciami, ktoré odpovedajú dvom možným derivačným stromom. Gramatika $G$ je preto viacznačná.
	\begin{figure}[!h]
		\centering
		\begin{minipage}[b]{0.45\textwidth}
			\centering
			\begin{align*}
				S \Rightarrow\ &\textbf{if}\ cond\ S \Rightarrow \\
							   &\textbf{if}\ cond\ \textbf{if}\ cond\ S\ \textbf{else}\ S \Rightarrow \\
							   &\textbf{if}\ cond\ \textbf{if}\ cond\ com\ \textbf{else}\ S \Rightarrow \\
							   &\textbf{if}\ cond\ \textbf{if}\ cond\ com\ \textbf{else}\ com
			\end{align*}
		\end{minipage}
		\begin{minipage}[b]{0.45\textwidth}
			\centering
			\begin{align*}
				S \Rightarrow\ &\textbf{if}\ cond\ S\ \textbf{else}\ S \Rightarrow \\
							   &\textbf{if}\ cond\ \textbf{if}\ cond\ S\ \textbf{else}\ S \Rightarrow \\
							   &\textbf{if}\ cond\ \textbf{if}\ cond\ com\ \textbf{else}\ S \Rightarrow \\
							   &\textbf{if}\ cond\ \textbf{if}\ cond\ com\ \textbf{else}\ com
			\end{align*}
		\end{minipage}
	\end{figure}
	\begin{figure}[!h]
		\centering
		\begin{minipage}[b]{0.45\textwidth}
			\begin{tikzpicture}[level 1/.style={sibling distance = 1.75cm}, level 2/.style={sibling distance = 1.5cm}]
				\node {$S$}
					child
					{
						node {\textbf{if}} edge from parent
					}
					child
					{
						node {$cond$} edge from parent
					}
					child
					{
						node {$S$} edge from parent
							child
							{
								node {\textbf{if}} edge from parent
							}
							child
							{
								node {$cond$} edge from parent
							}
							child
							{
								node {$S$} edge from parent
									child
									{
										node {$com$} edge from parent
									};
							}
							child
							{
								node {\textbf{else}} edge from parent
							}
							child
							{
								node {$S$} edge from parent
									child
									{
										node {$com$} edge from parent
									};
							};
					};
			\end{tikzpicture}
		\end{minipage}
		\begin{minipage}[b]{0.45\textwidth}
			\begin{tikzpicture}[level 1/.style={sibling distance = 1.75cm}, level 2/.style={sibling distance = 1.5cm}]
				\node {$S$}
					child
					{
						node {\textbf{if}} edge from parent
					}
					child
					{
						node {$cond$} edge from parent
					}
					child
					{
						node {$S$} edge from parent
							child
							{
								node {\textbf{if}} edge from parent
							}
							child
							{
								node {$cond$} edge from parent
							}
							child
							{
								node {$S$} edge from parent
									child
									{
										node {$com$} edge from parent
									};
							};
					}
					child
					{
						node {\textbf{else}} edge from parent
					}
					child
					{
						node {$S$} edge from parent
							child
							{
								node {$com$} edge from parent
							};
					};
			\end{tikzpicture}
		\end{minipage}
	\end{figure}
	\subtask Podľa definície 4.5 zo študíjnej opory hovoríme o jazyku s inherentnou viacznačnosťou, pokiaľ neexistuje
	jednoznačna gramatika generujúca tento jazyk. Určenie, či je gramatika viacznačna, alebo nie, je problém nerozhodnuteľný,
	preto nie je možné algoritmicky dokázať jednoznačnosť, avšak môžeme intuitívne lokalizovať problém, prečo je daná gramatika
	viacznačna a vhodnými úpravami z nej spraviť gramatiku jednoznačnú. V našej gramatike $G$ je problém ten, že nie je možné
	jednoznačne vytvoriť pár \textbf{if}-\textbf{else}, ktorý patrí k sebe. Túto gramatiku transformujeme na gramatiku
	$G' = (N \cup \{S'\},\Sigma,P',S)$.
	\begin{align*}
		P': S &\to \textbf{if}\ cond\ S \pipesep \textbf{if}\ cond\ S'\ \textbf{else}\ S \pipesep com \\
		S' &\to \textbf{if}\ cond\ S'\ \textbf{else}\ S' \pipesep com
	\end{align*}
	Táto gramatika rieši spomínaný problém tak, že vždy spáruje najbližšií \textbf{if}-\textbf{else} a tým povoluje tvorbu
	len jednej z možných derivácií. Veta $\textbf{if}\ cond\ \textbf{if}\ cond\ com\ \textbf{else}\ com$ má v~tejto gramatike
	len jediný derivačný strom a to nasledovný. Môžeme teda predpokladať, aj keď bez dôkazu, že tento jazyk je bez inherentnej viacznačnosti.
	\begin{figure}[!h]
		\centering
		\begin{minipage}[ht]{0.45\textwidth}
		\centering
			\begin{tikzpicture}[level 1/.style={sibling distance = 1.75cm}, level 2/.style={sibling distance = 1.5cm}]
				\node {$S$}
					child
					{
						node {\textbf{if}} edge from parent
					}
					child
					{
						node {$cond$} edge from parent
					}
					child
					{
						node {$S$} edge from parent
							child
							{
								node {\textbf{if}} edge from parent
							}
							child
							{
								node {$cond$} edge from parent
							}
							child
							{
								node {$S'$} edge from parent
									child
									{
										node {$com$} edge from parent
									};
							}
							child
							{
								node {\textbf{else}} edge from parent
							}
							child
							{
								node {$S$} edge from parent
									child
									{
										node {$com$} edge from parent
									};
							};
					};
			\end{tikzpicture}
		\end{minipage}
		\begin{minipage}[ht]{0.45\textwidth}
			\centering
			\begin{align*}
				S \Rightarrow\ &\textbf{if}\ cond\ S \Rightarrow \\
							   &\textbf{if}\ cond\ \textbf{if}\ cond\ S'\ \textbf{else}\ S \Rightarrow \\
							   &\textbf{if}\ cond\ \textbf{if}\ cond\ com\ \textbf{else}\ S \Rightarrow \\
							   &\textbf{if}\ cond\ \textbf{if}\ cond\ com\ \textbf{else}\ com
			\end{align*}
		\end{minipage}
	\end{figure}
	\subtask Zostrojíme DZKA $P = (Q,\Sigma,\Gamma,\delta,q_{0},z_{0},F)$, ktorý bude prijímať jazyk $L(G) = L(P)$
	prechodom do koncového stavu. Dôvodom je to, že v jazyku $L(G)$ existujú také 2 reťazce, že jeden je prefixom druhého.
	V prípade takéhoto jazyka nie je možné, aby bol prijímaný DKZA vyprázdnením zásobníku.

	\begin{figure}[!h]
		\begin{minipage}[ht]{0.25\textwidth}
			\begin{flalign*}
				&Q = \{q_{1},q_{2},q_{3},q_{4},q_{5}\} & \\
				&\Sigma = \{\textbf{if},\textbf{else},cond,com\} & \\
				&\Gamma = \{\textbf{if}, Z\} & \\
				&q_{0} = q_{1} & \\
				&z_{0} = Z & \\
				&F = \{q_{4}\} &
			\end{flalign*}
		\end{minipage}
		\begin{minipage}[h]{0.85\textwidth}
			\centering
			\begin{tikzpicture}[node distance = 2.95cm, ->, >=stealth, line width=1pt, scale=0.9, every node/.style={transform shape}]
				\node[state, initial, initial text=$Z$] (q1) {$q_{1}$};
				\node[state, right of = q1] (q2) {$q_{2}$};
				\node[state, right of = q2] (q3) {$q_{3}$};
				\node[state, accepting, right of = q3] (q4) {$q_{4}$};
				\node[state, below of = q4] (q5) {$q_{5}$};

				\path (q1) edge node[above]{\small$\textbf{if},Z/\textbf{if}\ Z$} (q2);
				\path (q2) edge node[above]{\small$cond,\textbf{if}/\textbf{if}$} (q3);
				\path (q3) edge[bend right=60] node[above]{\small$\textbf{if},\textbf{if}/\textbf{if\ if}$} (q2);
				\path (q3) edge node[above]{\small$com,\textbf{if}/\textbf{if}$} (q4);
				\path (q4) edge node[right]{\small$\textbf{else},\textbf{if}/\textbf{if}$} (q5);
				\path (q5) edge[bend left] node[left]{\small$com,\textbf{if}/\varepsilon$} (q4);
				\path (q5) edge[bend left=30] node[above]{\small$\textbf{if},\textbf{if}/\textbf{if}$} (q2);
			\end{tikzpicture}
		\end{minipage}
	\end{figure}

	Beh tohto automatu je znázornený pomocou prechodov automatu medzi konfiguráciami.
	\begin{align*}
		&(q_{1}, \textbf{if}\ cond\ \textbf{if}\ cond\ com\ \textbf{else}\ com, Z)
		\vdash \\
		&(q_{2}, cond\ \textbf{if}\ cond\ com\ \textbf{else}\ com, \textbf{if}\ Z)
		\vdash \\
		&(q_{3}, \textbf{if}\ cond\ com\ \textbf{else}\ com, \textbf{if}\ Z)
		\vdash \\
		&(q_{2}, cond\ com\ \textbf{else}\ com, \textbf{if}\ \textbf{if}\ Z)
		\vdash \\
		&(q_{3}, com\ \textbf{else}\ com, \textbf{if}\ \textbf{if}\ Z)
		\vdash \\
		&(q_{4}, \textbf{else}\ com, \textbf{if}\ \textbf{if}\ Z)
		\vdash \\
		&(q_{5}, com, \textbf{if}\ \textbf{if}\ Z)
		\vdash \\
		&(q_{4}, \varepsilon, \textbf{if}\ Z)
	\end{align*}

\end{mysolution}

\end{document}
