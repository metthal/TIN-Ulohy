\documentclass[12pt]{article}

\usepackage[margin=1in]{geometry} 
\usepackage{amsmath,amsthm,amssymb,amsfonts}
\usepackage[slovak]{babel}
\usepackage[utf8]{inputenc}
\usepackage{enumitem}
\usepackage{ifthen}
\usepackage{tikz}
\usepackage{tikz-qtree}
\usepackage{caption}
\usepackage{background}

\usetikzlibrary{arrows,automata,calc,positioning}

\newcommand{\task}[2]{\par \noindent \textbf{{#1}.} \hspace{3pt} #2 \vspace{10pt}}
\newcommand{\solution}{\vspace{10pt}}
\newcommand{\pipesep}{\hspace{3pt} \vert \hspace{3pt}}

\newenvironment{subtasklist}[0]{\begin{enumerate}[label=(\alph*)]}{\end{enumerate}}
\newenvironment{mysolution}[1]{
    \par \textbf{Riešenie} \newline
    \ifthenelse{\equal{#1}{subtasks}}{\begin{enumerate}[label=(\alph*)]}
            {\begin{enumerate}[label={}] \item}
}{\end{enumerate} \newpage}
\newcommand{\subtask}{\item}

\SetBgContents{
        \parbox{0.4\textwidth}{
            \raggedleft
                Marek Milkovič \\
                \texttt{xmilko01}
        }
}
\SetBgScale{0.75}
\SetBgOpacity{1}
\SetBgAngle{0}
\SetBgColor{black}
\SetBgPosition{current page.north east}
\SetBgVshift{-0.8cm}
\SetBgHshift{-3.5cm}
 
\begin{document}

\title{TIN 2015/2016: Úloha 2}
\author{Marek Milkovič \\ \small\texttt{xmilko01@stud.fit.vutbr.cz}}
\maketitle

\task{1}{Dokážte alebo vyvráťte, že jazyk $L = \{ a^{i}b^{j}c^{i}d^{j} \pipesep i,j \in
	\mathbb{N} \}$ bezkontextový.}

\begin{mysolution}

	TODO
\end{mysolution}

\task{2}{S využitím uzáverových vlastností bezkontextových a regulárnych jazykov dokážte
	alebo vyvráťte, že je jazyk $L \setminus L'$ nutne bezkontextový, a to pre každý zo štryroch
	nasledujúcich prípadov.
	\begin{subtasklist}
	\item $L \in \mathcal{L}_{3}, L' \in \mathcal{L}_{3}$
	\item $L \in \mathcal{L}_{3}, L' \in \mathcal{L}_{2}$
	\item $L \in \mathcal{L}_{2}, L' \in \mathcal{L}_{3}$
	\item $L \in \mathcal{L}_{2}, L' \in \mathcal{L}_{2}$
	\end{subtasklist}
}

\begin{mysolution}

	TODO
\end{mysolution}

\task{3}{Uvažujme bezkontextové gramatiky nad abecedou $\Sigma \subseteq \mathbb{Z}$.
	Definujme váhu slova $w = k_{1}...k_{2} \in \Sigma^{*}$ ako
	$\vert\vert w \vert\vert = \sum\limits_{i=1}^{n} k_{i}$. Váha bezkontextovej gramatiky $G$
	nad abecedou $\Sigma$ je potom definovaná ako minimálna váha slova jejho jazyka,
	$\vert\vert G \vert\vert = \text{min}\{\vert\vert w \vert\vert \pipesep w \in L(G)\}$.
	Minimum množiny čísel je definované štandardným spôsobom s tým, že
	$\text{min}(\varnothing) = \infty$, a pokiaľ je $S$ neprázdna množina, ktorá neobsahuje
	minimálny prvok, je $\text{min}(S) = -\infty$. Pre $-\infty$ a $\infty$ počítame s pravidlami
	$\text{min}(\{-\infty\} \cup S) = -\infty$ a $\text{min}(\{\infty\} \cup S) = \text{min}(S)$.
	Navrhnite algoritmus, ktorý pre danú $G$ vráti $\vert\vert G \vert\vert$, a to
	\begin{subtasklist}
	\item najskôr pre prípad, kedy $\Sigma \subset \mathbb{N}$, teda medzi symbolmi nie sú
		záporné čísla,
	\item a potom vo všeobecnom prípade, kedy $\Sigma \subset \mathbb{Z}$.
	\end{subtasklist}}

\begin{mysolution}

	TODO
\end{mysolution}

\task{4}{Dokážte formálne, že $L \subseteq L(G)$, kde
	$L = \{0^{i}1^{j} \pipesep 0 \le 2i \le j \le 3 \}$ a $G = (\{S\},\{0,1\},P,S)$ je
	bexkontextová gramatika s pravidlami
	\begin{equation*}
		S \to 0S11 \pipesep 0S111 \pipesep \varepsilon .
	\end{equation*}
	Dôkaz veďte indukciou k počtu symbolov $0$ v slove $w \in L$.}

\begin{mysolution}

	TODO
\end{mysolution}

\task{5}{Majme gramatiku $G = (\{S\},\{\textbf{if},\textbf{else},cond,com\},P,S)$ s pravidlami
\begin{equation*}
	S \to \textbf{if}\ cond\ S \pipesep \textbf{if}\ cond\ S\ \textbf{else}\ S \pipesep com .
\end{equation*}
\begin{subtasklist}
	\item Je gramatika jednoznačná? Dokážte.
	\item Je $L(G)$ jazyk s inherentnou viacznačnosťou? Zdôvodnite.
	\item Navrhnite deterministický zásobníkový automat akceptujúci $L(G)$. Prezentujte jej
		prechodovým diagramom a demonštrujte jeho beh na slove
	\begin{equation*}
	\textbf{if}\ cond\ \textbf{if}\ cond\ com\ \textbf{else}\ com\ \textbf{else}\ 
	\textbf{if}\ cond\ com.
	\end{equation*}
\end{subtasklist}}

\begin{mysolution}

	TODO
\end{mysolution}

\end{document}
