\documentclass[12pt]{article}

\usepackage[margin=1in]{geometry} 
\usepackage{amsmath,amsthm,amssymb,amsfonts}
\usepackage[slovak]{babel}
\usepackage[utf8]{inputenc}
\usepackage{enumitem}
\usepackage{ifthen}
\usepackage{tikz}
\usepackage{tikz-qtree}
\usepackage{caption}
\usepackage{background}

\usetikzlibrary{arrows,automata,calc,positioning}

\newcommand{\task}[2]{\par \noindent \textbf{{#1}.} \hspace{3pt} #2 \vspace{10pt}}
\newcommand{\solution}{\vspace{10pt}}
\newcommand{\pipesep}{\hspace{3pt} \vert \hspace{3pt}}

\newenvironment{subtasklist}[0]{\begin{enumerate}[label=(\alph*)]}{\end{enumerate}}
\newenvironment{mysolution}[1]{
	\par \textbf{Riešenie} \newline
	\ifthenelse{\equal{#1}{subtasks}}{\begin{enumerate}[label=(\alph*)]}
			{\begin{enumerate}[label={}] \item}
}{\end{enumerate} \newpage}
\newcommand{\subtask}{\item}

\SetBgContents{
		\parbox{0.4\textwidth}{
			\raggedleft
				Marek Milkovič \\
				\texttt{xmilko01}
		}
}
\SetBgScale{0.75}
\SetBgOpacity{1}
\SetBgAngle{0}
\SetBgColor{black}
\SetBgPosition{current page.north east}
\SetBgVshift{-0.8cm}
\SetBgHshift{-3.5cm}
 
\begin{document}

\title{TIN 2015/2016: Úloha 2}
\author{Marek Milkovič \\ \small\texttt{xmilko01@stud.fit.vutbr.cz}}
\maketitle

\task{1}{Dokážte alebo vyvráťte, že je jazyk $L = \{ a^{i}b^{j}c^{i}d^{j} \pipesep i,j \in
	\mathbb{N} \}$ bezkontextový.}

\begin{mysolution}

	Použijeme pumping lemmu pre bezkontexové jazyky podľa vety 4.19 v študíjnej opore. Nech je teda
	$L$ bezkontextový jazyk. Potom platí
	\begin{align*}
		\exists p > 0 : \forall z \in L :
		\vert z \vert \ge p \Rightarrow
		\exists u,v,w,x,y \in \Sigma^{*} :&\ z = uvwxy\ \land \\
									  &\ vx \ne \varepsilon\ \land \\
									  &\ \vert vwx \vert \le p\ \land \\
									  &\ \forall i \ge 0 : uv^{i}wx^{i}y \in L\
	\end{align*}
	Nech reťazec $z = a^{p}b^{p}c^{p}d^{p}$. Určite platí $|z| \ge p$, lebo vieme, že
	$|z| = 4p$. Nakoľko $|vwx| \le p$, tak môže nastať len istý počet možností toho, v akom
	tvare je $vwx$. Tie sú
	\begin{enumerate}[label=\arabic*.]
		\item $vwx$ sa skladá výhradne zo symbolov $a$
		\item $vwx$ sa skladá výhradne zo symbolov $b$
		\item $vwx$ sa skladá výhradne zo symbolov $c$
		\item $vwx$ sa skladá výhradne zo symbolov $d$
		\item $vwx$ sa nachádza na rozhraní symbolov $a,b$
		\item $vwx$ sa nachádza na rozhraní symbolov $b,c$
		\item $vwx$ sa nachádza na rozhraní symbolov $c,d$
	\end{enumerate}

	Najskôr vyriešime prípad číslo 1. Vieme, že pokiaľ $uv^{i}wx^{i}y$ má patriť
	do $L$, tak je to ekvivalentné tomu, že $a^{p + (i-1)|vx|}b^{p}c^{p}d^{p}$ má
	patriť do $L$. Pokiaľ položíme $i = 0$, tak $a^{p - |vx|}b^{p}c^{p}d^{p}$ má
	patriť do $L$. Vieme však, že musí platiť $\#_{a}(z) = \#_{c}(z)$. Potom hľadáme
	také $|vx|$, aby platilo $p - |vx| = p$. To však platí výhradne pre $|vx| = 0$. Z podmienok
	pumping lemmy však vieme, že $vx \not= \varepsilon$, a teda $|vx| \not= 0$. To je
	spor.

	Prípady 2,3 a 4 sú analogické ku 1. V prípade číslo 3, by sme len určili reťazec
	$uv^{i}wx^{i}y$ ako $a^{p}b^{p}c^{p + (i-1)|vx|}d^{p}$ a znova by sme dospeli k rovnakému
	sporu. V prípadoch číslo 3 a 4 by sme analogicky určili reťazce rovnako ako pre dvojicu
	$a,c$ a vychádzali z toho, že $\#_{b}(z) = \#_{d}(z)$. Dospejeme k rovnakému sporu.

	V prípadoch čislo 5,6 a 7 zanedbáme také tvary reťazca $uvwxy$, kedy podreťazec $v$ alebo $x$
	sa nebude skladať výhradne z jediného symbolu. Takéto prípady môžeme zanedbať kvôli tomu,
	že pri ich mocnení v reťazci $uv^{i}wx^{i}y$ by došlo k porušeniu poradia symbolov, ktoré
	musí byť v každom prípade zachované. Budeme preto uvažovať, že rozhranie symbolov na ktorých
	sa $vwx$ nachádza, je súčasťou podreťazca $w$.

	V prípade číslo 5 môžeme zapísať, že $uv^{i}wx^{i}y$ odpovedá $a^{p + (i-1)|v|}b^{p + (i-1)|x|}c^{p}d^{p}$.
	Položmé $i = 0$. Potom možeme zapísať, že $uv^{0}wx^{0}y = a^{p - |v|}b^{p - |x|}c^{p}d^{p}$.
	Hľadáme teda také $|v|$ a $|x|$, aby platilo $p - |v| = p$ a zároveň $p - |x| = p$, pretože vieme, že
	musí platiť $\#_{a}(z) = \#_{c}(z)$ a $\#_{b}(z) = \#_{d}(z)$. To platí len pre $|v| = 0$ a $|x| = 0$.
	Z toho ale vyplýva, že aj $|vx| = 0$. Z podmienok pumping lemmy ale musí platiť $vx \not= \varepsilon$
	a teda $|vx| \not= 0$. Dostávame sa do sporu.

	Prípady čislo 6 a 7 môžu byť vyriešené analogicky, ako prípad číslo 5. Nakoľko sme sa dostali v každom
	prípade do sporu, tak aj pôvodné tvrdenie, že $L$ je bezkontextový jazyk nemôže platiť. Jazyk $L$ tým
	pádom nie je bezkontextový.
\end{mysolution}

\task{2}{S využitím uzáverových vlastností bezkontextových a regulárnych jazykov dokážte
	alebo vyvráťte, že je jazyk $L \setminus L'$ nutne bezkontextový, a to pre každý zo štyroch
	nasledujúcich prípadov.
	\begin{subtasklist}
	\item $L \in \mathcal{L}_{3}, L' \in \mathcal{L}_{3}$
	\item $L \in \mathcal{L}_{3}, L' \in \mathcal{L}_{2}$
	\item $L \in \mathcal{L}_{2}, L' \in \mathcal{L}_{3}$
	\item $L \in \mathcal{L}_{2}, L' \in \mathcal{L}_{2}$
	\end{subtasklist}
}

\begin{mysolution}{subtasks}
	\subtask Podľa vety 3.22 zo študíjnej opory vieme, že trieda regulárnych jazykov $\mathcal{L}_{3}$
	je uzavrená voči zjednoteniu $\cup$, konkatenácií $\cdot$ a iterácií $*$. Taktiež podľa vety 3.23
	vieme, že trieda regulárnych jazykov $\mathcal{L}_{3}$ tvorí množinovú Booleovu algebru. Rozdiel
	$L \setminus L'$ si potom môžeme pomocou DeMorganových zákonov vyjadriť nasledovne.
	\begin{equation*}
		L \setminus L' = L \cap \overline{L'} = \overline{\overline{L \cap \overline{L'}}} =
		\overline{\overline{L} \cup \overline{\overline{L'}}} = \overline{\overline{L} \cup L'}
	\end{equation*}
	Z Booleovej algebry plynie uzavretosť voči komplementu a z definície regulárnych množín
	a ich ekvivalencií s regulárnymi jazykmi plynie uzavretosť voči $\cup$. Tým pádom
	$L \setminus L'$ je jazyk regulárny.

	\subtask Tu znova DeMorgan a ukazat, ze sice ich prienik je CF, ale ten vysledny komplement uz CF nie je
	\subtask Komplement je uzavrety na L3, prienik s L2 je znova L2, takze toto plati
	\subtask Tu to kazi jak komplement, tak prienik, urcite nie je CF
\end{mysolution}

\task{3}{Uvažujme bezkontextové gramatiky nad abecedou $\Sigma \subseteq \mathbb{Z}$.
	Definujme váhu slova $w = k_{1}...k_{2} \in \Sigma^{*}$ ako
	$\vert\vert w \vert\vert = \sum\limits_{i=1}^{n} k_{i}$. Váha bezkontextovej gramatiky $G$
	nad abecedou $\Sigma$ je potom definovaná ako minimálna váha slova jejho jazyka,
	$\vert\vert G \vert\vert = \text{min}\{\vert\vert w \vert\vert \pipesep w \in L(G)\}$.
	Minimum množiny čísel je definované štandardným spôsobom s tým, že
	$\text{min}(\varnothing) = \infty$, a pokiaľ je $S$ neprázdna množina, ktorá neobsahuje
	minimálny prvok, je $\text{min}(S) = -\infty$. Pre $-\infty$ a $\infty$ počítame s pravidlami
	$\text{min}(\{-\infty\} \cup S) = -\infty$ a $\text{min}(\{\infty\} \cup S) = \text{min}(S)$.
	Navrhnite algoritmus, ktorý pre danú $G$ vráti $\vert\vert G \vert\vert$, a to
	\begin{subtasklist}
	\item najskôr pre prípad, kedy $\Sigma \subset \mathbb{N}$, teda medzi symbolmi nie sú
		záporné čísla,
	\item a potom vo všeobecnom prípade, kedy $\Sigma \subset \mathbb{Z}$.
	\end{subtasklist}}

\begin{mysolution}

	TODO
\end{mysolution}

\task{4}{Dokážte formálne, že $L \subseteq L(G)$, kde
	$L = \{0^{i}1^{j} \pipesep 0 \le 2i \le j \le 3 \}$ a $G = (\{S\},\{0,1\},P,S)$ je
	bezkontextová gramatika s pravidlami
	\begin{equation*}
		S \to 0S11 \pipesep 0S111 \pipesep \varepsilon .
	\end{equation*}
	Dôkaz veďte indukciou k počtu symbolov $0$ v slove $w \in L$.}

\begin{mysolution}

	TODO
\end{mysolution}

\task{5}{Majme gramatiku $G = (\{S\},\{\textbf{if},\textbf{else},cond,com\},P,S)$ s pravidlami
\begin{equation*}
	S \to \textbf{if}\ cond\ S \pipesep \textbf{if}\ cond\ S\ \textbf{else}\ S \pipesep com .
\end{equation*}
\begin{subtasklist}
	\item Je gramatika jednoznačná? Dokážte.
	\item Je $L(G)$ jazyk s inherentnou viacznačnosťou? Zdôvodnite.
	\item Navrhnite deterministický zásobníkový automat akceptujúci $L(G)$. Prezentujte ho
		prechodovým diagramom a demonštrujte jeho beh na slove
	\begin{equation*}
	\textbf{if}\ cond\ \textbf{if}\ cond\ com\ \textbf{else}\ com\ \textbf{else}\ 
	\textbf{if}\ cond\ com.
	\end{equation*}
\end{subtasklist}}

\begin{mysolution}{subtasks}
	\subtask Podľa definície 4.5 zo študíjnej opory hovoríme, že veta $w$ generovaná gramatikou $G$
	je viacznačná, pokiaľ existujú aspoň 2 rôzne derivačné stromy s koncovými uzlami tvoriacmi vetu $w$.
	Gramatiku $G$ potom nazýva viacznačná, pokiaľ generuje aspoň jednu viacznačnú vetu. Stačí nám teda
	vygenerovať gramatikou $G$ nejakú vetu, pre ktorú vieme zostrojiť aspoň 2 derivačné stromy a tým
	dokázať jej viacznačnosť. Takou vetou je napríklad veta
	\begin{equation*}
	\textbf{if}\ cond\ \textbf{if}\ cond\ com\ \textbf{else}\ com
	\end{equation*}
	Táto veta mohla vzniknúť dvoma možnými deriváciami, ktoré odpovedajú dvom možným derivačným stromom.
	\begin{figure}[!h]
		\centering
		\begin{minipage}[b]{0.45\textwidth}
			\centering
			\begin{align*}
				S \Rightarrow\ &\textbf{if}\ cond\ S \Rightarrow \\
							   &\textbf{if}\ cond\ \textbf{if}\ cond\ S\ \textbf{else}\ S \Rightarrow \\
							   &\textbf{if}\ cond\ \textbf{if}\ cond\ com\ \textbf{else}\ S \Rightarrow \\
							   &\textbf{if}\ cond\ \textbf{if}\ cond\ com\ \textbf{else}\ com
			\end{align*}
		\end{minipage}
		\begin{minipage}[b]{0.45\textwidth}
			\centering
			\begin{align*}
				S \Rightarrow\ &\textbf{if}\ cond\ S\ \textbf{else}\ S \Rightarrow \\
							   &\textbf{if}\ cond\ \textbf{if}\ cond\ S\ \textbf{else}\ S \Rightarrow \\
							   &\textbf{if}\ cond\ \textbf{if}\ cond\ com\ \textbf{else}\ S \Rightarrow \\
							   &\textbf{if}\ cond\ \textbf{if}\ cond\ com\ \textbf{else}\ com
			\end{align*}
		\end{minipage}
	\end{figure}
	\begin{figure}[!h]
		\centering
		\begin{minipage}[b]{0.45\textwidth}
			\begin{tikzpicture}[level 1/.style={sibling distance = 1.75cm}, level 2/.style={sibling distance = 1.5cm}]
				\node {$S$}
					child
					{
						node {\textbf{if}} edge from parent
					}
					child
					{
						node {$cond$} edge from parent
					}
					child
					{
						node {$S$} edge from parent
							child
							{
								node {\textbf{if}} edge from parent
							}
							child
							{
								node {$cond$} edge from parent
							}
							child
							{
								node {$S$} edge from parent
									child
									{
										node {$com$} edge from parent
									};
							}
							child
							{
								node {\textbf{else}} edge from parent
							}
							child
							{
								node {$S$} edge from parent
									child
									{
										node {$com$} edge from parent
									};
							};
					};
			\end{tikzpicture}
		\end{minipage}
		\begin{minipage}[b]{0.45\textwidth}
			\begin{tikzpicture}[level 1/.style={sibling distance = 1.75cm}, level 2/.style={sibling distance = 1.5cm}]
				\node {$S$}
					child
					{
						node {\textbf{if}} edge from parent
					}
					child
					{
						node {$cond$} edge from parent
					}
					child
					{
						node {$S$} edge from parent
							child
							{
								node {\textbf{if}} edge from parent
							}
							child
							{
								node {$cond$} edge from parent
							}
							child
							{
								node {$S$} edge from parent
									child
									{
										node {$com$} edge from parent
									};
							};
					}
					child
					{
						node {\textbf{else}} edge from parent
					}
					child
					{
						node {$S$} edge from parent
							child
							{
								node {$com$} edge from parent
							};
					};
			\end{tikzpicture}
		\end{minipage}
	\end{figure}
	\subtask Podľa definície 4.5 zo študíjnej opory hovoríme o jazyku s inherentnou viacznačnosťou, pokiaľ neexistuje
	jednoznačna gramatika generujúca tento jazyk. Určenie, či je gramatika viacznačna, alebo nie, je problém nerozhodnuteľný,
	preto nie je možné algoritmicky dokázať jednoznačnosť, avšak môžeme intuitívne lokalizovať problém, prečo je daná gramatika
	viacznačna a vhodnými úpravami z nej spraviť gramatiku jednoznačnú. V našej gramatike $G$ je problém ten, že nie je možné
	jednoznačne vytvoriť pár \textbf{if}-\textbf{else}, ktorý patrí k sebe. Túto gramatiku transformujeme na gramatiku
	$G' = (N \cup \{S'\},\Sigma,P',S)$.
	\begin{align*}
		P': S &\to \textbf{if}\ cond\ S \pipesep \textbf{if}\ cond\ S'\ \textbf{else}\ S \pipesep com \\
		S' &\to \textbf{if}\ cond\ S'\ \textbf{else}\ S' \pipesep com
	\end{align*}
	Táto gramatika rieši spomínaný problém tak, že vždy spáruje najbližšií \textbf{if}-\textbf{else} a tým povoluje tvorbu
	len jednej z možných derivácií. Veta $\textbf{if}\ cond\ \textbf{if}\ cond\ com\ \textbf{else}\ com$ má v tejto gramatike
	len jediný derivačný strom a to nasledovný. Môžeme teda predpokladať, aj keď bez dôkazu, že tento jazyk je bez inherentnej viacznačnosti.
	\begin{figure}[!h]
		\centering
		\begin{minipage}[ht]{0.45\textwidth}
		\centering
			\begin{tikzpicture}[level 1/.style={sibling distance = 1.75cm}, level 2/.style={sibling distance = 1.5cm}]
				\node {$S$}
					child
					{
						node {\textbf{if}} edge from parent
					}
					child
					{
						node {$cond$} edge from parent
					}
					child
					{
						node {$S$} edge from parent
							child
							{
								node {\textbf{if}} edge from parent
							}
							child
							{
								node {$cond$} edge from parent
							}
							child
							{
								node {$S'$} edge from parent
									child
									{
										node {$com$} edge from parent
									};
							}
							child
							{
								node {\textbf{else}} edge from parent
							}
							child
							{
								node {$S$} edge from parent
									child
									{
										node {$com$} edge from parent
									};
							};
					};
			\end{tikzpicture}
		\end{minipage}
		\begin{minipage}[ht]{0.45\textwidth}
			\centering
			\begin{align*}
				S \Rightarrow\ &\textbf{if}\ cond\ S \Rightarrow \\
							   &\textbf{if}\ cond\ \textbf{if}\ cond\ S'\ \textbf{else}\ S \Rightarrow \\
							   &\textbf{if}\ cond\ \textbf{if}\ cond\ com\ \textbf{else}\ S \Rightarrow \\
							   &\textbf{if}\ cond\ \textbf{if}\ cond\ com\ \textbf{else}\ com
			\end{align*}
		\end{minipage}
	\end{figure}
	\subtask Zostrojíme DZKA $P = (Q,\Sigma,\Gamma,\delta,q_{0},z_{0},F)$, ktorý bude prijímať jazyk $L(G) = L(P)$
	prechodom do koncového stavu. Dôvodom je to, že v jazyku $L(G)$ existujú také 2 reťazce, že jeden je prefixom druhého.
	V prípade takéhoto jazyka nie je možné, aby bol prijímaný DKZA vyprázdnením zásobníku.

	\begin{figure}[!h]
		\begin{minipage}[ht]{0.25\textwidth}
			\begin{flalign*}
				&Q = \{q_{1},q_{2},q_{3},q_{4},q_{5}\} & \\
				&\Sigma = \{\textbf{if},\textbf{else},cond,com\} & \\
				&\Gamma = \{\textbf{if}, Z\} & \\
				&q_{0} = q_{1} & \\
				&z_{0} = Z & \\
				&F = \{q_{4}\} &
			\end{flalign*}
		\end{minipage}
		\begin{minipage}[h]{0.85\textwidth}
			\centering
			\begin{tikzpicture}[node distance = 2.95cm, ->, >=stealth, line width=1pt, scale=0.9, every node/.style={transform shape}]
				\node[state, initial, initial text=$Z$] (q1) {$q_{1}$};
				\node[state, right of = q1] (q2) {$q_{2}$};
				\node[state, right of = q2] (q3) {$q_{3}$};
				\node[state, accepting, right of = q3] (q4) {$q_{4}$};
				\node[state, below of = q4] (q5) {$q_{5}$};

				\path (q1) edge node[above]{\small$\textbf{if},Z/\textbf{if}$} (q2);
				\path (q2) edge node[above]{\small$cond,\textbf{if}/\textbf{if}$} (q3);
				\path (q3) edge[bend right=60] node[above]{\small$\textbf{if},\textbf{if}/\textbf{if\ if}$} (q2);
				\path (q3) edge node[above]{\small$com,\textbf{if}/\textbf{if}$} (q4);
				\path (q4) edge node[right]{\small$\textbf{else},\textbf{if}/\textbf{if}$} (q5);
				\path (q5) edge[bend left] node[left]{\small$com,\textbf{if}/\varepsilon$} (q4);
				\path (q5) edge[bend left=30] node[above]{\small$\textbf{if},\textbf{if}/\textbf{if}$} (q2);
			\end{tikzpicture}
		\end{minipage}
	\end{figure}

\end{mysolution}

\end{document}
