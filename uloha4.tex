\documentclass[12pt]{article}

\usepackage[margin=1in]{geometry} 
\usepackage{amsmath,amsthm,amssymb,amsfonts}
\usepackage[slovak]{babel}
\usepackage[utf8]{inputenc}
\usepackage{enumitem}
\usepackage{ifthen}
\usepackage{tikz}
\usepackage{tikz-qtree}
\usepackage{caption}
\usepackage{background}
\usepackage{float}
\restylefloat{table}

\usetikzlibrary{arrows,automata,calc,positioning}

\newcommand{\task}[2]{\par \noindent \textbf{{#1}.} \hspace{3pt} #2 \vspace{10pt}}
\newcommand{\solution}{\vspace{10pt}}
\newcommand{\pipesep}{\hspace{3pt} \vert \hspace{3pt}}

\newenvironment{subtasklist}[0]{\begin{enumerate}[label=(\alph*)]}{\end{enumerate}}
\newenvironment{mysolution}[1]{
	\par \textbf{Riešenie} \newline
	\ifthenelse{\equal{#1}{subtasks}}{\begin{enumerate}[label=(\alph*)]}
			{\begin{enumerate}[label={}] \item}
}{\end{enumerate} \newpage}
\newcommand{\subtask}{\item}

\SetBgContents{
		\parbox{0.4\textwidth}{
			\raggedleft
				Marek Milkovič \\
				\texttt{xmilko01}
		}
}
\SetBgScale{0.75}
\SetBgOpacity{1}
\SetBgAngle{0}
\SetBgColor{black}
\SetBgPosition{current page.north east}
\SetBgVshift{-0.8cm}
\SetBgHshift{-4.5cm}
 
\begin{document}
 
\title{TIN 2015/2016: Úloha 4}
\author{Marek Milkovič \\ \small\texttt{xmilko01@stud.fit.vutbr.cz}}
\maketitle

\task{1}{Pomocou počiatočných funkcií a operátorov kombinácie, kompozície a primitívnej rekurzie
vyjadrite funkciu počítajúcu tretiu odmocninu (zaokrúhlenú dolu na celé čísla):
\begin{equation*}
sqrt3: \mathbb{N} \to \mathbb{N}, sqrt3(x) = z \text{ také, že }z^{3} \le x \land (z+1)^{3} > x
\end{equation*}
Je možné použiť funkcie $plus(x,y), mult(x,y), monus(x,y)$ a $eq(x,y)$ definované v predáškach.
Okrem nich však nepoužívajte žiadne ďalšie funkcie zavedené na prednáškach mimo funkcie počiatočné.
Nepoužívajte zjednodušenú syntax zápisu funkcií -- dodržte presne definičný tvar operátorov
kombinácie, kompozície a primitívnej rekurzie.}

\begin{mysolution}

\vspace{-1cm}
\begin{align*}
	pow3(x) &= mult \circ (\pi_{1}^{1}(x) \times mult \circ (\pi_{1}^{1}(x) \times \pi_{1}^{1}(x))) \\
	\\
	sqrt3(0) &= \xi() \\
	sqrt3(x+1) &= plus \circ (\pi_{2}^{2}(x,sqrt(x)) \times (eq \circ (\sigma \circ \pi_{1}^{2}(x,sqrt(x)) \times (pow3 \circ \sigma \circ \pi_{2}^{2}(x,sqrt(x)))))) \\
\end{align*}
\end{mysolution}

\task{2}{Podľa dôkazu vety 11.4 z prednášok je každý výpočetný proces vykonávaný TS vyčíslením
parciálnej funkcie $f(w)$ v tomto dôkaze. Vyčíslite túto funkciu pre stroj $M_{2}$ zobrazený
na obrázku 1 a slowo $w = ab$.

Vyčíslenie funkcie $step$ na jednotlivé konfigurácie TS nie je treba rozpisovať do detailov.
Rozpíšte iba vyčíslenie funkcie $cursym$ pre každú konfiguráciu, na ktorú budete aplikovať $step$,
a následne môžete zapísať priamo výsledok funkcie $step$ ako $step(w_{i},p_{i},n_{i}) =
(w_{i+1},p_{i+1},n_{i+1})$. Rovnako tak nie je potrebné znovu vyčíslovať funkcie, ktorých
hodnoty už máte spočítané (jedná sa predovšetkým o funkciu $step$ na danej konfigurácií).

\begin{figure}[!h]
\centering
\begin{tikzpicture}[node distance = 2.5cm, ->, >=stealth, line width=1pt]
\node[state, initial, initial text=] (q0) {$q_{1}$};
\node[state, right of = q0] (q1) {$q_{2}$};
\node[state, right of = q1] (q2) {$q_{3}$};
\node[state, accepting, below of = q2] (q3) {$q_{4}$};

\path (q0) edge node[above]{$\Delta/R$} (q1);
\path (q1) edge node[above]{$a/R$} (q2);
\path (q2) edge[loop above] node[above]{$b/R$} (q2);
\path (q2) edge node[left]{$\Delta/R$} (q3);
\end{tikzpicture}
\caption{Stroj $M_{2}$}
\end{figure}}

\begin{mysolution}

\begin{align*}
	&q_{1} \mapsto 1 & q_{3} \mapsto 3& \\
	&q_{2} \mapsto 2 & q_{4} \mapsto 0&
\end{align*}
\begin{align*}
	&\Delta \mapsto 0 & a \mapsto 1 && b \mapsto 2&
\end{align*}
\begin{align*}
	w_{0} &= \Delta a b \Delta^{\omega} \approx 210_{3} \\
	p_{0} &= q_{1} \approx 1 \\
	n_{0} &= 1
\end{align*}
\begin{flalign*}
	cursym(w_{0},p_{0},n_{0}) &= cursym(210_{3},1,1)
		= quo(210_{3}, 3^{0})\ \dot{-}\ mult(3, quo(210_{3}, 3^{1})) = \\
		&= 210_{3}\ \dot{-}\ mult(3, 21_{3}) = 210_{3}\ \dot{-}\ 210_{3} = 0 \\
	step(w_{0},p_{0},n_{0}) &= step(210_{3},1,1) = (210_{3},2,2) \\
	\\
	cursym(w_{1},p_{1},n_{1}) &= cursym(210_{3},2,2)
		= quo(210_{3}, 3^{1})\ \dot{-}\ mult(3,quo(210_{3},3^{2})) = \\
		&= 21_{3}\ \dot{-}\ mult(3, 2_{3}) = 21_{3}\ \dot{-}\ 20_{3} = 1 \\
	step(w_{1},p_{1},n_{1}) &= step(210_{3},2,2) = (210_{3},3,3) \\
\end{flalign*}
\begin{flalign*}
	cursym(w_{2},p_{2},n_{2}) &= cursym(210_{3},3,3)
		= quo(210_{3}, 3^{2})\ \dot{-}\ mult(3,quo(210_{3},3^{3})) = \\
		&= 2_{3}\ \dot{-}\ mult(3, 0_{3}) = 2_{3}\ \dot{-}\ 0_{3} = 2 \\
	step(w_{2},p_{2},n_{2}) &= step(210_{3},3,3) = (210_{3},3,4) \\
	\\
	cursym(w_{3},p_{3},n_{3}) &= cursym(210_{3},3,4)
		= quo(210_{3}, 3^{3})\ \dot{-}\ mult(3,quo(210_{3},3^{4})) = \\
		&= 0_{3}\ \dot{-}\ mult(3, 0_{3}) = 0_{3}\ \dot{-}\ 0_{3} = 0 \\
	step(w_{3},p_{3},n_{3}) &= step(210_{3},3,4) = (210_{3},0,5) \\
\end{flalign*}
\end{mysolution}

\task{4}{Klub turistov plánuje vyznačiť v horách zimné turistické trasy. Každá trasa začína a končí
v nejakom \emph{rozcestí} a prechádza cez niekoľko ďalších rozcestí. Každá trasa bude označená
jednou z $k$ farieb a klub by rád vyznačil trasy tak, aby sa v žiadnom rozcestí nestretli
dve trasy označené rovnakou farbou.

Dokážte redukciou z niektorého známeho NP-ťažkého problému, že problém
\emph{"postačuje k farieb"} je NP-ťažký.}

\begin{mysolution}

Označme problém \emph{postačuje k farieb} ako \emph{COL}. Tento problém môžeme chápať taktiež ako problém
\emph{k-ofarbitelnosti grafu} (angl. \emph{k-graph coloring}). Jednotlivé trasy reprezentujeme uzlami v neorientovanom grafe.
Pokiaľ sa nejaké trasy stretávajú v rovnakom rozcestí, tak budú uzly reprezentujúce tieto trasy vzájomne spojené hranami.
Pridelenie farby uzlu v grafe potom odpovedá vyznačeniu trasy farbou.

Problém \emph{COL} a jeho instancie môžeme formálne definovať nasledovne.
\begin{equation*}
	\emph{COL} = \{\langle(G,k)\rangle \pipesep G\text{ je graf, ktorý je }k\text{-ofarbiteľný}\}
\end{equation*}
$\langle(G,k)\rangle$ je kódom grafu $G$ a kladného čísla $k$. Graf $G = (V,E)$, chápeme ako dvojicu skladajúcu sa z množiny uzlov $V$
a množiny hrán $E$. Množinu uzlov $V$ môžeme zakódovať ako postupnosť binárnych císel oddelených nejakým jednoznačným oddelovačom, napr. $|$. Množinu
hrán $E$ môžeme oddeliť od uzlov jednoznačným oddelovačom, napr. $\#$ a následne kódovať hrany ako dvojice uzlov, ktoré sú vzájomne prepojené.

Budeme postupovať polynomiálne vyčísliteľnou redukciou z problému \emph{3-SAT} na \emph{COL}. Redukcia bude založená
na transformácie formule v KNF (Konjuktivnéj Normálnej Forme) na graf, ktorý bude $k$-ofarbiteľný vtedy a len vtedy,
keď je daná formula splniteľná.
\begin{itemize}
	\item Farbenie grafu bude odpovedať ohodnocovaniu premenných vo formuli. Nakoľko však farieb môže existovať viacej ako booleovských hodnôť, tak iba
		2 farby budeme reprezentovať ako hodnoty \emph{true} a \emph{false}. Zvyšné farby nás nebudú zaujímať, ale nijak nesmú ovplyvniť riešenie.
	\item Uzly v grafe budú mať viac možných významov. Budú však existovať minimálne nasledovné uzly.
		\begin{itemize}
			\item Farebná paleta, ktorá sa bude skladať z $k$ uzlov, ktoré budú všetky prepojené navzájom. 
				Označme ich ako $T,F,C_{1}, ..., C_{k-2}$.
				Nakoľko sú všetky navzájom prepojené, tak je zjavné, že každý z nich bude mať pri valídnom ofarbení jedinečnú farbu.
				Uzly $T$ a $F$ budú reprezentovať
				zjavnú farbu pridelenú booleovským pravdivostným hodnotám \emph{true} a \emph{false}.
			\item Uzly reprezentujúce všetky možné literály nad všetkými premennými. Tieto uzly budú vždy označené $x_{i}$ a $\overline{x_{i}}$,
				pričom takto 2 označené uzly budú prepojené
				\begin{itemize}
					\item Navzájom, aby mali vždy rozdielnu farbu.
					\item S uzlami $C_{1},..., C_{k-2}$, aby vždy nadobudli inú farbu a teda reprezentovali niektorú z 2 pravdivostných hodnôt.
				\end{itemize}
			\item Uzly reprezentujúce realizáciu dizjunkcie medzi literálmi v klauzuli. Táto realizácia bude mať v grafe nasledovnú podobu.
				\begin{figure}[!h]
					\centering
					\begin{tikzpicture}
						\node[state] (a) {$A$};
						\node[state, below = 1cm of a] (b) {$B$};
						\node[state, right = 1cm of a] (aa) {};
						\node[state, right = 1cm of b] (bb) {};
						\node[state, below right = 0.4cm and 1cm of aa] (c) {};

						\path[line width = 1pt] (a) edge (aa);
						\path[line width = 1pt] (b) edge (bb);
						\path[line width = 1pt] (aa) edge (bb);
						\path[line width = 1pt] (aa) edge (c);
						\path[line width = 1pt] (bb) edge (c);
					\end{tikzpicture}
				\end{figure}

				Nech $K$ je klauzula $x_{1} \lor ... \lor x_{m}$ kde $m \in \{1,2,3\}$. Pre každé $x_{i} \lor x_{i+1}$ vytvoríme jednu spomínanú realizáciu dizjunkcie,
				kým nepokryjeme celú klauzulu. Uzol udávajúci hodnotu celej klauzule spojíme s uzlami $F,C_{1},...,C_{k-2}$, čím vynútime, že chceme aby klauzula bola
				splnená, teda mala hodnotu \emph{true}. Pokiaľ je $k > 3$, tak ešte navyše spojíme každý uzol s uzlami $C_{2},...,C_{k-2}$, s ktorými nebudeme chcieť pracovať.
		\end{itemize}
\end{itemize}
Formálny popis polynomiálnej redukcie, ktorú je možné realizovať pomocou DTS, je potom nasledovný.
\begin{itemize}
	\item Skontroluje sa správnosť vstupnej formule a ohodnotenia premenných. Takúto kontrolu je možné vykonať
		so zložitosťou $O(n)$ jednosmerným priechodom cez vstupný reťazec, pretože nie je potrebné robiť žiadne spätné kontroly.
	\item Na druhej, pracovnej, páske sa vytvoria uzly a hrany farebnej palety so zložitosťou $O(1)$.
	\item Pre každý možný literál každej premennej sa vytvorí uzol $x_{i}$ a $\overline{x_{i}}$, s tým že musí prebehnúť kontrola,
		či už taký uzol neexistuje. Tieto 2 uzly sa prepoja hranou. To je možné so zložitosťou $O(n^{2})$.
	\item Pre každú klauzulu sa musí vytvoriť realizácia dizjunkcie. K tomu patrí priechod cez všetky klauzule a vyhľadanie potrebných
		uzlov k jednotlivým literálom. Zložitosť je tým pádom $O(n^{2})$.
\end{itemize}

Zostáva ukázať, že formula v KNF je splniteľná $\Leftrightarrow$ je graf $G$ $k$-ofarbiteľný.
\begin{itemize}
	\item $\Rightarrow$
		\begin{itemize}
			\item Pokiaľ budú všetky literály ohodnotené \emph{false}, tak aj celá klauzula bude mať hodnotu \emph{false}.
				Nakoľko je formula v KNF, tak ak je jej jedna klauzula \emph{false}, tak celá formula nie je splniteľná.
				Výsledný uzol realizácie dizjunkcie je však spojený s uzlami $F,C_{1},...,C_{k-2}$ a tým pádom tento uzol nemôže byť
				ofarbený ako $F$, jedine ako $T$. Ak teda každá klauzula bude mať hodnotu \emph{true}, tak aj každý koncový uzol klauzule
				bude mať farbu ako $T$ a vtedy aj graf $k$-ofarbiteľný.
		\end{itemize}
	\item $\Leftarrow$
		\begin{itemize}
			\item Jednotlivé uzly farebnej palety $T,F,C_{1},...,C_{k-2}$
				budú mať unikátne farby, kvôli ich vzájomnému prepojeniu. Týchto farieb bude presne $k$. Uzly reprezentujúce literály
				nadobudnú farbu buďto ako $T$, alebo $F$. Vieme, že každý uzol realizujúci dizjunkciu je prepojený s $C_{2},...,C_{k-2}$, takže
				ich farby nie je možné v tejto štruktúre nadobúdať. Taktiež vieme, že posledný uzol pre klauzulu je navyše spojený s uzlami
				$C_{1}$ a $F$, čím pre tento uzol pripadá len farba uzlu $T$. Tento uzol je spojený s dvojicov uzlov, ktoré sú navyše
				spojené navzájom. Pre túto trojicu je možné ofarbenie len farbami uzlov $C_{1},T,F$. Ak $T$ pripadá pre výsledný uzol,
				tak pre prechádzajúce dva pripadá len $C_{1}$ a $F$. To ale znamená, že aspoň jeden zo vstupov do realizácie dizjunkcie
				musel byť uzol s farbou ako $T$, inak by nebolo možné ofarbovať farbou $F$. Takto sa vieme dostať až ku prvotným literálom.
				Ich farba odpovedá farbe uzlov $T$ alebo $F$, teda ohodnoteniu \emph{true} alebo \emph{false}. Ukázali sme, že výsledok
				dizjunkcie musela byť farba $T$, teda ohodnotenie \emph{true}. To bolo možné len ak jeden z prvotných uzlov bol ofarbený ako $T$
				teda aspoň jeden mal ohodnotenie \emph{true}. Vtedy je aj formula splniteľná.
		\end{itemize}
\end{itemize}
Je zjavné, že redukciu je možné vykonať v deterministickom polynomiálnom čase, teda je to platná polynomiálna redukcia. Problém
\emph{COL} je teda \emph{NP}-ťažký.
\end{mysolution}
\end{document}
