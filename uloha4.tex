\documentclass[12pt]{article}

\usepackage[margin=1in]{geometry} 
\usepackage{amsmath,amsthm,amssymb,amsfonts}
\usepackage[slovak]{babel}
\usepackage[utf8]{inputenc}
\usepackage{enumitem}
\usepackage{ifthen}
\usepackage{tikz}
\usepackage{tikz-qtree}
\usepackage{caption}
\usepackage{background}
\usepackage{float}
\restylefloat{table}

\usetikzlibrary{arrows,automata,calc,positioning}

\newcommand{\task}[2]{\par \noindent \textbf{{#1}.} \hspace{3pt} #2 \vspace{10pt}}
\newcommand{\solution}{\vspace{10pt}}
\newcommand{\pipesep}{\hspace{3pt} \vert \hspace{3pt}}

\newenvironment{subtasklist}[0]{\begin{enumerate}[label=(\alph*)]}{\end{enumerate}}
\newenvironment{mysolution}[1]{
	\par \textbf{Riešenie} \newline
	\ifthenelse{\equal{#1}{subtasks}}{\begin{enumerate}[label=(\alph*)]}
			{\begin{enumerate}[label={}] \item}
}{\end{enumerate} \newpage}
\newcommand{\subtask}{\item}

\SetBgContents{
		\parbox{0.4\textwidth}{
			\raggedleft
				Marek Milkovič \\
				\texttt{xmilko01}
		}
}
\SetBgScale{0.75}
\SetBgOpacity{1}
\SetBgAngle{0}
\SetBgColor{black}
\SetBgPosition{current page.north east}
\SetBgVshift{-0.8cm}
\SetBgHshift{-4.5cm}
 
\begin{document}
 
\title{TIN 2015/2016: Úloha 4}
\author{Marek Milkovič \\ \small\texttt{xmilko01@stud.fit.vutbr.cz}}
\maketitle

\task{1}{Pomocou počiatočných funkcií a operátorov kombinácie, kompozície a primitívnej rekurzie
vyjadrite funkciu počítajúcu tretiu odmocninu (zaokrúhlenú dolu na celé čísla):
\begin{equation*}
sqrt3: \mathbb{N} \to \mathbb{N}, sqrt3(x) = z \text{ také, že }z^{3} \le x \land (z+1)^{3} > x
\end{equation*}
Je možné použiť funkcie $plus(x,y), mult(x,y), monus(x,y)$ a $eq(x,y)$ definované v predáškach.
Okrem nich však nepoužívajte žiadne ďalšie funkcie zavedené na prednáškach mimo funkcie počiatočné.
Nepoužívajte zjednodušenú syntax zápisu funkcií -- dodržte presne definičný tvar operátorov
kombinácie, kompozície a primitívnej rekurzie.}

\begin{mysolution}

\vspace{-1cm}
\begin{align*}
	pow3(x) &= mult \circ (\pi_{1}^{1}(x) \times mult \circ (\pi_{1}^{1}(x) \times \pi_{1}^{1}(x))) \\
	\\
	sqrt3(0) &= \xi() \\
	sqrt3(x+1) &= plus \circ (\pi_{2}^{2}(x,sqrt(x)) \times (eq \circ (\pi_{1}^{2}(x,sqrt(x)) \times (pow3 \circ \sigma \circ \pi_{2}^{2}(x,sqrt(x)))))) \\
\end{align*}
\end{mysolution}

\task{2}{Podľa dôkazu vety 11.4 z prednášok je každý výpočetný proces vykonávaný TS vyčíslením
parciálnej funkcie $f(w)$ v tomto dôkaze. Vyčíslite túto funkciu pre stroj $M_{2}$ zobrazený
na obrázku 1 a slowo $w = ab$.

Vyčíslenie funkcie $step$ na jednotlivé konfigurácie TS nie je treba rozpisovať do detailov.
Rozpíšte iba vyčíslenie funkcie $cursym$ pre každú konfiguráciu, na ktorú budete aplikovať $step$,
a následne môžete zapísať priamo výsledok funkcie $step$ ako $step(w_{i},p_{i},n_{i}) =
(w_{i+1},p_{i+1},n_{i+1})$. Rovnako tak nie je potrebné znovu vyčíslovať funkcie, ktorých
hodnoty už máte spočítané (jedná sa predovšetkým o funkciu $step$ na danej konfigurácií).

\begin{figure}[!h]
\centering
\begin{tikzpicture}[node distance = 2.5cm, ->, >=stealth, line width=1pt]
\node[state, initial, initial text=] (q0) {$q_{1}$};
\node[state, right of = q0] (q1) {$q_{2}$};
\node[state, right of = q1] (q2) {$q_{3}$};
\node[state, accepting, below of = q2] (q3) {$q_{4}$};

\path (q0) edge node[above]{$\Delta/R$} (q1);
\path (q1) edge node[above]{$a/R$} (q2);
\path (q2) edge[loop above] node[above]{$b/R$} (q2);
\path (q2) edge node[left]{$\Delta/R$} (q3);
\end{tikzpicture}
\caption{Stroj $M_{2}$}
\end{figure}}

\begin{mysolution}

\begin{align*}
	&q_{1} \mapsto 1 & q_{3} \mapsto 3& \\
	&q_{2} \mapsto 2 & q_{4} \mapsto 0&
\end{align*}
\begin{align*}
	&\Delta \mapsto 0 & a \mapsto 1 && b \mapsto 2&
\end{align*}
\begin{align*}
	w_{0} &= \Delta a b \Delta^{\omega} \approx 210_{3} \\
	p_{0} &= q_{1} \approx 1 \\
	n_{0} &= 1
\end{align*}
\begin{flalign*}
	cursym(w_{0},p_{0},n_{0}) &= cursym(210_{3},1,1)
		= quo(210_{3}, 3^{0})\ \dot{-}\ mult(3, quo(210_{3}, 3^{1})) = \\
		&= 210_{3}\ \dot{-}\ mult(3, 21_{3}) = 210_{3}\ \dot{-}\ 210_{3} = 0 \\
	step(w_{0},p_{0},n_{0}) &= step(210_{3},1,1) = (210_{3},2,2) \\
	\\
	cursym(w_{1},p_{1},n_{1}) &= cursym(210_{3},2,2)
		= quo(210_{3}, 3^{1})\ \dot{-}\ mult(3,quo(210_{3},3^{2})) = \\
		&= 21_{3}\ \dot{-}\ mult(3, 2_{3}) = 21_{3}\ \dot{-}\ 20_{3} = 1 \\
	step(w_{1},p_{1},n_{1}) &= step(210_{3},2,2) = (210_{3},3,3) \\
\end{flalign*}
\begin{flalign*}
	cursym(w_{2},p_{2},n_{2}) &= cursym(210_{3},3,3)
		= quo(210_{3}, 3^{2})\ \dot{-}\ mult(3,quo(210_{3},3^{3})) = \\
		&= 2_{3}\ \dot{-}\ mult(3, 0_{3}) = 2_{3}\ \dot{-}\ 0_{3} = 2 \\
	step(w_{2},p_{2},n_{2}) &= step(210_{3},3,3) = (210_{3},3,4) \\
	\\
	cursym(w_{3},p_{3},n_{3}) &= cursym(210_{3},3,4)
		= quo(210_{3}, 3^{3})\ \dot{-}\ mult(3,quo(210_{3},3^{4})) = \\
		&= 0_{3}\ \dot{-}\ mult(3, 0_{3}) = 0_{3}\ \dot{-}\ 0_{3} = 0 \\
	step(w_{3},p_{3},n_{3}) &= step(210_{3},3,4) = (210_{3},0,5) \\
\end{flalign*}
\end{mysolution}

\task{3}{Dokážte, že pre každý jazyk $L \in \mathcal{L}_{2}$ platí, že $L \in \emph{NTIME}(n)$.}

\begin{mysolution}

TODO
\end{mysolution}

\task{4}{Klub turistov plánuje vyznačiť v horách zimné turistické trasy. Každá trasa začína a končí
v nejakom \emph{rozcestí} a prechádza cez niekoľko ďalších rozcestí. Každá trasa bude označená
jednou z $k$ farieb a klub by rád vyznačil trasy tak, aby sa v žiadnom rozcestí nestretli
dve trasy označené rovnakou farbou.

Dokážte redukciou z niektorého známeho NP-ťažkého problému, že problém
\emph{"postačuje k farieb"} je NP-ťažký.}

\begin{mysolution}

TODO
\end{mysolution}
\end{document}
