\documentclass[12pt]{article}

\usepackage[margin=1in]{geometry} 
\usepackage{amsmath,amsthm,amssymb,amsfonts}
\usepackage[slovak]{babel}
\usepackage[utf8]{inputenc}
\usepackage{enumitem}
\usepackage{ifthen}
\usepackage{tikz}
\usepackage{tikz-qtree}
\usepackage{caption}
\usepackage{background}
\usepackage{float}
\restylefloat{table}

\usetikzlibrary{arrows,automata,calc,positioning}

\newcommand{\task}[2]{\par \noindent \textbf{{#1}.} \hspace{3pt} #2 \vspace{10pt}}
\newcommand{\solution}{\vspace{10pt}}
\newcommand{\pipesep}{\hspace{3pt} \vert \hspace{3pt}}

\newenvironment{subtasklist}[0]{\begin{enumerate}[label=(\alph*)]}{\end{enumerate}}
\newenvironment{mysolution}[1]{
	\par \textbf{Riešenie} \newline
	\ifthenelse{\equal{#1}{subtasks}}{\begin{enumerate}[label=(\alph*)]}
			{\begin{enumerate}[label={}] \item}
}{\end{enumerate} \newpage}
\newcommand{\subtask}{\item}

\SetBgContents{
		\parbox{0.4\textwidth}{
			\raggedleft
				Marek Milkovič \\
				\texttt{xmilko01}
		}
}
\SetBgScale{0.75}
\SetBgOpacity{1}
\SetBgAngle{0}
\SetBgColor{black}
\SetBgPosition{current page.north east}
\SetBgVshift{-0.8cm}
\SetBgHshift{-4.5cm}
 
\begin{document}
 
\title{TIN 2015/2016: Úloha 4}
\author{Marek Milkovič \\ \small\texttt{xmilko01@stud.fit.vutbr.cz}}
\maketitle

\task{1}{Pomocou počiatočných funkcií a operátorov kombinácie, kompozície a primitívnej rekurzie
vyjadrite funkciu počítajúcu tretiu odmocninu (zaokrúhlenú dolu na celé čísla):
\begin{equation*}
sqrt3: \mathbb{N} \to \mathbb{N}, sqrt3(x) = z \text{ také, že }z^{3} \le x \land (z+1)^{3} > x
\end{equation*}
Je možné použiť funkcie $plus(x,y), mult(x,y), monus(x,y)$ a $eq(x,y)$ definované v predáškach.
Okrem nich však nepoužívajte žiadne ďalšie funkcie zavedené na prednáškach mimo funkcie počiatočné.
Nepoužívajte zjednodušenú syntax zápisu funkcií -- dodržte presne definičný tvar operátorov
kombinácie, kompozície a primitívnej rekurzie.}

\begin{mysolution}

\vspace{-1cm}
\begin{align*}
	pow3(x) &= mult \circ (\pi_{1}^{1}(x) \times mult \circ (\pi_{1}^{1}(x) \times \pi_{1}^{1}(x))) \\
	\\
	sqrt3(0) &= \xi() \\
	sqrt3(x+1) &= plus \circ (\pi_{2}^{2}(x,sqrt(x)) \times (eq \circ (\pi_{1}^{2}(x,sqrt(x)) \times (pow3 \circ \sigma \circ \pi_{2}^{2}(x,sqrt(x)))))) \\
\end{align*}
\end{mysolution}

\task{2}{Podľa dôkazu vety 11.4 z prednášok je každý výpočetný proces vykonávaný TS vyčíslením
parciálnej funkcie $f(w)$ v tomto dôkaze. Vyčíslite túto funkciu pre stroj $M_{2}$ zobrazený
na obrázku 1 a slowo $w = ab$.

Vyčíslenie funkcie $step$ na jednotlivé konfigurácie TS nie je treba rozpisovať do detailov.
Rozpíšte iba vyčíslenie funkcie $cursym$ pre každú konfiguráciu, na ktorú budete aplikovať $step$,
a následne môžete zapísať priamo výsledok funkcie $step$ ako $step(w_{i},p_{i},n_{i}) =
(w_{i+1},p_{i+1},n_{i+1})$. Rovnako tak nie je potrebné znovu vyčíslovať funkcie, ktorých
hodnoty už máte spočítané (jedná sa predovšetkým o funkciu $step$ na danej konfigurácií).

\begin{figure}[!h]
\centering
\begin{tikzpicture}[node distance = 2.5cm, ->, >=stealth, line width=1pt]
\node[state, initial, initial text=] (q0) {$q_{1}$};
\node[state, right of = q0] (q1) {$q_{2}$};
\node[state, right of = q1] (q2) {$q_{3}$};
\node[state, accepting, below of = q2] (q3) {$q_{4}$};

\path (q0) edge node[above]{$\Delta/R$} (q1);
\path (q1) edge node[above]{$a/R$} (q2);
\path (q2) edge[loop above] node[above]{$b/R$} (q2);
\path (q2) edge node[left]{$\Delta/R$} (q3);
\end{tikzpicture}
\caption{Stroj $M_{2}$}
\end{figure}}

\begin{mysolution}

\begin{align*}
	&q_{1} \mapsto 1 & q_{3} \mapsto 3& \\
	&q_{2} \mapsto 2 & q_{4} \mapsto 0&
\end{align*}
\begin{align*}
	&\Delta \mapsto 0 & a \mapsto 1 && b \mapsto 2&
\end{align*}
\begin{align*}
	w_{0} &= \Delta a b \Delta^{\omega} \approx 210_{3} \\
	p_{0} &= q_{1} \approx 1 \\
	n_{0} &= 1
\end{align*}
\begin{flalign*}
	cursym(w_{0},p_{0},n_{0}) &= cursym(210_{3},1,1)
		= quo(210_{3}, 3^{0})\ \dot{-}\ mult(3, quo(210_{3}, 3^{1})) = \\
		&= 210_{3}\ \dot{-}\ mult(3, 21_{3}) = 210_{3}\ \dot{-}\ 210_{3} = 0 \\
	step(w_{0},p_{0},n_{0}) &= step(210_{3},1,1) = (210_{3},2,2) \\
	\\
	cursym(w_{1},p_{1},n_{1}) &= cursym(210_{3},2,2)
		= quo(210_{3}, 3^{1})\ \dot{-}\ mult(3,quo(210_{3},3^{2})) = \\
		&= 21_{3}\ \dot{-}\ mult(3, 2_{3}) = 21_{3}\ \dot{-}\ 20_{3} = 1 \\
	step(w_{1},p_{1},n_{1}) &= step(210_{3},2,2) = (210_{3},3,3) \\
\end{flalign*}
\begin{flalign*}
	cursym(w_{2},p_{2},n_{2}) &= cursym(210_{3},3,3)
		= quo(210_{3}, 3^{2})\ \dot{-}\ mult(3,quo(210_{3},3^{3})) = \\
		&= 2_{3}\ \dot{-}\ mult(3, 0_{3}) = 2_{3}\ \dot{-}\ 0_{3} = 2 \\
	step(w_{2},p_{2},n_{2}) &= step(210_{3},3,3) = (210_{3},3,4) \\
	\\
	cursym(w_{3},p_{3},n_{3}) &= cursym(210_{3},3,4)
		= quo(210_{3}, 3^{3})\ \dot{-}\ mult(3,quo(210_{3},3^{4})) = \\
		&= 0_{3}\ \dot{-}\ mult(3, 0_{3}) = 0_{3}\ \dot{-}\ 0_{3} = 0 \\
	step(w_{3},p_{3},n_{3}) &= step(210_{3},3,4) = (210_{3},0,5) \\
\end{flalign*}
\end{mysolution}

\task{3}{Dokážte, že pre každý jazyk $L \in \mathcal{L}_{2}$ platí, že $L \in \emph{NTIME}(n)$.}

\begin{mysolution}

TODO
\end{mysolution}

\task{4}{Klub turistov plánuje vyznačiť v horách zimné turistické trasy. Každá trasa začína a končí
v nejakom \emph{rozcestí} a prechádza cez niekoľko ďalších rozcestí. Každá trasa bude označená
jednou z $k$ farieb a klub by rád vyznačil trasy tak, aby sa v žiadnom rozcestí nestretli
dve trasy označené rovnakou farbou.

Dokážte redukciou z niektorého známeho NP-ťažkého problému, že problém
\emph{"postačuje k farieb"} je NP-ťažký.}

\begin{mysolution}

Označme problém \emph{postačuje k farieb} ako \emph{COL}. Tento problém môžeme chápať taktiež ako problém
\emph{k-ofarbitelnosti grafu} (angl. \emph{k-graph coloring}). Jednotlivé trasy reprezentujeme uzlami v neorientovanom grafe.
Pokiaľ sa nejaké trasy stretávajú v rovnakom rozcestí, tak budú uzly reprezentujúce tieto trasy vzájomne spojené hranami.
Pridelenie farby uzlu v grafe potom odpovedá vyznačeniu trasy farbou.

Problém \emph{COL} a jeho instancie môžeme formálne definovať nasledovne.
\begin{equation*}
	\emph{COL} = \{\langle(G,k)\rangle \pipesep G\text{ je graf, ktory je }k\text{-ofarbitelny}\}
\end{equation*}
$\langle(G,k)\rangle$ je kódom grafu $G$ a kladného čísla $k$. Graf $G = (V,E)$, chápeme ako dvojicu skladajúcu sa z množiny uzlov $V$
a množiny hrán $E$. Množinu uzlov $V$ môžeme zakódovať ako postupnosť binárnych císel oddelených nejakým jednoznačným oddelovačom, napr. $|$. Množinu
hrán $E$ môžeme oddeliť od uzlov jednoznačným oddelovačom, napr. $\#$ a následne kódovať hrany ako dvojice uzlov, ktoré sú vzájomne prepojené.

\textbf{1. \emph{COL}$\ \in\ $\emph{NP}}\\
Zostrojíme 2-páskový NTS $M$, ktorý bude fungovať nasledovne.
\begin{itemize}
	\item TS najskôr overí, či je na vstupe platná instancia \emph{COL} problému. Toto overenie spočíva v kontrole, či každý uzol
		má pridelený jednoznačný identifikátor resp. kód. Túto kontrolu je možné vykonať so zložitosťou $O(n^{2})$ (prechod časťou vstupu
		-- $O(n)$ a pre každý uzol, kontrola unikátnosti -- $O(n)$).
	\item Následne prebehne
		kontrola či každá hrana obsahuje na oboch svojich koncoch platný kód uzlu. To je možné vykonať so složitosťou $O(n^{3})$
		(prechod časťou vstupu -- $O(n)$ a kontrola existencie uzla na jednom konci hrany -- $O(n)$).
	\item Po úvodnej kontrole, začne proces samotného ofarbovania. Na druhú pásku sa vygeneruje pre každý uzol
		nedeterministicky kód farby, tak aby bolo možné jednoznačne určiť, ktorý uzol ma pridelenú ktorú farbu.
		To sa vykoná so zložitosťou $O(n)$.
	\item Potom prebieha overenie toho, či je graf správne ofarbený, teda že žiadne dva uzly spojené priamou hranou nie sú ofarbené
		rovnakou farbou. Kontrola je možná so zložitosťou $O(n^{2})$ (priechod každým uzlom -- $O(n)$ a kontrola uzlov, ktoré
		sú s ním priamo spojené -- $O(n)$).
\end{itemize}
Je zjavné, že tento TS dokáže prijať jazyk problému \emph{COL} v nedeterministickom polynomiálnom čase, teda \emph{COL}$\ \in\ $\emph{NP}.

\textbf{2. \emph{COL} je \emph{NP}-ťažký}\\
Budeme postupovať polynomiálne vyčísliteľnou redukciou z problému \emph{SAT} na \emph{COL}. Redukcia bude založená
na transformácie formule v KNF (Konjuktivnéj Normálnej Forme) na graf, ktorý bude $k$-ofarbiteľný vtedy a len vtedy,
keď je daná formula splniteľná. $k$ bude odpovedať maximálnemu počtu literálov z ktorých sa skladajú klauzule vo formuli.
Idea redukcie je nasledovná.
\begin{itemize}
	\item Farbenie grafu bude odpovedať ohodnocovaniu premenných vo formuli. Nakoľko však farieb môže existovať viacej ako booleovských hodnôť, tak iba
		2 farby budeme reprezentovať ako hodnoty \emph{true} a \emph{false}. Zvyšné farby nás nebudú zaujímať, ale nijak nesmú ovplyvniť riešenie.
	\item Uzly v grafe budú mať viac možných významov. Budú však existovať minimálne nasledovné uzly.
		\begin{itemize}
			\item Farebná paleta, ktorá sa bude skladať z $k$ uzlov, ktoré budú všetky prepojené navzájom. Dva z týchto uzlov
				budú reprezentovať hodnoty \emph{true} a \emph{false}, resp. ich farba bude reprezentovať dané hodnoty.
				Označme tieto uzly ako $T$ pre \emph{true} a $F$ pre \emph{false}. Zvyšné označme ako $C_{1}, ..., C_{k-2}$.
				Nakoľko sú všetky navzájom prepojené, tak je zjavné, že každý z nich bude mať pri valídnom ofarbení jedinečnú farbu.
			\item Uzly reprezentujúce všetky možné literály nad všetkými premennými. Tieto uzly budú vždy označené $x_{i}$ a $\overline{x_{i}}$,
				pričom takto 2 označené uzly budú prepojené
				\begin{itemize}
					\item Navzájom, aby mali vždy rozdielnu farbu.
					\item S uzlami $C_{1},..., C_{k-2}$, aby vždy nadobudli rovnakú farbu ako uzly $T$ a $F$.
				\end{itemize}
			\item Uzly reprezentujúce realizáciu dizjunkcie medzi literálmi v klauzuli. Nech $K$ je klauzula $x_{1} \lor ... \lor x_{m}$.
				V takomto prípade vytvoríme $m$ uzlov, ktoré všetky navzájom prepojíme. Každý uzol reprezentujúci literál $x_{i}$, či $\overline{x_{i}}$
				z klauzule $K$ spojíme hranou s jedným z novo vytvorených uzlov. Všetkých týchto $m$ uzlov zároveň spojíme s uzlom $T$. To z toho dôvodu, že
				jediné ohodnotenie premenných, ktoré spôsobí nesplnitelnosť je také, kedy sa klauzula vyhodnotí ako \emph{false}. Vtedy dôjde k
				ofarbeniu minimálne 1 novo vzniknutého uzlu na farbu rovnakú ako uzol $T$. To však nie je validné hodnotenie, pokiaľ dojde k spojeniu s
				uzlom $T$.
		\end{itemize}
\end{itemize}
Formálny popis polynomiálnej redukcie, ktorú je možné realizovať pomocou DTS, je potom nasledovný.
\begin{itemize}
	\item Skontroluje sa správnosť vstupnej formule a ohodnotenia premenných. Takúto kontrolu je možné vykonať
		so zložitosťou $O(n)$ jednosmerným priechodom cez vstupný reťazec, pretože nie je potrebné robiť žiadne spätné kontroly.
	\item Zistí sa $k$ -- maximalný počet literálov v klauzuli pomocou jednoduchého priechodu so zložitosťou $O(n)$.
	\item Na druhej, pracovnej, páske sa vytvoria uzly a hrany farebnej palety so zložitosťou $O(1)$.
	\item Pre každý možný literál každej premennej sa vytvorí uzol $x_{i}$ a $\overline{x_{i}}$, s tým že musí prebehnúť kontrola,
		či už taký uzol neexistuje. Tieto 2 uzly sa prepoja hranou. To je možné so zložitosťou $O(n^{2})$.
	\item Pre každú klauzulu sa musí vytvoriť realizácia dizjunkcie. K tomu patrí priechod cez všetky klauzule a vyhľadanie potrebných
		uzlov k jednotlivým literálom. Zložitosť je tým pádom $O(n^{2})$.
	\item Taktiež je potrebné vytvoriť prepojenia uzlov s uzlom $T$, čo má v optimálnom prípade zložitosť $O(n)$.
\end{itemize}
Je zjavné, že redukciu je možné vykonať v deterministickom polynomiálnom čase, teda je to platná polynomiálna redukcia. Problém
\emph{COL} je teda \emph{NP}-ťažký. Nakoľko \emph{COL}$\ \in\ \emph{NP}$ a zároveň je \emph{NP}-ťažký, tak je aj \emph{NP}-úplný.

Pre názornú ukážku fungovania tejto redukcie pridávam konkrétnu instanciu \emph{SAT} problému trasnformovanú na instanciu \emph{COL} problému.

\begin{align*}
	(x_{1} \lor \overline{x_{2}} \lor x_{3}) \land (\overline{x_{1}} \lor x_{2} \lor \overline{x_{3}})
\end{align*}
\begin{figure}[!h]
	\centering
	\begin{tikzpicture}
	\node[state] (x1) {$x_{1}$};
	\node[state, below = 1.5cm of x1] (nx1) {$\overline{x_{1}}$};
	\node[state, below = 0.5cm of nx1] (x2) {$x_{2}$};
	\node[state, below = 1.5cm of x2] (nx2) {$\overline{x_{2}}$};
	\node[state, below = 0.5cm of nx2] (x3) {$x_{3}$};
	\node[state, below = 1.5cm of x3] (nx3) {$\overline{x_{3}}$};
	\node[state, above right = 0.5cm and 2cm of nx2] (c1) {$C_{1}$};
	\node[state, above right = 1cm and 2cm of c1] (T) {$T$};
	\node[state, below right = 1cm and 2cm of c1] (F) {$F$};
	\node[state, above left = 4cm and 0.5cm of c1] (or11) {};
	\node[state, above right = 0.5cm and 1cm of or11] (or12) {};
	\node[state, below right = 0.5cm and 1cm of or11] (or13) {};
	\node[state, below left = 4cm and 0.5cm of c1] (or21) {};
	\node[state, above right = 0.5cm and 1cm of or21] (or22) {};
	\node[state, below right = 0.5cm and 1cm of or21] (or23) {};

	\path[line width = 1pt] (x1) edge (nx1);
	\path[line width = 1pt] (x2) edge (nx2);
	\path[line width = 1pt] (x3) edge (nx3);
	\path[line width = 1pt] (x1) edge (c1);
	\path[line width = 1pt] (x2) edge (c1);
	\path[line width = 1pt] (x3) edge (c1);
	\path[line width = 1pt] (nx1) edge (c1);
	\path[line width = 1pt] (nx2) edge (c1);
	\path[line width = 1pt] (nx3) edge (c1);
	\path[line width = 1pt] (c1) edge (T);
	\path[line width = 1pt] (c1) edge (F);
	\path[line width = 1pt] (T) edge (F);

	\path[line width = 1pt] (or11) edge (or12);
	\path[line width = 1pt] (or11) edge (or13);
	\path[line width = 1pt] (or12) edge (or13);
	\path[line width = 1pt, bend right = 25] (T) edge (or11);
	\path[line width = 1pt, bend right = 40] (T) edge (or12);
	\path[line width = 1pt] (T) edge (or13);

	\path[line width = 1pt] (or11) edge (x1);
	\path[line width = 1pt] (or12) edge (nx2);
	\path[line width = 1pt] (or13) edge (x3);
	\path[line width = 1pt] (or21) edge[bend left = 60] (nx1);
	\path[line width = 1pt] (or22) edge (x2);
	\path[line width = 1pt] (or23) edge (nx3);

	\path[line width = 1pt] (or21) edge (or22);
	\path[line width = 1pt] (or21) edge (or23);
	\path[line width = 1pt] (or22) edge (or23);
	\path[line width = 1pt] (T) edge (or21);
	\path[line width = 1pt] (T) edge (or22);
	\path[line width = 1pt] (T) edge (or23);
	\end{tikzpicture}
\end{figure}


\end{mysolution}
\end{document}
