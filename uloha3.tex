\documentclass[12pt]{article}

\usepackage[margin=1in]{geometry} 
\usepackage{amsmath,amsthm,amssymb,amsfonts}
\usepackage[slovak]{babel}
\usepackage[utf8]{inputenc}
\usepackage{enumitem}
\usepackage{ifthen}
\usepackage{tikz}
\usepackage{tikz-qtree}
\usepackage{caption}
\usepackage{background}
\usepackage{float}
\restylefloat{table}

\usetikzlibrary{arrows,automata,calc,positioning}

\newcommand{\task}[2]{\par \noindent \textbf{{#1}.} \hspace{3pt} #2 \vspace{10pt}}
\newcommand{\solution}{\vspace{10pt}}
\newcommand{\pipesep}{\hspace{3pt} \vert \hspace{3pt}}

\newenvironment{subtasklist}[0]{\begin{enumerate}[label=(\alph*)]}{\end{enumerate}}
\newenvironment{mysolution}[1]{
	\par \textbf{Riešenie} \newline
	\ifthenelse{\equal{#1}{subtasks}}{\begin{enumerate}[label=(\alph*)]}
			{\begin{enumerate}[label={}] \item}
}{\end{enumerate} \newpage}
\newcommand{\subtask}{\item}

\SetBgContents{
		\parbox{0.4\textwidth}{
			\raggedleft
				Marek Milkovič \\
				\texttt{xmilko01}
		}
}
\SetBgScale{0.75}
\SetBgOpacity{1}
\SetBgAngle{0}
\SetBgColor{black}
\SetBgPosition{current page.north east}
\SetBgVshift{-0.8cm}
\SetBgHshift{-3.5cm}
 
\begin{document}
 
\title{TIN 2015/2016: Úloha 3}
\author{Marek Milkovič \\ \small\texttt{xmilko01@stud.fit.vutbr.cz}}
\maketitle

\task{1}{Zostrojte gramatiku generujúcu jazyk Turingovho stroja na obrázku. Pre nejaké slovo z tohto jazyka popíšte
prijímajúci beh tohoto TS a deriváciu tohto slova vo vašej gramatike.
	\begin{figure}[!h]
		\centering
		\begin{tikzpicture}[node distance = 2.3cm, ->, >=stealth, line width=1pt]
			\node[state, initial, initial text=] (0) {0};
			\node[state, right of = 0] (1) {1};
			\node[state, right of = 1] (2) {2};
			\node[state, above of = 2] (3) {3};
			\node[state, right = 3cm of 2] (5) {5};
			\node[state, above of = 5] (6) {6};
			\node[state, below of = 5] (7) {7};
			\node[state, right of = 7] (8) {8};
			\node[state, accepting, right of = 8] (9) {9};

			\path (0) edge node[above]{$\Delta/R$} (1);
			\path (1) edge node[above]{$a/R$} (2);
			\path (2) edge node[right]{$b/R$} (3);
			\path (3) edge[loop right] node[right]{$e/R$} (3);
			\path (3) edge[bend right] node[left]{$a/R$} (1);
			\path (2) edge node[above]{$\Delta/L$} (5);
			\path (5) edge node[left]{$a/\Delta$} (6);
			\path (5) edge[bend right=70] node[right]{$b/\Delta$} (6);
			\path (6) edge[bend right=70] node[left]{$\Delta/L$} (5);
			\path (5) edge node[left]{$\Delta/R$} (7);
			\path (7) edge node[above]{$\Delta/Y$} (8);
			\path (8) edge node[above]{$Y/L$} (9);
		\end{tikzpicture}
	\end{figure}
}

\begin{mysolution}

	\vspace{-1.5cm}
	\begin{flalign*}
		G &= (N,\Sigma,P,S) & \\
		N &= \{S,A\} & \\
		\Sigma &= \{a,b,e\} & \\
		P&: & \\
		  &S \to aA & \\
		  &A \to baaA \pipesep \varepsilon &
	\end{flalign*}

	Za ľubovolné slovo zvolme slovo $abaa$. Beh TS je znázornený konfiguráciou pásky nasledovne.

	\begin{align*}
		\underline{\Delta} abaa \Delta\Delta^{\omega} && \Delta aba\underline{a} \Delta\Delta^{\omega} && \Delta ab\underline{a}\Delta \Delta\Delta^{\omega}
		&&	\Delta \underline{a}\Delta\Delta\Delta \Delta\Delta^{\omega} && \Delta \underline{Y}\Delta\Delta\Delta \Delta\Delta^{\omega}\\
		\Delta \underline{a}baa \Delta\Delta^{\omega} && \Delta abaa \underline{\Delta}\Delta^{\omega} && \Delta ab\underline{\Delta}\Delta \Delta\Delta^{\omega}
		&&	\Delta \underline{\Delta}\Delta\Delta\Delta \Delta\Delta^{\omega} && \underline{\Delta} Y\Delta\Delta\Delta \Delta\Delta^{\omega}\\
		\Delta a\underline{b}aa \Delta\Delta^{\omega} && \Delta aba\underline{a} \Delta\Delta^{\omega} && \Delta a\underline{b}\Delta\Delta \Delta\Delta^{\omega}
		&&	\underline{\Delta} \Delta\Delta\Delta\Delta \Delta\Delta^{\omega}\\
		\Delta ab\underline{a}a \Delta\Delta^{\omega} && \Delta aba\underline{\Delta} \Delta\Delta^{\omega} && \Delta a\underline{\Delta}\Delta\Delta \Delta\Delta^{\omega} 
		&&	\Delta \underline{\Delta}\Delta\Delta\Delta \Delta\Delta^{\omega}
	\end{align*}

	Derivácia slova $abaa$ v gramatike $G$ je nasledovná.
	\begin{equation*}
		S \Rightarrow aA \Rightarrow abaaA \Rightarrow abaa
	\end{equation*}
\end{mysolution}

\task{2}{Technikou redukcie dokážte, že problém prázdnosti jazyka daného Turingového stroja nie je ani čiastočne rozhodnuteľný.}

\begin{mysolution}

	Definujme problém prázdnosti jazyka ako problém \emph{EP}$\ = \{\langle M \rangle \pipesep L(M) = \varnothing\}$.
	Budeme dokazovať redukciou z komplementu problému zastavenia \emph{co-HP}, ktorý nie je ani čiastočne rozhodnuteľný, čo plynie
	z vety 6.2.1 a 5.6.3 zo študíjnej opory. Ukážeme, že \emph{co-HP} je možné úplným TS $M_{\sigma}$ redukovať na problém \emph{EP}.
	Redukcia je definovaná zobrazením $\sigma: \{0,1,\#\}^{*} \to \{0,1\}^{*}$. $M_{\sigma}$, vykonávajúci toto zobrazenie priradí každému
	svojmu vstupu $x \in \{0,1,\#\}^{*}$ kód TS $\langle M_{x} \rangle$ fungujúceho nasledovne.
	\begin{enumerate}[label=\arabic*.]
		\item Pokiaľ $x$ nie je platnou instanciou \emph{co-HP}, teda nie je v tvare $x_{1}\#x_{2}$, tak TS $M_{x}$ automaticky príjme bez ohľadu na jeho vstup.
		\item Pokiaľ $x$ je platnou instanciou \emph{co-HP}, teda je v tvare $x_{1}\#x_{2}$, kde $x_{1}$ je platným kódom TS a $x_{2}$ platným kódom jeho vstupu,
			tak TS $M_{x}$ spustí simuláciu TS $x_{1}$ nad vstupom $x_{2}$ pomocou UTS. Pokiaľ $x_{1}$ zastaví na $x_{2}$, tak $M_{x}$ odmietne.
			Ak beh $x_{1}$ cyklí na $x_{2}$, tak $M_{x}$ príjme.
	\end{enumerate}

	Ukážeme, že $M_{\sigma}$ je možné skonštruovať ako úplný TS. $M_{\sigma}$ potrebuje v podstate vykonávať nasledujúce veci
	\begin{itemize}
		\item Zistiť, či $x$ je platnou instaciou \emph{co-HP}, čo odpovedá zisteniu príslušnosti v regulárnom jazyku.
		\item Zostrojiť $M_{x}$, ktorý automaticky príjme, čo odpovedá len prechodu do koncového stavu, alebo $M_{x}$, ktorý
			zmaže svoju vstupnú pásku, zapíše na ňu $x_{1}\#x_{2}$ a spustí UTS, aby odsimuloval beh $x_{1}$ nad vstupom $x_{2}$.
	\end{itemize}
	Zistenie príslušnosti do regulárneho jazyka je možné pomocou konečného automatu, ktorého simulácia na TS je možná čítaním vstupnej pásky
	a stálym posunom doprava. Zostrojiť $M_{x}$ je možné tak, že najskôr sa zostrojí časť kódu TS $M_{x}$, ktorá zmaže svoju vstupnú pásku a zapíše
	na ňu $x_{1}\#x_{2}$. Potom sa vloží časť predstavujúca UTS a to sa prepojí časťou, ktorá spustí simuláciu UTS po zapísaní $x_{1}\#x_{2}$.
	
	O $L(M_{x})$ vieme, že
	\begin{enumerate}[label=\arabic*.]
		\item $L(M_{x}) = \varnothing \Leftrightarrow x$ nie je v tvare $x_{1}\#x_{2}$ alebo je v tvare $x_{1}\#x_{2}$, ale beh $x_{1}$ na $x_{2}$ nikdy neskončí.
		\item $L(M_{x}) = \{0,1\}^{*} \Leftrightarrow x$ je v tvare $x_{1}\#x_{2}$ ale beh TS $x_{1}$ zastaví na $x_{2}$.
	\end{enumerate}
	Môžeme teda zapísať, že $\sigma$ zachováva príslušnosť do jazyka nasledovne.
	\begin{align*}
		\langle M_{x} \rangle &\in \emph{EP} \\
		&\Leftrightarrow \\
		L(M_{x}) &= \varnothing \\
		&\Leftrightarrow \\
		x\text{ nie je v tvare }x_{1}\#x_{2}\text{ alebo je v tvare }&x_{1}\#x_{2}\text{ ale beh }x_{1}\text{ na }x_{2}\text{ nikdy nezastaví} \\
		&\Leftrightarrow \\
		&x \in \emph{co-HP}
	\end{align*}
	Ukázali sme, že $M_{\sigma}$ vieme zkonštruovať ako úplny TS a že $\sigma$ zachováva príslušnosť do \emph{co-HP} a \emph{EP}. Vykonali sme teda platnú
	redukciu. Podľa vety 6.3.1 zo študíjnej opory ale plynie, že pokiaľ nie je jazyk \emph{co-HP} rekurzívne vyčísliteľný, tak ani jazyk \emph{EP} nie je jazyk
	rekurzívne vyčísliteľný. Takže jazyk \emph{EP} nie je rekurzívne vyčísliteľný a tým pádom nie je ani čiastočne rozhodnuteľný.
\end{mysolution}

\task{3}{Pre danú množinu slov $P$ povieme, že TS $P$-\emph{rozhoduje} jazyk $L$, pokiaľ pre všetky slová mimo $P$ zastaví,
	a zo slov mimo $P$ prijíma práve tie, ktoré patria do $L$. Jazyk je potom $P$-\emph{rozhodnuteľný}, práve keď existuje TS,
	ktorý ho $P$-\emph{rozhoduje}. Intuitívne je $P$-rozhodnuteľnosť "jednoduchšou" variantou rozhodnuteľnosti -- TS nemusí vedieť
	rozhodnúť slová z $P$. Dokážte alebo vyvráťte nasledujúce tvrdenia:
	\begin{subtasklist}
		\subtask Existuje konečná množina $P$, pre ktorú je \emph{HP} $P$-rozhodnuteľný.
		\subtask Existuje nekonečná množina $P$, pre ktorú je \emph{HP} $P$-rozhodnuteľný.
		\subtask Pre všetky nekonečné množiny $P$ je \emph{HP} $P$-rozhodnuteľný.
	\end{subtasklist}
V bode (c) skúste vybrať vhodnú nekonečnú množinu $P$ a modifikovať pre ňu prednášaný dôkaz nerozhodnuteľnosti \emph{HP} diagonalizáciou.}

	\begin{mysolution}{subtasks}
	\subtask Nakoľko má pri $P$-rozhodnuteľnosti existovať TS, ktorý zastaví pre
	všetky vstupy $x \not\in P$ a prijme vtedy ak $x \not\in P \land x \in\ $\emph{HP}, tak sa pokúsime zvoliť $P$ také, aby zahrňovalo všetky $x$, ktoré
	majú platný tvar kódovania TS a jeho vstupu -- $\langle M \rangle \# \langle w \rangle$ a zároveň TS $M$ cyklí nad vstupom $w$. To nám
	zaručí existenciu TS $N$, ktorý buďto rozpozná neplatný tvar kódovania TS a jeho vstupu, alebo príslušnosť slova do $P$ a vtedy abnormálne zastaví.
	Inak spustí simuláciu TS $M$ nad vstupom $w$, ktorá vždy zastaví a tým pádom vždy príjme. To však je spor s tým, že \emph{HP} je problém nerozhodnuteľný.
	Pokiaľ by totiž existovala taká konečná
	množina $P$, že by obsahovala všetky reťazce $\langle M \rangle \# \langle w \rangle$ také, že TS $M$ cyklí nad vstupom $w$, tak by bol problém
	\emph{HP} rozhodnuteľný, pretože by bolo možné príslušnosťou $\langle M \rangle \# \langle w \rangle$ do regulárneho jazyka $P$ zistiť, kedy má
	úply TS rozhodujúci \emph{HP} zastaviť. Neexistuje teda konečná množina $P$, pre ktorú je problém \emph{HP} $P$-rozhodnuteľný.
	\subtask Vyberme jednu instanciu problému \emph{HP} $\langle M_{1} \rangle \# \langle w_{1} \rangle$ takú, že TS $M_{1}$ zastaví nad vstupom $w_{1}$.
	Zvoľme $P = \{0,1,\#\}^{*} \setminus \{\langle M_{1} \rangle \# \langle w_{1} \rangle\}$. Tým pádom postačí, že TS $N$ $P$-rozhodujúci \emph{HP}
	bude fungovať tak, že si overí, či má na vstupe práve $\langle M_{1} \rangle \# \langle w_{1} \rangle$,  čo odpovedá zisteniu príslušnosti do regulárneho jazyka.
	Ak jeho vstup odpovedá $\langle M_{1} \rangle \# \langle w_{1} \rangle$, tak zastaví a príjme a ak nie, tak môže abnormálne zastaviť, ale môže aj cykliť. Existuje teda
	nekonečná množina $P$, pre ktorú je \emph{HP} $P$-rozhodnuteľný.
	\subtask Zvoľme $P = \{x \in \{0,1,\#\}^{*} \pipesep \forall x_{1},x_{2} \in \{0,1\}^{*}: x \not= x_{1}\#x_{2}\}$. V $P$ sa teda nachádzajú všetky
	reťazce také, ktorých tvar neodpovedá kódu TS a kódu vstupu oddelených $\#$. Tým pádom vieme, že všetky reťazce v tvare $x_{1}\#x_{2}$ sú mimo $P$.
	Pokiaľ má byť \emph{HP} $P$-rozhodnuteľný pre každú nekonečnú množinu $P$, tak musí byť aj pre nami definovanú. Dôkaz povedieme diagonalizáciou
	založenou na dôkaze nerozhodnuteľnosti \emph{HP} diagonalizáciou v študijnej opore na strane 126.
	\begin{itemize}
		\item Pre $x \in \{0,1\}^{*}$, nech $M_{x}$ je TS s kódom $x$. Pokiaľ $x$ nie je platný kód TS, tak $M_{x}$ je TS, ktorý len abnormálne zastaví.
		\item Zostavíme postupnosť $M_{\varepsilon}, M_{0}, M_{1}, M_{00}, ...$, ktorá zahrňuje všetky TS nad $\{0,1\}$.
		\item Uvažujeme nekonečnú maticu
			\begin{table}[H]
				\centering
				\begin{tabular}{llllll}
										& $\varepsilon$                    & $0$                    & $1$                    & $00$                    & ... \\
					$M_{\varepsilon}$   & $h(M_{\varepsilon},\varepsilon)$ & $h(M_{\varepsilon},0)$ & $h(M_{\varepsilon},1)$ & $h(M_{\varepsilon},00)$ & \ \\
					$M_{0}$			    & $h(M_{0},\varepsilon)$           & $h(M_{0},0)$           & $h(M_{0},1)$           & $h(M_{0},00)$           & \ \\
					$M_{1}$			    & $h(M_{1},\varepsilon)$           & $h(M_{1},0)$           & $h(M_{1},1)$           & $h(M_{1},00)$           & \ \\
					$M_{00}$			& $h(M_{00},\varepsilon)$          & $h(M_{00},0)$          & $h(M_{00},1)$          & $h(M_{00},00)$          & \ \\
					...                 &                                  &                        &                        &                         & \ 
				\end{tabular}
			\end{table}
		kde pre $h$ platí
		\begin{equation*}
			h(M_{x}, x) = \begin{cases}
				\textbf{Z} & M_{x}\text{ zastaví na }x \\
				\textbf{C} & M_{x}\text{ cyklí na }x \\
			\end{cases}
		\end{equation*}
		\item Predpokladajme, že existuje TS $K$, ktorý $P$-rozhoduje \emph{HP}. Teda inak povedané, zastaví pre každé
			$\langle M_{x} \rangle \# \langle x \rangle \not\in P$. Chovanie $K$ je definované nasledovne.
			\begin{enumerate}[label=\arabic*.]
				\item Príjme, pokiaľ $M_{x}$ zastaví na $x$ a $\langle M_{x} \rangle \# \langle x \rangle \not\in P$
				\item Odmietne, pokiaľ $M_{x}$ cyklí na $x$ a $\langle M_{x} \rangle \# \langle x \rangle \not\in P$
				\item Cyklí, pokiaľ $\langle M_{x} \rangle \# \langle x \rangle \in P$
			\end{enumerate}
			Vieme, že iný prípad číslo 3 aktuálne nastať nemôže, lebo $P$ obsahuje výhradne reťazce, ktoré nemajú správny tvar vstupu. Nebudeme ho teda ďalej uvažovať.
		\item Zostrojíme TS $N$, ktorý pre svoj vstup $x \in \{0,1\}^{*}$
			\begin{itemize}
				\item Zostrojí $M_{x}$ z $x$ a zapíše $\langle M_{x} \rangle\#x$ na svoju pásku
				\item Simuluje $K$ na svojej páske a pokiaľ $K$ odmietne, tak
					$N$ zastaví a ak $K$ príjme, tak $N$ začne cykliť.
			\end{itemize}
		\item Dostávame, že
			\begin{equation*}
				N\text{ zastaví na }x \Leftrightarrow K\text{ odmietne }\langle M_{x} \rangle \# \langle x \rangle \Leftrightarrow M_{x}\text{ cyklí na }x
			\end{equation*}
			Vieme s určitosťou povedať, že $\langle N \rangle \# \langle x \rangle \not\in P$. Taktiež vieme, že $N$ je rozdielny od každého $M_{x}$ minimálne
			v jednom vstupe a to práve $x$. Potom ale postupnosť $M_{\varepsilon}, M_{0}, M_{1}, M_{00}, ...$ nebola postupnosť všetkých TS nad $\{0,1\}$.
			Taký TS $K$ teda pre tento konkrétny prípad existovať nemôže.
	\end{itemize}

	Dokázali sme, že pre špecifické nekonečné $P$ nie je \emph{HP} $P$-rozhodnuteľný. Teda nemôže byť $P$-rozhodnuteľný pre každé nekonečné $P$.
	\end{mysolution}

\task{4}{Rendez-vous sieť je graf, v ktorého každom uzle beží proces vykonávajúci rovnaký konečne stavový program. Procesy spojené hranou
	(komunikačným kanálom) spolu komunikujú formou tzv. rendez-vous -- sú schopné sa atomicky (v jednom výpočetnom kroku) zhodnúť na informácií
	komunikovanej komunikačným kanálom. Komunikačné kanály sú o\-čís\-lo\-va\-né a procesy môžu reagovať na komunikáciu rôznymi kanálmi rôzne. Pokiaľ
	sa proces snaží o komunikáciu kanálom, na ktorého druhom konci nie je žiadny iný proces, tak je taký proces najskôr vytvorený, a potom prebehne
	komunikácia.

	Formálne je \emph{rendez-vous sieť} trojica $S = (A,P,K)$ kde $A = (Q, \Sigma \times \{1,2\}, \delta, q_{0}. F)$ je konečný automat popisujúci
	chovanie procesov (všetky sa chovajú rovnako). Abeceda automatu je tvorená pármi, kde prvá zložka $a \in \Sigma$ je komunikovanou správou,
	a druhá zložka, $1$ alebo $2$, je číslom komunikačného kanálu, ktorým sa má správa $a$ komunikovať. $P$ je konečná množina procesov a
	$K \subseteq P \times \{1,2\} \times P$ je množina \emph{komunikačných kanálov} označenými číslami $1$ alebo $2$. \emph{Konfigurácia} siete
	$S$ je dvojica $(S, stav)$, kde $stav: P \to Q$ je funkcia priradzujúca stavy procesom. Konfigurácia $(S' = (A,P',K'),stav')$ vznikne
	\emph{výpočetným krokom} z konfigurácie $(S,stav)$, písané $(S,stav) \to (S',stav')$, v nasledujúcich dvoch prípadoch:

	\begin{subtasklist}
		\subtask Dva existujúce procesy spojené komunikačným kanálom sa dohodnú na správe. Teda $S' = S$, a pre nejaké $u,v \in P$ a $i \in \{1,2\}$
		existuje kanál $(u,i,v) \in K$ a symbol $a \in \Sigma$ tak, že $stav'(u) \in \delta(stav(u),(a,i))$, $stav'(v) \in \delta(stav(v), (a,i))$ a
		$stav'(w) = stav(w)$ pre všetky $w \in P$ rôzne od $u$ a $v$.
		\subtask Nejaký proces $u \in P$ by rád komunikoval kanálom $i$, ale nie je týmto kanálom spojený so žiadnym iným procesom. V tom prípade sa
		najprv vytvorí proces v iniciálnom stave spojený s prvým procesom kanálom $i$, a potom prebehne komunikácia ako v predchádzajúcom prípade.
		Formálne, pre nejaké $i \in \{1,2\}$ neexistuje kanál $(u',i,u'') \in K$ kde $u = u'$ alebo $u = u''$, $P' = P \cup \{v\}$ kde $v \not\in P$
		je nový proces, $K' = K \cup \{(u,i,v)\}$, $stav'(u) \in \delta(stav(u),(a,i))$, $stav'(v) \in \delta(q_{0}, (a,i))$ a $stav'(w) = stav(w)$
		pre všetky $w \in P'$ rôzne od $u$ a $v$.
	\end{subtasklist}

	Povieme, že konfigurácia $(S_{0},\sigma_{0})$ je dosiahnuteľný koncový stav, pokiaľ existuje sekvencia výpočetných krokov
	$(S_{0},stav_{0}) \to (S_{1}, stav_{1}) \to ... \to (S_{n}, stav_{n})$ taká, že $stav_{n}(v) \in F$ pre nejaký proces $v$ siete $S_{n}$.
	
	Dokážte, že problém dosiahnuteľnosti koncového stavu z danej konfigurácie rendez-vous siete je nerozhodnuteľný. Postupujte redukciou z problému
	náležitosti slova do jazyka Turingového stroja. Redukcia bude založená na simulácií Turingovho stroja rendez-vous sietí. Budete potrebovať kódovať
	konfiguráciu Turingovho stroja (obsah pásky, pozícia hlavy, riadiaci stav) konfiguráciou siete, a simulovať krok výpočtu Turingovho stroja výpočetným
	krokom siete.}

	\begin{mysolution}

		TODO
	\end{mysolution}

\end{document}
