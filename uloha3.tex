\documentclass[12pt]{article}

\usepackage[margin=1in]{geometry} 
\usepackage{amsmath,amsthm,amssymb,amsfonts}
\usepackage[slovak]{babel}
\usepackage[utf8]{inputenc}
\usepackage{enumitem}
\usepackage{ifthen}
\usepackage{tikz}
\usepackage{tikz-qtree}
\usepackage{caption}
\usepackage{background}
\usepackage{float}
\restylefloat{table}

\usetikzlibrary{arrows,automata,calc,positioning}

\newcommand{\task}[2]{\par \noindent \textbf{{#1}.} \hspace{3pt} #2 \vspace{10pt}}
\newcommand{\solution}{\vspace{10pt}}
\newcommand{\pipesep}{\hspace{3pt} \vert \hspace{3pt}}

\newenvironment{subtasklist}[0]{\begin{enumerate}[label=(\alph*)]}{\end{enumerate}}
\newenvironment{mysolution}[1]{
	\par \textbf{Riešenie} \newline
	\ifthenelse{\equal{#1}{subtasks}}{\begin{enumerate}[label=(\alph*)]}
			{\begin{enumerate}[label={}] \item}
}{\end{enumerate} \newpage}
\newcommand{\subtask}{\item}

\SetBgContents{
		\parbox{0.4\textwidth}{
			\raggedleft
				Marek Milkovič \\
				\texttt{xmilko01}
		}
}
\SetBgScale{0.75}
\SetBgOpacity{1}
\SetBgAngle{0}
\SetBgColor{black}
\SetBgPosition{current page.north east}
\SetBgVshift{-0.8cm}
\SetBgHshift{-3.5cm}
 
\begin{document}
 
\title{TIN 2015/2016: Úloha 3}
\author{Marek Milkovič \\ \small\texttt{xmilko01@stud.fit.vutbr.cz}}
\maketitle

\task{1}{Zostrojte gramatiku generujúcu jazyk Turingovho stroja na obrázku. Pre nejaké slovo z tohto jazyka popíšte
prijímajúci beh tohoto TS a deriváciu tohto slova vo vašej gramatike.
	\begin{figure}[!h]
		\centering
		\begin{tikzpicture}[node distance = 2.3cm, ->, >=stealth, line width=1pt]
			\node[state, initial, initial text=] (0) {0};
			\node[state, right of = 0] (1) {1};
			\node[state, right of = 1] (2) {2};
			\node[state, above of = 2] (3) {3};
			\node[state, right = 3cm of 2] (5) {5};
			\node[state, above of = 5] (6) {6};
			\node[state, below of = 5] (7) {7};
			\node[state, right of = 7] (8) {8};
			\node[state, accepting, right of = 8] (9) {9};

			\path (0) edge node[above]{$\Delta/R$} (1);
			\path (1) edge node[above]{$a/R$} (2);
			\path (2) edge node[right]{$b/R$} (3);
			\path (3) edge[loop right] node[right]{$e/R$} (3);
			\path (3) edge[bend right] node[left]{$a/R$} (1);
			\path (2) edge node[above]{$\Delta/L$} (5);
			\path (5) edge node[left]{$a/\Delta$} (6);
			\path (5) edge[bend right=70] node[right]{$b/\Delta$} (6);
			\path (6) edge[bend right=70] node[left]{$\Delta/L$} (5);
			\path (5) edge node[left]{$\Delta/R$} (7);
			\path (7) edge node[above]{$\Delta/Y$} (8);
			\path (8) edge node[above]{$Y/L$} (9);
		\end{tikzpicture}
	\end{figure}
}

\begin{mysolution}

	\vspace{-1.5cm}
	\begin{flalign*}
		G &= (N,\Sigma,P,S) & \\
		N &= \{S,A\} & \\
		\Sigma &= \{a,b,e\} & \\
		P&: & \\
		  &S \to aA & \\
		  &A \to baaA \pipesep \varepsilon &
	\end{flalign*}

	Za ľubovolné slovo zvolme slovo $abaa$. Beh TS je znázornený konfiguráciou pásky nasledovne.

	\begin{align*}
		\underline{\Delta} abaa \Delta\Delta^{\omega} && \Delta aba\underline{a} \Delta\Delta^{\omega} && \Delta ab\underline{a}\Delta \Delta\Delta^{\omega}
		&&	\Delta \underline{a}\Delta\Delta\Delta \Delta\Delta^{\omega} && \Delta \underline{Y}\Delta\Delta\Delta \Delta\Delta^{\omega}\\
		\Delta \underline{a}baa \Delta\Delta^{\omega} && \Delta abaa \underline{\Delta}\Delta^{\omega} && \Delta ab\underline{\Delta}\Delta \Delta\Delta^{\omega}
		&&	\Delta \underline{\Delta}\Delta\Delta\Delta \Delta\Delta^{\omega} && \underline{\Delta} Y\Delta\Delta\Delta \Delta\Delta^{\omega}\\
		\Delta a\underline{b}aa \Delta\Delta^{\omega} && \Delta aba\underline{a} \Delta\Delta^{\omega} && \Delta a\underline{b}\Delta\Delta \Delta\Delta^{\omega}
		&&	\underline{\Delta} \Delta\Delta\Delta\Delta \Delta\Delta^{\omega}\\
		\Delta ab\underline{a}a \Delta\Delta^{\omega} && \Delta aba\underline{\Delta} \Delta\Delta^{\omega} && \Delta a\underline{\Delta}\Delta\Delta \Delta\Delta^{\omega} 
		&&	\Delta \underline{\Delta}\Delta\Delta\Delta \Delta\Delta^{\omega}
	\end{align*}

	Derivácia slova $abaa$ v gramatike $G$ je nasledovná.
	\begin{equation*}
		S \Rightarrow aA \Rightarrow abaaA \Rightarrow abaa
	\end{equation*}
\end{mysolution}

\task{2}{Technikou redukcie dokážte, že problém prázdnosti jazyka daného Turingového stroja nie je ani čiastočne rozhodnuteľný.}

\begin{mysolution}

	Definujme problém prázdnosti jazyka ako problém \emph{EP}$\ = \{\langle M \rangle \pipesep L(M) = \varnothing\}$.
	Budeme dokazovať redukciou z komplementu problému zastavenia \emph{co-HP}, ktorý nie je ani čiastočne rozhodnuteľný, čo plynie
	z vety 6.2.1 a 5.6.3 zo študíjnej opory. Ukážeme, že \emph{co-HP} je možné úplným TS $M_{\sigma}$ redukovať na problém \emph{EP}.
	Redukcia je definovaná zobrazením $\sigma: \{0,1,\#\}^{*} \to \{0,1\}^{*}$. $M_{\sigma}$, vykonávajúci toto zobrazenie priradí každému
	svojmu vstupu $x \in \{0,1,\#\}^{*}$ kód TS $\langle M_{x} \rangle$ fungujúceho nasledovne.
	\begin{enumerate}[label=\arabic*.]
		\item Pokiaľ $x$ nie je platnou instanciou \emph{co-HP}, teda nie je v tvare $x_{1}\#x_{2}$, tak TS $M_{x}$ automaticky príjme bez ohľadu na jeho vstup.
		\item Pokiaľ $x$ je platnou instanciou \emph{co-HP}, teda je v tvare $x_{1}\#x_{2}$, kde $x_{1}$ je platným kódom TS a $x_{2}$ platným kódom jeho vstupu,
			tak TS $M_{x}$ spustí simuláciu TS $x_{1}$ nad vstupom $x_{2}$ pomocou UTS. Pokiaľ $x_{1}$ zastaví na $x_{2}$, tak $M_{x}$ odmietne.
			Ak beh $x_{1}$ cyklí na $x_{2}$, tak $M_{x}$ príjme.
	\end{enumerate}

	Ukážeme, že $M_{\sigma}$ je možné skonštruovať ako úplný TS. $M_{\sigma}$ potrebuje v podstate vykonávať nasledujúce veci
	\begin{itemize}
		\item Zistiť, či $x$ je platnou instaciou \emph{co-HP}, čo odpovedá zisteniu príslušnosti v regulárnom jazyku.
		\item Zostrojiť $M_{x}$, ktorý automaticky príjme, čo odpovedá len prechodu do koncového stavu, alebo $M_{x}$, ktorý
			zmaže svoju vstupnú pásku, zapíše na ňu $x_{1}\#x_{2}$ a spustí UTS, aby odsimuloval beh $x_{1}$ nad vstupom $x_{2}$.
	\end{itemize}
	Zistenie príslušnosti do regulárneho jazyka je možné pomocou konečného automatu, ktorého simulácia na TS je možná čítaním vstupnej pásky
	a stálym posunom doprava. Zostrojiť $M_{x}$ je možné tak, že najskôr sa zostrojí časť kódu TS $M_{x}$, ktorá zmaže svoju vstupnú pásku a zapíše
	na ňu $x_{1}\#x_{2}$. Potom sa vloží časť predstavujúca UTS a to sa prepojí časťou, ktorá spustí simuláciu UTS po zapísaní $x_{1}\#x_{2}$.
	
	O $L(M_{x})$ vieme, že
	\begin{enumerate}[label=\arabic*.]
		\item $L(M_{x}) = \varnothing \Leftrightarrow x$ nie je v tvare $x_{1}\#x_{2}$ alebo je v tvare $x_{1}\#x_{2}$, ale beh $x_{1}$ na $x_{2}$ nikdy neskončí.
		\item $L(M_{x}) = \{0,1\}^{*} \Leftrightarrow x$ je v tvare $x_{1}\#x_{2}$ ale beh TS $x_{1}$ zastaví na $x_{2}$.
	\end{enumerate}
	Môžeme teda zapísať, že $\sigma$ zachováva príslušnosť do jazyka nasledovne.
	\begin{align*}
		\langle M_{x} \rangle &\in \emph{EP} \\
		&\Leftrightarrow \\
		L(M_{x}) &= \varnothing \\
		&\Leftrightarrow \\
		x\text{ nie je v tvare }x_{1}\#x_{2}\text{ alebo je v tvare }&x_{1}\#x_{2}\text{ ale beh }x_{1}\text{ na }x_{2}\text{ nikdy nezastaví} \\
		&\Leftrightarrow \\
		&x \in \emph{co-HP}
	\end{align*}
	Ukázali sme, že $M_{\sigma}$ vieme zkonštruovať ako úplny TS a že $\sigma$ zachováva príslušnosť do \emph{co-HP} a \emph{EP}. Vykonali sme teda platnú
	redukciu. Podľa vety 6.3.1 zo študíjnej opory ale plynie, že pokiaľ nie je jazyk \emph{co-HP} rekurzívne vyčísliteľný, tak ani jazyk \emph{EP} nie je jazyk
	rekurzívne vyčísliteľný. Takže jazyk \emph{EP} nie je rekurzívne vyčísliteľný a tým pádom nie je ani čiastočne rozhodnuteľný.
\end{mysolution}

\task{3}{Pre danú množinu slov $P$ povieme, že TS $P$-\emph{rozhoduje} jazyk $L$, pokiaľ pre všetky slová mimo $P$ zastaví,
	a zo slov mimo $P$ prijíma práve tie, ktoré patria do $L$. Jazyk je potom $P$-\emph{rozhodnuteľný}, práve keď existuje TS,
	ktorý ho $P$-\emph{rozhoduje}. Intuitívne je $P$-rozhodnuteľnosť "jednoduchšou" variantou rozhodnuteľnosti -- TS nemusí vedieť
	rozhodnúť slová z $P$. Dokážte alebo vyvráťte nasledujúce tvrdenia:
	\begin{subtasklist}
		\subtask Existuje konečná množina $P$, pre ktorú je \emph{HP} $P$-rozhodnuteľný.
		\subtask Existuje nekonečná množina $P$, pre ktorú je \emph{HP} $P$-rozhodnuteľný.
		\subtask Pre všetky nekonečné množiny $P$ je \emph{HP} $P$-rozhodnuteľný.
	\end{subtasklist}
V bode (c) skúste vybrať vhodnú nekonečnú množinu $P$ a modifikovať pre ňu prednášaný dôkaz nerozhodnuteľnosti \emph{HP} diagonalizáciou.}

	\begin{mysolution}{subtasks}
	\subtask Nakoľko má pri $P$-rozhodnuteľnosti existovať TS, ktorý zastaví pre
	všetky vstupy $x \not\in P$ a prijme vtedy ak $x \not\in P \land x \in\ $\emph{HP}, tak sa pokúsime zvoliť $P$ také, aby zahrňovalo všetky $x$, ktoré
	majú platný tvar kódovania TS a jeho vstupu -- $\langle M \rangle \# \langle w \rangle$ a zároveň TS $M$ cyklí nad vstupom $w$. To nám
	zaručí existenciu TS $N$, ktorý buďto rozpozná neplatný tvar kódovania TS a jeho vstupu, alebo príslušnosť slova do $P$ a vtedy abnormálne zastaví.
	Inak spustí simuláciu TS $M$ nad vstupom $w$, ktorá vždy zastaví a tým pádom vždy príjme. To však je spor s tým, že \emph{HP} je problém nerozhodnuteľný.
	Pokiaľ by totiž existovala taká konečná
	množina $P$, že by obsahovala všetky reťazce $\langle M \rangle \# \langle w \rangle$ také, že TS $M$ cyklí nad vstupom $w$, tak by bol problém
	\emph{HP} rozhodnuteľný, pretože by bolo možné príslušnosťou $\langle M \rangle \# \langle w \rangle$ do regulárneho jazyka $P$ zistiť, kedy má
	úply TS rozhodujúci \emph{HP} zastaviť. Neexistuje teda konečná množina $P$, pre ktorú je problém \emph{HP} $P$-rozhodnuteľný.
	\subtask Vyberme jednu instanciu problému \emph{HP} $\langle M_{1} \rangle \# \langle w_{1} \rangle$ takú, že TS $M_{1}$ zastaví nad vstupom $w_{1}$.
	Zvoľme $P = \{0,1,\#\}^{*} \setminus \{\langle M_{1} \rangle \# \langle w_{1} \rangle\}$. Tým pádom postačí, že TS $N$ $P$-rozhodujúci \emph{HP}
	bude fungovať tak, že si overí, či má na vstupe práve $\langle M_{1} \rangle \# \langle w_{1} \rangle$,  čo odpovedá zisteniu príslušnosti do regulárneho jazyka.
	Ak jeho vstup odpovedá $\langle M_{1} \rangle \# \langle w_{1} \rangle$, tak zastaví a príjme a ak nie, tak môže abnormálne zastaviť, ale môže aj cykliť. Existuje teda
	nekonečná množina $P$, pre ktorú je \emph{HP} $P$-rozhodnuteľný.
	\subtask Zvoľme $P = \{x \in \{0,1,\#\}^{*} \pipesep \forall x_{1},x_{2} \in \{0,1\}^{*}: x \not= x_{1}\#x_{2}\}$. V $P$ sa teda nachádzajú všetky
	reťazce také, ktorých tvar neodpovedá kódu TS a kódu vstupu oddelených $\#$. Tým pádom vieme, že všetky reťazce v tvare $x_{1}\#x_{2}$ sú mimo $P$.
	Pokiaľ má byť \emph{HP} $P$-rozhodnuteľný pre každú nekonečnú množinu $P$, tak musí byť aj pre nami definovanú. Dôkaz povedieme diagonalizáciou
	založenou na dôkaze nerozhodnuteľnosti \emph{HP} diagonalizáciou v študijnej opore na strane 126.
	\begin{itemize}
		\item Pre $x \in \{0,1\}^{*}$, nech $M_{x}$ je TS s kódom $x$. Pokiaľ $x$ nie je platný kód TS, tak $M_{x}$ je TS, ktorý len abnormálne zastaví.
		\item Zostavíme postupnosť $M_{\varepsilon}, M_{0}, M_{1}, M_{00}, ...$, ktorá zahrňuje všetky TS nad $\{0,1\}$.
		\item Uvažujeme nekonečnú maticu
			\begin{table}[H]
				\centering
				\begin{tabular}{llllll}
										& $\varepsilon$                    & $0$                    & $1$                    & $00$                    & ... \\
					$M_{\varepsilon}$   & $h(M_{\varepsilon},\varepsilon)$ & $h(M_{\varepsilon},0)$ & $h(M_{\varepsilon},1)$ & $h(M_{\varepsilon},00)$ & \ \\
					$M_{0}$			    & $h(M_{0},\varepsilon)$           & $h(M_{0},0)$           & $h(M_{0},1)$           & $h(M_{0},00)$           & \ \\
					$M_{1}$			    & $h(M_{1},\varepsilon)$           & $h(M_{1},0)$           & $h(M_{1},1)$           & $h(M_{1},00)$           & \ \\
					$M_{00}$			& $h(M_{00},\varepsilon)$          & $h(M_{00},0)$          & $h(M_{00},1)$          & $h(M_{00},00)$          & \ \\
					...                 &                                  &                        &                        &                         & \ 
				\end{tabular}
			\end{table}
		kde pre $h$ platí
		\begin{equation*}
			h(M_{x}, x) = \begin{cases}
				\textbf{Z} & M_{x}\text{ zastaví na }x \\
				\textbf{C} & M_{x}\text{ cyklí na }x \\
			\end{cases}
		\end{equation*}
		\item Predpokladajme, že existuje TS $K$, ktorý $P$-rozhoduje \emph{HP}. Teda inak povedané, zastaví pre každé
			$\langle M \rangle \# \langle w \rangle \not\in P$ na svojom vstupe. Chovanie $K$ je definované nasledovne.
			\begin{enumerate}[label=\arabic*.]
				\item Príjme, pokiaľ $M$ zastaví na $w$ a $\langle M \rangle \# \langle w \rangle \not\in P$
				\item Odmietne, pokiaľ $M$ cyklí na $w$ a $\langle M \rangle \# \langle w \rangle \not\in P$
				\item Cyklí, pokiaľ $\langle M \rangle \# \langle w \rangle \in P$
			\end{enumerate}
			Vieme, že prípad číslo 3 aktuálne nastať nemôže, lebo $P$ obsahuje výhradne reťazce, ktoré nemajú správny tvar vstupu. Nebudeme ho teda ďalej uvažovať.
		\item Zostrojíme TS $N$, ktorý má vstup $x \in \{0,1\}^{*}$. Ak $\langle N \rangle \# x \in P$, tak nás nezaujíma čo $N$ spraví, môže hoc aj cykliť, ale s určitosťou
			vieme, že $\langle N \rangle \# x \not\in P$ z dôvodu predpisu množiny $P$. TS $N$ sa teda bude chovať nasledovne.
			\begin{itemize}
				\item Zostrojí $M_{x}$ z $x$ a zapíše $\langle M_{x} \rangle\#x$ na svoju pásku
				\item Simuluje $K$ na svojej páske a pokiaľ $K$ odmietne, tak
					$N$ zastaví a ak $K$ príjme, tak $N$ začne cykliť.
			\end{itemize}
		\item Dostávame, že
			\begin{equation*}
				N\text{ zastaví na }x \Leftrightarrow K\text{ odmietne }\langle M_{x} \rangle \# \langle x \rangle \Leftrightarrow M_{x}\text{ cyklí na }x
			\end{equation*}
			Vieme, že $N$ je rozdielny od každého $M_{x}$ minimálne
			v jednom vstupe a to práve $x$. Potom ale postupnosť $M_{\varepsilon}, M_{0}, M_{1}, M_{00}, ...$ nebola postupnosť všetkých TS nad $\{0,1\}$.
			Taký TS $K$ teda pre tento konkrétny prípad $P$ existovať nemôže.
	\end{itemize}

	Dokázali sme, že pre špecifické nekonečné $P$ nie je \emph{HP} $P$-rozhodnuteľný. Teda nemôže byť $P$-rozhodnuteľný pre každé nekonečné $P$.
	\end{mysolution}

\task{4}{Rendez-vous sieť je graf, v ktorého každom uzle beží proces vykonávajúci rovnaký konečne stavový program. Procesy spojené hranou
	(komunikačným kanálom) spolu komunikujú formou tzv. rendez-vous -- sú schopné sa atomicky (v jednom výpočetnom kroku) zhodnúť na informácií
	komunikovanej komunikačným kanálom. Komunikačné kanály sú o\-čís\-lo\-va\-né a procesy môžu reagovať na komunikáciu rôznymi kanálmi rôzne. Pokiaľ
	sa proces snaží o komunikáciu kanálom, na ktorého druhom konci nie je žiadny iný proces, tak je taký proces najskôr vytvorený, a potom prebehne
	komunikácia.

	Formálne je \emph{rendez-vous sieť} trojica $S = (A,P,K)$ kde $A = (Q, \Sigma \times \{1,2\}, \delta, q_{0}. F)$ je konečný automat popisujúci
	chovanie procesov (všetky sa chovajú rovnako). Abeceda automatu je tvorená pármi, kde prvá zložka $a \in \Sigma$ je komunikovanou správou,
	a druhá zložka, $1$ alebo $2$, je číslom komunikačného kanálu, ktorým sa má správa $a$ komunikovať. $P$ je konečná množina procesov a
	$K \subseteq P \times \{1,2\} \times P$ je množina \emph{komunikačných kanálov} označenými číslami $1$ alebo $2$. \emph{Konfigurácia} siete
	$S$ je dvojica $(S, stav)$, kde $stav: P \to Q$ je funkcia priradzujúca stavy procesom. Konfigurácia $(S' = (A,P',K'),stav')$ vznikne
	\emph{výpočetným krokom} z konfigurácie $(S,stav)$, písané $(S,stav) \to (S',stav')$, v nasledujúcich dvoch prípadoch:

	\begin{subtasklist}
		\subtask Dva existujúce procesy spojené komunikačným kanálom sa dohodnú na správe. Teda $S' = S$, a pre nejaké $u,v \in P$ a $i \in \{1,2\}$
		existuje kanál $(u,i,v) \in K$ a symbol $a \in \Sigma$ tak, že $stav'(u) \in \delta(stav(u),(a,i))$, $stav'(v) \in \delta(stav(v), (a,i))$ a
		$stav'(w) = stav(w)$ pre všetky $w \in P$ rôzne od $u$ a $v$.
		\subtask Nejaký proces $u \in P$ by rád komunikoval kanálom $i$, ale nie je týmto kanálom spojený so žiadnym iným procesom. V tom prípade sa
		najprv vytvorí proces v iniciálnom stave spojený s prvým procesom kanálom $i$, a potom prebehne komunikácia ako v predchádzajúcom prípade.
		Formálne, pre nejaké $i \in \{1,2\}$ neexistuje kanál $(u',i,u'') \in K$ kde $u = u'$ alebo $u = u''$, $P' = P \cup \{v\}$ kde $v \not\in P$
		je nový proces, $K' = K \cup \{(u,i,v)\}$, $stav'(u) \in \delta(stav(u),(a,i))$, $stav'(v) \in \delta(q_{0}, (a,i))$ a $stav'(w) = stav(w)$
		pre všetky $w \in P'$ rôzne od $u$ a $v$.
	\end{subtasklist}

	Povieme, že konfigurácia $(S_{0},\sigma_{0})$ je dosiahnuteľný koncový stav, pokiaľ existuje sekvencia výpočetných krokov
	$(S_{0},stav_{0}) \to (S_{1}, stav_{1}) \to ... \to (S_{n}, stav_{n})$ taká, že $stav_{n}(v) \in F$ pre nejaký proces $v$ siete $S_{n}$.
	
	Dokážte, že problém dosiahnuteľnosti koncového stavu z danej konfigurácie rendez-vous siete je nerozhodnuteľný. Postupujte redukciou z problému
	náležitosti slova do jazyka Turingového stroja. Redukcia bude založená na simulácií Turingovho stroja rendez-vous sietí. Budete potrebovať kódovať
	konfiguráciu Turingovho stroja (obsah pásky, pozícia hlavy, riadiaci stav) konfiguráciou siete, a simulovať krok výpočtu Turingovho stroja výpočetným
	krokom siete.}

	\begin{mysolution}

		Problém náležitosti slova do jazyka TS definujeme ako \emph{MP}$\ = \{\langle M \rangle \# \langle w \rangle \pipesep w \in L(M)\}$. Problém
		dosiahnuteľnosti koncového stavu z danej konfigurácie rendez-vous siete definujeme ako
		\begin{align*}
			\emph{RNP}\ = \{\langle (S_{0}, stav_{0}) \rangle \pipesep & \exists (S_{n}, stav_{n}): (S_{0}, stav_{0}) \to^{*} (S_{n}, stav_{n})\ \land \\
																	   & S_{n} = (A = (Q, \Sigma \times \{1,2\}, \delta, q_{0}, F), P_{n}, K_{n})\ \land \\
																	   & \exists p \in P_{n}: stav_{n}(p) \in F\}
		\end{align*}
		Ako $\langle (S_{0}, stav_{0}) \rangle$ sa myslí kód konfigurácie rendez-vous siete, ktorý vznikne postupným zobrazovaním jej zložiek na množinu $\{0,1\}$.
		Vy výsledku by kód konfigurácie siete $(S = (A,P,K), stav)$ mohol vyzerať ako $\# \langle A \rangle \# \langle P \rangle \# \langle K \rangle \# \langle stav \rangle \#$,
		kde $\langle A \rangle$ je kód automatu $A$, ktorý vznikne obdobne ako kód TS podľa podkapitoly 6.2.1 v študíjnej opore, kde sa vynechá časť o kódovaní
		akcie $y \in \Gamma \cup \{L,R\}$, ktorú automat nemá. Kódom $\langle P \rangle$ sa rozumie ľubovoľne zakódovaná množina procesov, tak aby bolo možné
		jednoznačne rozlíšiť procesy, ktoré sa v sieti nachádzajú. Kód $\langle K \rangle$ je kódom komunikačných kanálov z ktorého musí byť možné jednoznačne určiť,
		ktoré procesy sú spojené s ktorými procesmi komunikačnými kanálmi 1 a 2. Kód zobrazenia $\langle stav \rangle$ je ľubovoľné kódovanie zobraznia $stav$,
		aby bolo možné jednoznačne priradiť každému procesu jeho stav. 

		Podľa vety 6.4.1 zo študíjnej opory, problém \emph{MP} nie je problém rozhodnuteľný, ale je čiastočne rozhodnuteľný. Redukciou z problému \emph{MP} na problém \emph{RNP} myslíme
		zobrazenie $\sigma: \{0,1,\#\}^{*} \to \{0,1,\#\}^{*}$, pričom platí $\forall x \in \{0,1,\#\}^{*}: x \in \text{\emph{MP}} \Leftrightarrow \sigma(x) \in \text{\emph{RNP}}$.
		Zostrojíme teda úplný TS $M_{\sigma}$, ktorý bude vykonávať zobrazenie $\sigma$ nad svojim vstupom. Základna idea redukcie je vyfiltrovať najskôr vstupy tvaru
		$x_{1}\#x_{2}$, kde $x_{1} \in \{0,1\}^{*}$ je platným kódom TS a $x_{2} \in \{0,1\}^{*}$ platným kódom jeho vstupu. Následne vytvoriť toľko procesov, aká je
		dĺžka vstupu $x_{2}$. Stavy týchto procesov budú odpovedať páskovej abecede TS $x_{1}$. Počiatočným stavom bude stav $\Delta$.
		Proces $p_{i}$ bude mať stav určený symbolom $a_{i}$ zo vstupného reťazca $x_{2}$.
		Pre prípad simulácie pozície hlavy a stavu TS avšak zavedieme ešte ďaľšie možné stavy automatu $A$, ktoré budú odpovedať dvojiciam $(q,a)$, kde $q$ je stav TS $x_{1}$
		a $a$ je symbol páskovej abecedy TS $x_{1}$. Proces $p_{i}$ v takomto stave $(q,a)$ značí, že simulovaný TS $x_{1}$ v sieti má hlavu na pozícií $i$ a stav tohto
		TS je $q$. Množina koncových stavov je daná stavmi $(q_{F}^{x_{1}}, a)$ pre ľubovolné $a$, pričom $q_{F}^{x_{1}}$ odpovedá koncovému stavu TS $x_{1}$.
		Komunikačné kanály sú definované medzi susednými procesmi, čo odpovedá znakom vo vstupe $x_{2}$, ktoré idú bezprostredne za sebou. Proces $p_{i}$
		je spojený kanálom 1 s procesom $p_{i-1}$ (výnimkou je $p_{1}$) a kanálom 2 s procesom $p_{i+1}$ (výnimkou je proces $p_{|x_{2}|}$). Každý proces je zároveň
		spojený sám so sebou kanálom 1. Výpočetný krok siete simuluje výpočetný krok TS $x_{1}$ tak, že komunikujúcim procesom je vždy proces v stave
		simulujúcom pozíciu hlavy a stav TS, teda v stave $(q,a)$. Komunikujúcou správou je $a$, kanál a aj proces, s ktorým bude komunikácia prebiehať je vybraný nedeterministicky.
		Nedeterminizmus postačuje, pretože je požadovaná existencia sekvencie výpočetných krokov.

		Presný a formálny popis fungovania $M_{\sigma}$ je potom nasledovný.
		\begin{itemize}
			\item $M_{\sigma}$ má v sebe zapísaný konštantný úplný TS $M_{RN}$, ktorý je na jeho výstupe vždy pre ľubovoľný vstup $x \in \{0,1,\#\}^{*}$.
				Jedná sa o stroj, ktorý dokáže simulovať rendez-vous sieť na vstupe $\langle (S, stav) \rangle$ a to nasledovne.
				\begin{itemize}
					\item Pokiaľ je na vstupe $M_{RN}$ neplatný kód konfigurácie siete, tak stroj odmietne.
					\item Stroj vyberie proces $p \in P$, pre ktorý platí, že $stav(p) = (q,a)$, nedeterministicky zvolí
						kanál 1 alebo 2 a proces s ktorým bude komunikovať. Následne prebehne komunikácia so správou $a$
						medzi timito procesmi, ako je definované v zadaní. Stroj vykonáva túto činnosť opakovane, pokiaľ
						nebude platiť, že $\exists p \in P: stav(p) \in F$. Potom príjme. Pokiaľ sa nikdy do koncovej konfigurácie nedostane,
						tak odmietne.
				\end{itemize}
			\item $M_{\sigma}$ má na vstupe $x \in \{0,1,\#\}^{*}$. Pokiaľ $x$ nie je v tvare $x_{1}\#x_{2}$, kde $x_{1},x_{2} \in \{0,1\}^{*}$,
				či je v tvare $x_{1}\#x_{2}$, ale $x_{1}$ nie je platným kódom TS alebo $x_{2}$ nie je platným kódom jeho vstupu,
				tak $M_{\sigma}$ zostrojí taký kód konfigurácie sieťe $\langle (S,stav) \rangle$, že $P = \varnothing$
				a kód TS $\langle M_{x} \rangle$, ktorý odmietne čokoľvek má na vstupe.
			\item Pokiaľ vstup $x$ je v tvare $x_{1}\#x_{2}$ a $x_{1}$ je platný kód TS a $x_{2}$ platný kód jeho vstupu, tak $M_{\sigma}$
				zostrojí kód konfigurácie siete $\langle (S, stav) \rangle$ nasledovne.
				\begin{itemize}
					\item Nech $x_{1}$ je kód nejakého TS $M = (Q_{M}, \Sigma_{M}, \Gamma_{M}, \delta_{M}, q_{0}^{M}, q_{F}^{M})$ a
						$x_{2}$ je kód jeho vstupu $w$ (teda $x_{1}\#x_{2} = \langle M \rangle \# \langle w \rangle)$.
					\item Automat $A = (Q, \Sigma \times \{1,2\}, \delta, q_{0}, F)$ sa zostrojí nasledovne
						\begin{itemize}
							\item $Q = \Gamma_{M} \cup (Q_{M} \times \Gamma_{M})$ pokiaľ $\Gamma_{M} \cap Q_{M} = \varnothing$, teda sú disjunktné.
								Pokiaľ nie, zvolíme premenovanie stavov $Q_{M}$ tak aby $\Gamma_{M}$ a $Q_{M}$ boli disjunktné.
							\item $\Sigma = \Gamma_{M}$
							\item $\delta$ sa zostrojí postupne podľa algoritmu:
								\begin{itemize}
									\item Pokiaľ pre nejaké $q,p \in Q_{M}$ a $a \in \Gamma_{M}$ platí, že $(p,R) \in \delta_{M}(q,a)$, tak
										do $\delta((q,a),(a,2))$ pridaj $a$ a pre každé $b \in \Gamma_{M}$ do $\delta(b,(a,2))$ pridaj $(p,b)$.
									\item Pokiaľ pre nejaké $q,p \in Q_{M}$ a $a \in \Gamma_{M}$ platí, že $(p,L) \in \delta_{M}(q,a)$, tak
										do $\delta((q,a),(a,1))$ pridaj $a$ a pre každé $b \in \Gamma_{M}$ do $\delta(b,(a,1))$ pridaj $(p,b)$.
									\item Pokiaľ pre nejaké $q,p \in Q_{M}$ a $a,b \in \Gamma_{M}$ platí, že $(p,b) \in \delta_{M}(q,a)$, tak
										do $\delta((q,a),(a,1))$ pridaj $(p,b)$.
								\end{itemize}
							\item $q_{0} = \Delta$
							\item $F = \{(q,a) \in (Q_{M} \times \Gamma_{M}) \pipesep q = q_{F}^{M}\}$
						\end{itemize}
					\item Množinu procesov $P$ zostrojíme tak, že zapíšeme $w = a_{1}...a_{n}$, kde $a_{i} \in \Sigma_{M}$ pre $1 \le i \le n$.
						Pre každý symbol $a_{i}$ vytvoríme proces $p_{i}$ a umiestnime do množiny $P$. Pokiaľ $w = \varepsilon$, tak $P = \varnothing$.
					\item Množinu kanálov $K$ zostrojíme tak, že zapíšeme $w = a_{1}...a_{n}$, kde $a_{i} \in \Sigma_{M}$ pre $1 \le i \le n$.
						Pre každú dvojicu $a_{i}$ a $a_{i+1}$ pre $1 \le i < n$ pridaj do $K$ kanály
						$(p_{i}, 1, p_{i}), (p_{i}, 2, p_{i+1}), (p_{i+1}, 1, p_{i})$ a $(p_{i+1}, 1, p_{i+1})$.
					\item Zobrazenie $stav$ zostrojíme tak, že každému procesu $p_{i} \in P$, priradíme stav podľa hodnoty symbolu $a_{i}$ na vstupe $w$.
						Procesu $p_{1}$ ako jedinému priradíme stav $(q_{0}, a_{1})$, čo značí pozíciu hlavy a stav TS $M$.
				\end{itemize}
			\item $M_{\sigma}$ taktiež zostrojí kód TS $\langle M_{x} \rangle$ = $\langle M_{x_{1}\#x_{2}} \rangle$. Tento TS bude mať na vstupe
				kód konfigurácie siete $\langle (S,stav) \rangle$, ktorý vznikol v $M_{\sigma}$. $M_{x}$ bude pozostávať z 2 častí
				\begin{itemize}
					\item Kontrola prítomnosti počiatočného stavu TS daného kódom $x_{1}$ (kód tohto stavu je konštantne zapisaný v $M_{x}$) v stave procesu
						$p_{1}$, čo odpovedá kontrole počiatočnej konfigurácie pásky a simulovaného TS daného kódom $x_{1}$. Pokiaľ vstup neprejde kontrolou,
						tak $M_{x}$ odmietne.
					\item Stroja $M_{RN}$, ktorý je spustený pokiaľ vstupný kód konfigurácie siete prejde úvodnou kontrolou. Pokiaľ simulácia $M_{RN}$
						na $\langle (S,stav) \rangle$ skončí a $M_{RN}$ príjme, tak $M_{x}$ taktiež príjme. Pokiaľ simulácia $M_{RN}$ skončí, ale $M_{RN}$
						nepríjme, prípadne simulácia cyklí, tak $M_{x}$ odmietne.
				\end{itemize}
			\item Ukážeme, že TS $M_{\sigma}$ je naozaj úplný a je ho možné zostrojiť. Postačí totiž aby $M_{\sigma}$ zvládal nasledujúce činnosti.
				\begin{itemize}
					\item Rozoznať, či je jeho vstup $x \in \{0,1,\#\}^{*}$ v tvare $x_{1}\#x_{2}$, $x_{1},x_{2} \in {0,1}*$, čo odpovedá zisteniu príslušnosti
						do regulárneho jazyka.
					\item Zakódovanie konfigurácie siete z kódu TS $x_{1}$ na vstupe. Spojenie $\Gamma_{M}$ a $Q_{M}$ je možné vyriešiť napríklad prekódovaním
						stavov z $Q_{M}$ konkatenáciou s najdlhším kódovaním symbolu z $\Gamma_{M}$, čo je jednoducho algoritmizovateľná úloha. Zostavenie $\gamma$
						je opäť zisťovanie príslušnosti do regulárneho jazyka. Zakódovanie procesov a kanálov je len priraďovanie nových kódov, čo je jednoducho algoritmizovateľné.
					\item Zostrojenie $M_{x}$, ktorý buďto vždy abnormálne zastaví, alebo $M_{x}$, ktorý zistí či proces $stav(p_{1})$ odpovedá dvojici $(q,a)$ pre
						ľubovoľné $a$ a pre $q = q_{0}^{M}$ a následné spustenie simulácie $M_{RN}$ na vstupe $\langle (S,stav) \rangle$.
				\end{itemize}
			\item Dostávame, že platí
				\begin{align*}
					\langle (S,stav) \rangle &\in \text{\emph{RNP}}\\
					&\Leftrightarrow \\
					M_{x}\text{ príjme } &\langle (S,stav) \rangle \\
					&\Leftrightarrow \\
					M_{RN}\text{ príjme } &\langle (S,stav) \rangle\ \land \\
					x =\ &x_{1}\#x_{2}\ \land \\
					\text{ TS daný kódom }x_{1}\text{ je v počiatočnom}&\text{ stave a hlava je na najlavejšej pozícií} \\
					&\Leftrightarrow \\
					(S,stav)\text{ je zakódovaný TS }&x_{1}\text{ so vstupom }x_{2} \\
					&\Leftrightarrow \\
					\text{TS daný kódom }x_{1}\text{ prí}&\text{jme vstup daný kódom }x_{2} \\
					&\Leftrightarrow \\
					x_{1}\#x_{2} &\in \text{\emph{MP}}
				\end{align*}
		\end{itemize}
	\end{mysolution}

\end{document}
